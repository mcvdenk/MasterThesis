\chapter{Conceptual Framework}

    \section{Concept maps}

        \subsection{Comparison to other visual mapping techniques}

        Concept maps are not the only type of visual mapping techniques, \citeA{eppler} distinguishes four different types of visual mapping techniques. These types are quite similar to each other, and therefore the differences between them will be elaborated further. Table~\ref{tab:vmtechs} displays the different types and there specific characteristics.

        \begin{table}[]
            \centering
            \begin{tabular}{p{.20\textwidth}p{.20\textwidth}p{.20\textwidth}p{.20\textwidth}}
                \hline
                                  & Concept map                                                                                                                         & Mind map                                                                                                                                            & Conceptual diagram                                                                                                                             \\ \hline
                Definition                        & A top-down diagram showing the relationships between concepts, including cross connections among concepts, and their manifestations & A multicoloured and image-centred, radial diagram that represents semantic or other connections between portions of learned material hierarchically & A systematic depiction of an abstract concept in pre-defined category boxes with specified relationships, typically based on a theory or model \\
                Main function or benefit          & To show systematic relationships among sub-concepts relating to one main concept                                                    & To show sub-topics of a domain in a creative and seamless manner                                                                                    & To analyse a topic or situation through a proven analytic framework                                                                            \\
                Macro structure adaptability      & Flexible, but always branching out                                                                                                  & Somewhat flexible, but always radial                                                                                                                & Fixed                                                                                                                                          \\
                Lelvel of difficulty to construct & Medium to high                                                                                                                      & Low                                                                                                                                                 & Medium to high                                                                                                                                 \\
                Extensibility                     & Limited                                                                                                                             & Open                                                                                                                                                & Limited                                                                                                                                        \\
                Memorability                      & Low                                                                                                                                 & Medium to high                                                                                                                                      & Low to medium                                                                                                                                  \\
                Understandability by others       & High                                                                                                                                & Low                                                                                                                                                 & Medium                                                                                                                                         \\ \hline
            \end{tabular}
            \caption{A comparison between different concept mapping techniques, as described by \protect\citeA{eppler}}
            \label{tab:vmtechs}
        \end{table}

        \subsection{Novakian concept map}

        \subsection{Conditions}

        \subsection{Research}

            \paragraph{Effectiveness}

            \paragraph{Attitudes}

    \section{Paired-Associate Learning}

        \subsection{Testing Effect}

        \subsection{Flashcards}

        \subsection{Research}

            \paragraph{Effectiveness}

            \paragraph{Student attitudes}

        \subsection{Criticism}
