\chapter{Abstract}

Modern day society requires students to memorise and understand a large number of facts. Currently, a powerful learning tool for comprehension is concept mapping, which entails drawing meaningful relations between concepts in an associative network. For rote memorisation, flashcard systems are widely employed, which entails students going repeatedly through a set of questions and recalling their answers from memory. Critics have found concept mapping not entailing any method for retaining facts in memory, where others found flashcard learning extracting all meaning and context from the learning process. Therefore, a new learning tool is developed within this study, aiming to bridge the gap between aforementioned learning tools by integrating the visualisation of concept maps within the retrieval mechanism of flashcard learning. The new tool is an augmentation on the flashcard system, which asks the students to fill in empty concepts within given concept maps instead of to answer provided questions. Since retrieval practices --- such as flashcard learning --- have already been found to provide more meaninful learning than concept mapping in a recent study, the new tool is compared with a generic flashcard system for knowledge retention and comprehension. Furthermore, the usefulness and ease of use perceived by the participants are compared. Because of a low response rate, the results from the comparisons are only indicatory for further research. These results include the users of the new tools having a higher learning gain on knowledge retention questions than the flashcard users, and there being no significant difference in increased comprehension, perceived usefulness, and perceived ease of use of the systems.
