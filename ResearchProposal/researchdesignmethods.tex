\chapter{Research Design and Methods}

\section{Research design}

%What kind of research is this? (e.g. exploratory, confirmatory, intervention-based, evaluationbased, design-based, etc.). Justify the type of research design(s) you intend to use :descriptive (e.g., case study) ;correlational (longitudinal) ; (quasi-)experimental; review etc.

\section{Respondents}

%This section is where you describe the who will be approached to participate in your study and how many. Explain how they will be selected (sampling method). Make sure this justification fits your research design and questions.

\section{Instrumentation}

%Describe the instruments you will use; make a link between the research variables and the instruments explicit (operationalization process) and describe their measurement level if applicable. If you are using existing instruments, provide references.

\section{Procedure}

%Describe the procedure for your data collection; what will respondents in your study do? Here you can also address any constraints you may need to cope with in the research and the actions to guard its quality and validity. Potential ethical concerns can be addressed here too.

\section{Data Analysis}

%Describe the type of data you will generate (qualitative, quantitative, mixed-method) and the methods you intend to use to analyze your data. Make sure this fits with your research design and questions. Here you may also address reliability checks for your instrumentation.
