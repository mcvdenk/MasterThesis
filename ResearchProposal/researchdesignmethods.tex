\chapter{Research Design and Methods}

\section{Research design}

%What kind of research is this? (e.g. exploratory, confirmatory, intervention-based, evaluationbased, design-based, etc.). Justify the type of research design(s) you intend to use :descriptive (e.g., case study) ;correlational (longitudinal) ; (quasi-)experimental; review etc.

Research questions \ref{benefits}\ref{effectiveness}, \ref{efficiency}, \ref{perception}\ref{usefulness} and \ref{ease} will be investigated using intervention-based research. Because of the systems being used for self-study by the students, they can be individually assigned to a condition, and this enables the use of a true experimental design. Since this will provide the most valid and reliable results, this research design is implemented in this experiment.

Furthermore, research question~\ref{howused} is a qualitative research question, and could be measured by observing participants or interviewing them afterwards. However, an observation would not be feasible, since the researcher would be required to follow the students around in their daily lives waiting for them to use the system requiring a high amount of effort with only a low benefit to the research. Furthermore, any involvement of the researcher would influence the participants and thereby create a bias in the results. Therefore, only an interview will trake place after the experiment in order to investigate how the students used the system.

MIXED METHOD PURPOSE
    TRIANGULATION
    EXPANSION
    POINT OF ADDITION
    POINT OF INTEGRATION
    CONCURRENT
    PARTIALLY
    QUAN DOMINANT

\section{Respondents}

%This section is where you describe the who will be approached to participate in your study and how many. Explain how they will be selected (sampling method). Make sure this justification fits your research design and questions.

%1.5.4

10TH GRADE HIGHSCHOOL STUDENTS

DUTCH, STEDELIJK LYCEUM ENSCHEDE

VOLUNTARY PARTICIPATION

RANDOM SAMPLING

BONUS POINT

INTRODUCTION USEFULNESS

\section{Instrumentation}

%Describe the instruments you will use; make a link between the research variables and the instruments explicit (operationalization process) and describe their measurement level if applicable. If you are using existing instruments, provide references.

%n1.5.6

LEARNING GAIN
    LEARNING GAIN CONTROLLED
    PRE TEST
    POST TEST
    BASED ON CONCEPT MAP
    TAXONOMY BLOOM
    ITEM BANK

SURVEY
    TAM

INTERVIEW

\section{Procedure}

%Describe the procedure for your data collection; what will respondents in your study do? Here you can also address any constraints you may need to cope with in the research and the actions to guard its quality and validity. Potential ethical concerns can be addressed here too.

%n1.6

INFORMED CONSENT

Firstly, the students will be provided with a general introduction on flashcards by both the teacher and the researcher within the classroom. Then, when the students log into the system for the first time, the server assigns them randomly to either the flashcard or the flashmap condition. By making the introduction ambiguous enough, the students will not be able to recognise this condition in order to guarantee a double-blind experiment.

PRE-TEST

SYSTEM USE
    EXPERIMENT DURATION
    DAILY WORKLOAD

POST-TEST

SURVEY

INTERVIEW

BONUS POINT

\section{Data Analysis}

%Describe the type of data you will generate (qualitative, quantitative, mixed-method) and the methods you intend to use to analyze your data. Make sure this fits with your research design and questions. Here you may also address reliability checks for your instrumentation.

T-TEST
    BENEFITS
        LEARNING GAIN
        LEARNING GAIN CONTROLLED
    PERCEPTION
        USEFULNESS
        EASE

INTERVIEW
    CODING
