\chapter{Project Description}

\section{Problem Statement}

%Describe a rationale for the focus and aim of this study; why is the topic of your research important? This can be done from a theoretical, as well as from a practical point of view (or both). For example, what gaps exist in current literature, what problems need to be solved, What tool/intervention do we need, what societal changes have led to a need for this research?

%IDEAL AFFAIRS

Within the history of educational psychology, three major distinct perspectives of the learning process have been proposed, namely the behavioural, the cognitive and the constructivist perspective \cite{ertmer}. Learning theory first shifted from behaviourist models to cognitivist models, resulting in a change in focus on observable performance to what is happening within the learner itself. However, both these theories still assume a primarily objectivistic world view, whereas the final perspective of constructivism offer an alternative, relativistic world view. Within this model, the student is not supplied with information, but instead has to construct his own model of reality. \citeA{glaserfield} however criticised constructivism for its lack of rote memorisation, and argues for a need to train students so that they permanently possess facts and are able to repeat them flawlessly whenever they are needed, while also understanding what is placed into their memory.

%introduction flashcards

%n1.1.7, n1.1.3
A currently existing method to efficiently rote memorise information is the flashcard system, where declarative knowledge is studied in a paired-associate format. Within this format, learners are asked to associate terms with other terms outside meaning-focused tasks, for example by associating a definition with a presented concept \cite{nakata}. With flashcards, large numbers of words can be memorised in a very short time, and are more resistant to decay \cite{nakata, joseph}. \citeA{macquarrie} adds to this by stating that increasing the amount of drill or pracice is the most effective device that can be applied to learning. Finally, when evaluating flashcards in a psychology setting, it was found that students who use flashcards have a significantly higher final average than those who do not \cite{burgess, golding}.

%EXPLAIN THE PROBLEM

%n1.1.2, n1.1.3.2, n1.1.3.8, n1.5.9, n1.2.1.2
Per contra, not all research favours using flashcards for textual comprehension. \citeA{zirkle} states that flashcards are especially useful for learning declarative knowledge, while learning from a textbook is a form of learning for intellectual skills \cite{instructionaldesign}. This problem is also emphasised by \citeA{mccullough}, who states that the use of flashcards is helpful for language learning but the main emphasis of flashcards is memorisation, not comprehension. \citeA{zirkle} points out the overemphasis placed upon the rote memorisation of disconnected facts, whereas whatever it is that students are to place into memory they should, more importantly, understand. Furthermore, \citeA{hulstijn} describes flashcards as a relic of the old-fashioned behaviourist learning model, and states that we have to look for more modern constructivist models.

%EXPLAIN WHY THE PROBLEM IS IMPORTANT

%n1.1.1.7
Solving the aforementioned problem could lead to better understanding of memory, and could lead to better utilisation by teachers and students with the intent to produce a store of knowledge that remains flexibly retrievable in a variety of contexts over a period of time, in contrast to only segregated paired associations which depend on specific cues in order to be retrieved. Furthermore, it could pave the way for the design of new educational activities based on consideration of retrieval processes. Furthermore, using computer-based flashcards have been used very widely \cite{nakata}, and more recently textbooks have started making flashcards available on their websites \cite{burgess, golding}. \citeA{kornell} stated that "Perhaps no memorisation technique is more widely used than flashcards" (p. 125). Improving currently existing flashcards therefore has the potential of reaching a wide audience of future users of flashcard systems. Finally, it might be a solution to the need expressed by \citeA{glaserfield} for more meaningful rote memorisation.

%PROPOSE A SOLUTION/IDEA AND ITS BENEFITS

%introduction concept maps

%n1.2.1, n1.(2.1),(1.3).1
An instructional tool more in line with constructivistic approaches is the concept map, which is defined by \citeA{eppler} as a hierarchical diagram showing the relationship between concepts, including cross connections among concepts, and their manifestations (see figure~\ref{fig:conceptmap}. Multiple researchers have found by means of both qualitative and quantitative studies that concept maps can promote meaningful learning leading to positive effects on students \cite{hwang2, subramaniam, canas}. This has been demonstrated in comparison to activities such as reading text passages, attending lectures, and participating in class discussions \cite{singh, nesbit2}. \citeA{canas} describes the process of concept mapping as the only effective way of using the concept map, which refers to students constructing their own concept maps. This is why the concept map is generally viewed as a tool in alignment with the constructivist perspective. Because of this, the concept map might seem as a solution to the need asked by \citeA{glaserfield} and his peers. However, a recent article by \citeA{karpicke2} reveals that paired associate learning produced better performance than elaborative concept mapping for meaningful learning, even on the short-term.

%introduction flashmaps

%n1.2.6.8 and n1.2.6.9 (Counterarguments), n1.2.5, n1.2.10
Therefore, another solution might be the development of a new tool, namely the flashmap. The intention behind the flashmap is to combine the paired associate mechanism of the flashcard system with the visual represenation of the concept map, and is a new tool designed and developed for this research project. This tool might have the potential to bridge the gap between the two systems and therefore make meaningful and effective rote memorisation possible, for it makes the relations between the concepts explicit to the student.

%SUMMARISE, END WITH RESEARCH GOALS

In conlusion, flashcards systems are an effective tool for meaningful learning, but could be enhanced by visualising it with concept maps. The objective of this research is therefore to evaluate whether learning with a flashmap is a more effective or efficient for meaningful learning than flashcards, and whether it might be more affective.

\section{Theoretical Conceptual Framework}

%Describe theoretical framework where you introduce the most important concepts in your study and their relation

%FLASHCARDS

%n1.1.6

%CONCEPT MAPS

%n1.2.4
%n1.2.8
%n1.4.1.4

%FLASHMAPS

\citeA{canas} describes that fill-in-the-cmap or memorise the concept map conditions are not recommended, for meaningful learning does not work this way. However, they do not provide statistics or literature in order to support this claim, and furthermore the findings from \citeA{karpicke2} about paired associate learning being more effective for meaningful learning than concept mapping also puts this claim into doubt. Finally, the flashmap creates the opportunity for a more interactive concept map that starts with a parsimonious and theme-oriented structure which gradually expand the details along with the instruction, advised by \citeA{tzeng} to mitigate map schock. This phenomenon occurs when students view the kind of larger concept maps that might more fully capture textbook knowledge structures, but is a type of cognitive overload that prevents students from effectively processing the concept map and thereby inhibiting their ability to learn from it \cite{moore}.

\section{Research Question and Model}

%The research questions (hypotheses if applicable) and model are described here. A figure on your research model is optional.

\section{Scientific and Practical Relevance}

%Describe the expected contribution of your study (how can scientists, practitioners benefit)

%n1.1.3.1
%n2.1
