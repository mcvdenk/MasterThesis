\chapter{Project Description}

\section{Problem Statement}

%Describe a rationale for the focus and aim of this study; why is the topic of your research important? This can be done from a theoretical, as well as from a practical point of view (or both). For example, what gaps exist in current literature, what problems need to be solved, What tool/intervention do we need, what societal changes have led to a need for this research?

%IDEAL AFFAIRS

%Within the history of educational psychology, three major distinct perspectives of the learning process have been proposed, namely the behavioural, the cognitive and the constructivist perspective \cite{ertmer}. Learning theory first shifted from behaviourist models to cognitivist models, resulting in a change in focus on observable performance to what is happening within the learner itself. However, both these theories still assume a primarily objectivistic world view, whereas the final perspective of constructivism offer an alternative, relativistic world view. Within this model, the student is not supplied with information, but instead has to construct his own model of reality. \citeA{glaserfield} however criticised constructivism for its lack of rote memorisation, and argues for a need to train students so that they permanently possess facts and are able to repeat them flawlessly whenever they are needed, while also understanding what is placed into their memory.

%introduction flashcards

%n1.1.7, n1.1.3
A currently existing method to efficiently rote memorise information is the flashcard system, where declarative knowledge is studied in a paired-associate format. Within this format, learners are asked to associate terms with other terms outside meaning-focused tasks, for example by associating a definition with a presented concept \cite{nakata}. With flashcards, large numbers of words can be memorised in a very short time, and are more resistant to decay \cite{nakata, joseph}. \citeA{macquarrie} adds to this by stating that increasing the amount of drill or practice is the most effective device that can be applied to learning. Finally, when evaluating flashcards in a psychology setting, it was found that students who use flashcards have a significantly higher final average than those who do not \cite{burgess, golding}.

%EXPLAIN THE PROBLEM

%n1.1.2, n1.1.3.2, n1.1.3.8, n1.5.9, n1.2.1.2
Per contra, not all research favours using flashcards for textual comprehension. \citeA{zirkle} states that flashcards are especially useful for learning declarative knowledge, while learning from a textbook is a form of learning for intellectual skills \cite{instructionaldesign}. This problem is also emphasised by \citeA{mccullough}, who states that the use of flashcards is helpful for language learning but the main emphasis of flashcards is memorisation, not comprehension. \citeA{zirkle} points out the overemphasis placed upon the rote memorisation of disconnected facts, whereas whatever it is that students are to place into memory they should, more importantly, understand. Furthermore, \citeA{hulstijn} describes flashcards as a relic of the old-fashioned behaviourist learning model, and states that we have to look for more modern constructivist models.

%EXPLAIN WHY THE PROBLEM IS IMPORTANT

%n1.1.1.7
Solving the aforementioned problem could lead to better understanding of memory, and could lead to better utilisation by teachers and students with the intent to produce a store of knowledge that remains flexibly retrievable in a variety of contexts over a period of time, in contrast to only segregated paired associations which depend on specific cues in order to be retrieved. Furthermore, it could pave the way for the design of new educational activities based on consideration of retrieval processes. Furthermore, using computer-based flashcards have been used very widely \cite{nakata}, and more recently textbooks have started making flashcards available on their websites \cite{burgess, golding}. \citeA{kornell} stated that "Perhaps no memorisation technique is more widely used than flashcards" (p. 125). Improving currently existing flashcards therefore has the potential of reaching a wide audience of future users of flashcard systems. Finally, it might be a solution to the need expressed by \citeA{glaserfield} for more meaningful rote memorisation.

%PROPOSE A SOLUTION/IDEA AND ITS BENEFITS

%introduction concept maps

%n1.2.1, n1.(2.1),(1.3).1
An instructional tool more in line with constructivistic approaches is the concept map, which is a graph consisting of nodes representing concepts and labeled lines denoting the relation between a pair of nodes \cite{ruiz1} (see figure~\ref{fig:conceptmap}. Multiple researchers have found by means of both qualitative and quantitative studies that concept maps can promote meaningful learning leading to positive effects on students \cite{hwang2, subramaniam, canas}. This has been demonstrated in comparison to activities such as reading text passages, attending lectures, and participating in class discussions \cite{singh, nesbit2}. \citeA{canas} describes the process of concept mapping as the only effective way of using the concept map, which refers to students constructing their own concept maps. This is why the concept map is generally viewed as a tool in alignment with the constructivist perspective. Because of this, the concept map might seem as a solution to the need asked by \citeA{glaserfield} and his peers. However, a recent article by \citeA{karpicke2} reveals that paired associate learning produced better performance than elaborative concept mapping for meaningful learning, even on the short-term.

%introduction flashmaps

%n1.2.6.8 and n1.2.6.9 (Counterarguments), n1.2.5, n1.2.10
Therefore, another solution might be the development of a new tool, namely the flashmap system. The intention behind the flashmap system is to combine the paired associate mechanism of the flashcard system with the visual representation of the concept map, and is a new tool designed and developed for this research project. This tool might have the potential to bridge the gap between the two systems and therefore make meaningful and effective rote memorisation possible, for it makes the relations between the concepts explicit to the student.

For evaluating this flashmap system, a group of Dutch highschool teachers of the Stedelijk Lyceum has been found willing to participate, with their students using either the flashmap or the flashcard system for self study parallell with classroom instruction. The content of the instruction will be the history of Dutch literature during the sixteenth and seventeenth century. For example, the students have to learn what the influence is of the Dutch War of Independence on the \emph{Spaanschen Brabander} by Bredero. Because of the content existing mainly of concepts with meaningful relations it fits to the concept map technique and thereby the flashmap system could be significantly beneficial over the flashcard system.

%SUMMARISE, END WITH RESEARCH GOALS

In conlusion, flashcards systems are an effective tool for meaningful learning, but could be enhanced by visualising it with concept maps, and therefore the effects of using a flashmap system over using a flashcard system will be investigated. 

\section{Theoretical Conceptual Framework}

%Describe theoretical framework where you introduce the most important concepts in your study and their relation

\subsection{Flashcards systems}

%n1.1.6, n1.1.1.8.6.1, n1.1.1.3.8.2
There are many different flashcard systems, varying in scheduling algorithms \cite{microlearning}, and offline or online applications \cite{nakata}. The simplest and earliest example is a deck of physical cards, with on one side a question and on the other side the answer to that question. Every day, the student has to go through the deck trying to answer the question on the card. After answering it, the student turns around the card to check whether was correct. If the answer was correct the card goes to the deck for the next day, and if incorrect the card goes to the bottom of the current day's deck.

The main disadvantage of this system is that it becomes time-intensive when more flashcards are introduced, because the student has to go through all of the cards every day. Because of this, newer systems relying on spaced repetition were introduced, with which the time intervals between repetitions increase every time the student answers correctly. \citeA{microlearning} describes three different types of spaced algorithms, namely progressive, responsive, and adaptive. Within progressive algorithms, the rescheduling of cards are always increasing. Responsive algorithms reset the time interval of a card every time the student makes a mistake. Finally, adaptive systems vary the base increase value of the time interval in order to raise success rates towards a given percentage, meaning that the chance of answering the card correctly is estimated to be equal to that percentage. It was found that the last strategy was more effective and more satisfactory to the user than the other strategies \cite{microlearning}.

Furthermore, the transition from physical to digital flashcards is worthwile to consider. The previously described algorithms can more efficiently be conducted by a computer, since it is able to keep track of a learner's performance and control the sequence of items which can be cumbersome if done manually \cite{nakata}. Furthermore, many students have smartphones with them most of the time, and are more convenient than stacks of traditional flashcards \cite{nakata}. The ownly downside to using digital flashcards is that they are less frequently used than traditional flashcards \cite{burgess}. Reasons for this are technical issues, simply forgetting about it, distraction by entertainment apps and preference for traditional flashcards.

The effects of flashcards have mainly been attributed to the spacing effect \cite{nakata, microlearning}, which means that repeated items are better remembered when both occurrences are separated by other events or items than when they are presented in immediate succession \cite{verkoeijen, logan, siegel, xue, karpicke2}.

\subsection{Concept maps}

%n1.2.4, n1.2.8
According to \citeA{eppler}, a concept map is a hierarchical graph showing the relationships between concepts, including cross connections among concepts and their manifestations. The edges contain labels describing the relation between the concepts. They compare the concept map to several different other visual mapping techniques, which are the mind map, the conceptual diagram and the visual metaphor. A mind map is multicoloured and image-centred, is radial and represents semantic or other connections between portions of learned material hierarchically. The benefit of constructing a mind map is that it has a higher memorability than a concept map, but is more difficult to understand by others \cite{eppler}. The conceptual diagram entails abstract concepts in pre-defined category boxes with specified relationships, typically based on a theory or model. This diagram is more suitablefor analysing topics or situations through a proven analytic framework, however they have only a medium memorability and medium understandability by others. Finally, a visual metaphor is a praphic structure using the shape and elements of a familiar artiefact, activity, or story to organise content meaningfully and use the associations with the metaphor to convey additional meaning about the content. This technique is the most meaningful, memorable and understandable in comparison to the other technique, however it has a very limited extensibility \cite{eppler}.

%1.2.6
%\citeA{canas} describes several conditions for using concept maps in learning: no conditions, focus question, root concept, list of concepts, restricting list of concepts, expert skeleton concept maps, concept mapping games, fill-in-the-Cmap and memorise the concept map. These concitions range from low to high scaffolding respectively. The no conditions condition asks students to create a concept map without any scaffolding. This often results in a descriptive instead of an explanatory map, and users are often intimidated and have difficulty constructing a concept map. A focus question helps student to create a more explanatory map by trying to answer specific questions. This has a positive effect on the quality of the resulting maps. A root concept provides the students with a minimum concept map containing the main topics, having a stronger effect than the focus question. The list of concepts condition entails that students get a list of concepts they can use for the construction of their map, resulting in better maps than under no conditions or when provided with a text including the topics. The restrincting list of concepts entails that the students are only allowed to use concepts from the list, which is an effective way of determining the student's prior knowledge. The expert skeleton map restricts the freedom of content and sturture, and overcomes the difficulty of students when getting started. Within coyncept mapping games students are iteratively provided with two concepts and are asked to label the links between them, this has proven to be both effective and affective. Filling in the cmap means that the students are provided with a concept map containing empty nodes which they have to fill in themselves. This technique is not recommended, because this is meant for rote learning purposes and not for meaningful learning. Finally, Memorising the concept map is not recommended for its lack of meaningful learning and integration with other relevant knowledge. It also undermines the need for learners to be actively engaged in assimilating new concepts and propositions into their cognitive structures.

\subsection{Flashmaps}

The flashmap is intended as an integration between the flashcard system and the concept map. The system uses a predefined concept map constructed by an expert which the users have to rote memorise. Concept maps are chosen here, because it has the best combination of understandability and extensibility, and the memorability is facilitated by the flashcard system already. Where a flashcard system would then show a question, the flashmap shows a part of this concept map, where one or more nodes are empty. The user has to think of which concepts would fit in these nodes, and when requesting the answer the flashmaps shows the actual nodes. The student then can indicate per node whether he had it right or not, and they will be rescheduled for review according to the adaptive scheduling algorithm (see figure~\ref{fig:flashmap}. \citeA{canas} describes that fill-in-the-cmap or memorise the concept map conditions are not recommended, because of the information in memorised concept maps not being integrated with other relevant knowledge and the lack of learners being actively engaged in assimilating new concepts and propositions into their cognitive structures. However, they do not provide statistics or literature in order to support this claim, and furthermore the findings from \citeA{karpicke2} about paired associate learning being more effective for meaningful learning than concept mapping also puts this claim into doubt.

Finally, the flashmap creates the opportunity for a more interactive concept map that starts with a parsimonious and theme-oriented structure which gradually expand the details along with the instruction, advised by \citeA{tzeng} to mitigate map shock. This phenomenon occurs when users view the kind of larger concept maps that might more fully capture textbook knowledge structures, but is a type of cognitive overload that prevents students from effectively processing the concept map and thereby inhibiting their ability to learn from it \cite{moore}. This mitigation will be facilitated by scheduling the central concepts towards the beginning and the details towards the end.

\section{Research Question and Model}
\newcounter{researchquestion}
\renewcommand{\theresearchquestion}{\Roman{researchquestion}}
\newcounter{subquestion}[researchquestion]
\renewcommand{\thesubquestion}{\alph{subquestion}}


%The research questions (hypotheses if applicable) and model are described here. A figure on your research model is optional.

For researching the effects of the flashmap system relative to the effects of the flashcard system, it is important to consider two main factors: its actual benefits (research question~\ref{benefits}\ref{effectiveness} and~\ref{efficiency}), and its perceived benefits (research question~\ref{perception}\ref{usefulness} and~\ref{ease}). Furthermore, for the validity of the system and of the experiment it is important to investigate how the system was used by the students (research question~\ref{howused}).

To research whether the flashmap system is more effective or efficient than the flashcard system, the learning gain of high school the students will be measured, referring to the knowledge obtained by a student over the course of an instruction. Sequentially, the efficiency of the system is determined by the learning gain controlled for time spend on the system.

For measuring the affectiveness of the systems, the Technology Acceptance Model by \citeA{tam} will be used (see figure~\ref{fig:tam}). This model predicts the use of an information system by measuring the Perceived Usefulness and the Perceived Ease of Use of the user. These variables are mediators between External Variables and Attitude toward using, leading to Behavioural intention to use, which in turn leads to the Actual system use.

This leads to the following research questions: Regarding highschool students learning for Dutch literature using the flashmap system in comparison to them using the flashcard system...

\refstepcounter{researchquestion}\label{benefits}
\refstepcounter{subquestion}\label{effectiveness}
\Roman{researchquestion}\alph{subquestion}. ...is the learning gain larger?

\refstepcounter{subquestion}\label{efficiency}
\Roman{researchquestion}\alph{subquestion}. ...is the learning gain larger controlled for the time spend with the system?

\refstepcounter{researchquestion}\label{perception}
\refstepcounter{subquestion}\label{usefulness}
\Roman{researchquestion}\alph{subquestion}. ...do they perceive the system to be more useful?

\refstepcounter{subquestion}\label{ease}
\Roman{researchquestion}\alph{subquestion}. ...do they perceive the system to be easier to use?

\refstepcounter{researchquestion}\label{howused}
\Roman{researchquestion} How did the students use the flashmap or flashcard system?

\section{Scientific and Practical Relevance}

%Describe the expected contribution of your study (how can scientists, practitioners benefit)

%n1.1.3.1
%n2.1
