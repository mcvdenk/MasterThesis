\chapter{Planning}

\section{Timeline}

%Include an overview for whole project

Table~\ref{tab:planning} displays the activities together with the intended week number of execution and generated outputs. Before the start of this research, quite some effort has already been invested in the development of the flashmap software to a near-complete state. From this point onward, in week 16 a request will be send to the ethical committee for the approval of the research. Then, the concept map, flashcards and pre- and post-test will be finalised in week 17. This will also be send to the teacher for evaluation in week 18. In the mean time, the Technology Acceptance Model questionnaire will be adapted to this particular research needs, and work will be conducted towards the finalisation of the required software of week 19. In this week, the students will receive their first instruction on the topic provided by the teacher, and will also receive the letter containing the briefing and informed consent form. This instruction will be followed up in week 20, which will also be the start of the experiment with the introduction, pre-test and the experiment itself. The post-test and the questionnaire will be conducted in week 21, and in 22 the students will perform the actual test. During this week, the sample items for determining the inter-rater reliability will be scored by the researcher and the teacher. The interviews are planned to take place in 23, however this is flexible. This week the inter-rater reliability will be determined as well. In week 25, all items will be scored. The interviews will be transcribed in week 27 and coded in week 28. In week 29, the t-tests will be conducted, after which they will be interpreted and documented and mixed with the results of the server logs and interviews for triangulation and expansion purposes. Finally, in week 32 the research report will be finalised.

\setcounter{table}{0}

\begin{table}
    \begin{tabular}{ l | p{.4\textwidth} | p{.4\textwidth}}
        \centering
        \textbf{Week \textnumero} & \textbf{Activity} & \textbf{Outputs} \\ \hline
        16 & Request approval ethical committee & Approval ethical committee \\ \hline
        17 & Finalising concept map & Concept map \\ \hline
        17 & Finalising flashcards & Flash cards \\ \hline
        17 & Finalising item bank questions & Item bank questions \\ \hline
        18 & Finalising adaption Technology Acceptance Model questionnaire & Adapted Technology Acceptance Model questionnaire \\ \hline
        18 & Evaluation concept map with teacher & Improved concept map \\ \hline
        18 & Evaluation flashcards with teacher & Improved flashcards \\ \hline
        18 & Evaluation item bank questions with teacher & Improved item bank questions \\ \hline
        19 & Finalising software & Flashcard and Flashmap software \\ \hline
        19 & Briefing and informed concent & Approval parents and children \\ \hline
        19 & Instruction I & \\ \hline
        20 & Instruction II & \\ \hline
        20 & Introduction flashcard and -map systems to students & \\ \hline
        20 & Pre-test & Initial answers \\ \hline
        20 & Experiment & User data \\ \hline
        21 & Post-test & Final answers \\ \hline
        21 & Questionnaire & Perceived usage and ease of use \\ \hline
        21 & Finalise scoring sample items & Scored sample items \\ \hline
        22 & School test & \\ \hline
        23 & Interviews & Audio conversations \\ \hline
        23 & Determine inter-rater reliability with teacher & Cohen's kappa \\ \hline
        25 & Finalise scoring all items & Scored items \\ \hline
        27 & Finalise transcribing interviews & Transcribed interviews \\ \hline
        28 & Finalise coding interviews & Coded fragments \\ \hline
        29 & Perform t-tests & Quantitative results \\ \hline
        32 & Finalising report & Research report \\
    \end{tabular}
    \caption{The intended planning for research activities and outputs during this project \label{tab:planning}}
\end{table}

\section{Outputs}

%Include descriptions and target dates for final outputs (e.g., advice reports, delivery of products, scientific article) as well as those along the way (e.g. literature review; instruments; data collection).

Their are several products being developed and data being collected during the project. First of all, a concept map will be developed containing the main concepts on 17th century Dutch literature, which will be the basis for the flashcards and the item bank questions. Furthermore, software will be developed being capable of both providing flashcards and flashmaps. The data being collected includes initial and final student answers, questionnaire results audio conversations, and server logs from the interviews, which will then be used for conducting the t-tests and triangulation and expansion purposes. This will finally be documented in the research report.
