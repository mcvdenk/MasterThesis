\chapter{Summary}

In literature, a distinction is often made between meaningful encoding and rote learning in terms of retrieval practices, of which the former is regarded to be more beneficial than the latter. A prominent example of meaningful encoding is the concept map, where retrieval practice is often used by means of flashcards. However, recently studies have indicated the importance of both aspects. Because of this, a new tool is developed, attempting to bridge the gap between meaningless rote memorisation and meaningful learning by combining the active retrieval mechanisms of flashcards with the meaningful visualisation by concept maps. In the proposed research, the effects of flashmaps will be investigated in terms of the learning effects measured by learning gain, and the affective effects measured by a questionnaire. Furthermore, the use of the tool will be investigated by means of interviews and user logs. The research will take place within a Dutch classroom setting by first measuring prior knowledge with a pre-test, then letting the students use the tool for 7 days and finally measuring their final knowledge with a post-test. Finally, a sample of the students will be interviewed. This research hopefully leads to a better way of meaningful rote memorisation and gains more insight in how flashcard systems and concept maps can benefit from each other.
