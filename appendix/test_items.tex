\chapter{Instruments used within the experiment}
\label{app:instruments}

\section{Rubric for the test questions}
\label{sec:rubic}

\subsection{Knowledge questions}\label{knowledge-questions}

\begin{itemize}
\itemsep1pt\parskip0pt\parsep0pt
\item
  Door welke personen werd het protestantisme geleid? (13.1)

  \begin{itemize}
  \itemsep1pt\parskip0pt\parsep0pt
  \item
    Maarten Luther
  \item
    Johannes Calvijn
  \end{itemize}
\item
  Hoe werd het protestantisme verspreid? (13.1)

  \begin{itemize}
  \itemsep1pt\parskip0pt\parsep0pt
  \item
    Door de boekdrukkunst
  \end{itemize}
\item
  Welk land werd door Spanje bezet? (13.1)

  \begin{itemize}
  \itemsep1pt\parskip0pt\parsep0pt
  \item
    De Nederlanden
  \end{itemize}
\item
  Waardoor werd Spanje geregeerd? (13.1)

  \begin{itemize}
  \itemsep1pt\parskip0pt\parsep0pt
  \item
    Door koningen
  \end{itemize}
\item
  Wie waren de koningen van Spanje? (13.1)

  \begin{itemize}
  \itemsep1pt\parskip0pt\parsep0pt
  \item
    Karel V
  \item
    Fillips II
  \end{itemize}
\item
  Wie volgde Karel V op? (13.1)

  \begin{itemize}
  \itemsep1pt\parskip0pt\parsep0pt
  \item
    Fillips II
  \end{itemize}
\item
  Wat wilden de koningen van Spanje? (13.1)

  \begin{itemize}
  \itemsep1pt\parskip0pt\parsep0pt
  \item
    Centralisatie van katholicisme
  \end{itemize}
\item
  Waartoe leidde de centralisatie van het katholicisme? (13.1)

  \begin{itemize}
  \itemsep1pt\parskip0pt\parsep0pt
  \item
    Geloofsvervolging van protestantisme
  \end{itemize}
\item
  Waartoe leidde de geloofsvervolging? (13.1)

  \begin{itemize}
  \itemsep1pt\parskip0pt\parsep0pt
  \item
    Verzet van de Nederlanden
  \end{itemize}
\item
  Door wie werd het verzet geleid? (13.1)

  \begin{itemize}
  \itemsep1pt\parskip0pt\parsep0pt
  \item
    Door Willem van Oranje
  \end{itemize}
\item
  Waartoe leidde het verzet? (13.1)

  \begin{itemize}
  \itemsep1pt\parskip0pt\parsep0pt
  \item
    Tot de Tachtigjarige Oorlog
  \end{itemize}
\item
  Door wie werd het protestantisme aangehangen? (13.2)

  \begin{itemize}
  \itemsep1pt\parskip0pt\parsep0pt
  \item
    Calvinisten
  \end{itemize}
\item
  Wat was Antwerpen voor de geloofsvervolging? (13.2)

  \begin{itemize}
  \itemsep1pt\parskip0pt\parsep0pt
  \item
    Een Europees handelscentrum
  \end{itemize}
\item
  Wat voor implicaties had de geloofsvervolging voor Antwerpen? (13.2)

  \begin{itemize}
  \itemsep1pt\parskip0pt\parsep0pt
  \item
    De vlucht van kooplieden en calvinisten naar Amsterdam
  \end{itemize}
\item
  Wat werd Amsterdam na de geloofsvervolging? (13.2)

  \begin{itemize}
  \itemsep1pt\parskip0pt\parsep0pt
  \item
    Een Europees handelscentrum
  \end{itemize}
\item
  Waartoe leidde het vluchten van kooplieden? (13.2)

  \begin{itemize}
  \itemsep1pt\parskip0pt\parsep0pt
  \item
    Een explosieve groei van de handel in Amsterdam
  \end{itemize}
\item
  Door wie werden de Nederlanden bestuurd? (13.2)

  \begin{itemize}
  \itemsep1pt\parskip0pt\parsep0pt
  \item
    Door regenten
  \end{itemize}
\item
  Wat wilden de regenten in de stad? (13.2)

  \begin{itemize}
  \itemsep1pt\parskip0pt\parsep0pt
  \item
    Stedelijke cultuur
  \end{itemize}
\item
  Door wie werd de kunst en literatuur in de Nederlanden gefinancierd?
  (13.2)

  \begin{itemize}
  \itemsep1pt\parskip0pt\parsep0pt
  \item
    Door de regenten
  \end{itemize}
\item
  Waar droegen de kunst en literatuur aan bij? (13.2)

  \begin{itemize}
  \itemsep1pt\parskip0pt\parsep0pt
  \item
    Aan de stedelijke gedragscode
  \end{itemize}
\item
  Wat vormde de stedelijke gedragscode? (13.2)

  \begin{itemize}
  \itemsep1pt\parskip0pt\parsep0pt
  \item
    De stedelijke cultuur
  \end{itemize}
\item
  Wat gaf de stedelijke gedragscode aan? (13.2)

  \begin{itemize}
  \itemsep1pt\parskip0pt\parsep0pt
  \item
    Gewenst en ongewenst gedrag
  \end{itemize}
\item
  Hoe werd ongewenst gedrag ook wel beschreven? (13.2)

  \begin{itemize}
  \itemsep1pt\parskip0pt\parsep0pt
  \item
    Als boers, onbeschaafd gedrag
  \end{itemize}
\item
  Noem een voorbeeld van een literair werk die de stedelijke gedragscode
  aangeeft. (13.2)

  \begin{itemize}
  \itemsep1pt\parskip0pt\parsep0pt
  \item
    Boeren-gezelschap
  \end{itemize}
\item
  Door wie is Boeren-gezelschap geschreven? (13.2)

  \begin{itemize}
  \itemsep1pt\parskip0pt\parsep0pt
  \item
    Door G.A. Bredero
  \end{itemize}
\item
  Wat voor gedrag wordt beschreven door Boeren-gezelschap? (13.2)

  \begin{itemize}
  \itemsep1pt\parskip0pt\parsep0pt
  \item
    Boers, onbeschaafd gedrag
  \end{itemize}
\item
  Waartoe leidde boers gedrag? (13.2)

  \begin{itemize}
  \itemsep1pt\parskip0pt\parsep0pt
  \item
    Tot conflicten
  \end{itemize}
\item
  Hoe werd boers gedrag gezien? (13.2)

  \begin{itemize}
  \itemsep1pt\parskip0pt\parsep0pt
  \item
    Als onmatig gedrag
  \end{itemize}
\item
  Wat waren voorbeelden van onmatig gedrag? (13.2)

  \begin{itemize}
  \itemsep1pt\parskip0pt\parsep0pt
  \item
    Alcoholisme en ongeremde seksualiteit.
  \end{itemize}
\item
  Wat voor nieuwe bevolkingsgroep ontstond er door de explosieve groei
  van handel in Amsterdam? (13.2)

  \begin{itemize}
  \itemsep1pt\parskip0pt\parsep0pt
  \item
    Galante jongeren uit de bovenlaag
  \end{itemize}
\item
  Wat hadden calvinisten ten opzichte van voorechtelijk
  geslachtsverkeer? (13.2)

  \begin{itemize}
  \itemsep1pt\parskip0pt\parsep0pt
  \item
    Een strenge moraal
  \end{itemize}
\item
  Waar richtte zich de lyriek op? (13.2)

  \begin{itemize}
  \itemsep1pt\parskip0pt\parsep0pt
  \item
    Op de jongeren
  \end{itemize}
\item
  Waar moesten de jongeren mee leven? (13.2)

  \begin{itemize}
  \itemsep1pt\parskip0pt\parsep0pt
  \item
    Met de strenge moraal van calvinisten op voorechtelijk
    geslachtsverkeer
  \end{itemize}
\item
  Welk nieuw tijdperk luidde de explosieve groei van handel in Amsterdam
  in? (13.3)

  \begin{itemize}
  \itemsep1pt\parskip0pt\parsep0pt
  \item
    De gouden eeuw
  \end{itemize}
\item
  Welke rederijkerskamers waren er te vinden in Amsterdam? (13.3.1)

  \begin{itemize}
  \itemsep1pt\parskip0pt\parsep0pt
  \item
    De Oude Kamer (D'Eglentier)
  \item
    Wit Lavendel (Brabantse Kamer)
  \end{itemize}
\item
  Wat waren de rederijkers? (13.3.1)

  \begin{itemize}
  \itemsep1pt\parskip0pt\parsep0pt
  \item
    Dichters
  \item
    Toneelschrijvers
  \end{itemize}
\item
  Welke belangrijke dichters waren er geschoold door de Oude Kamer?
  (13.3.1)

  \begin{itemize}
  \itemsep1pt\parskip0pt\parsep0pt
  \item
    G.A. Bredero
  \item
    P.C. Hooft
  \end{itemize}
\item
  Welke belangrijke dichter was er geschoold door het Wit Lavendel?
  (13.3.1)

  \begin{itemize}
  \itemsep1pt\parskip0pt\parsep0pt
  \item
    J. van den Vondel
  \end{itemize}
\item
  Wat was de functie van de rederijkerskamers? (13.3.1)

  \begin{itemize}
  \itemsep1pt\parskip0pt\parsep0pt
  \item
    Een opiniërende rol hebben
  \end{itemize}
\item
  Wat werd er in de rederijkerskamers bediscussieerd? (13.3.1)

  \begin{itemize}
  \itemsep1pt\parskip0pt\parsep0pt
  \item
    Protestantisme
  \item
    Renaissance
  \item
    Humanisme
  \end{itemize}
\item
  Waar ontstond de renaissance? (13.3.2)

  \begin{itemize}
  \itemsep1pt\parskip0pt\parsep0pt
  \item
    In de Italiaanse republikeinse stadstaten
  \end{itemize}
\item
  Wanneer ontstond de renaissance in Nederland? (13.3.2)

  \begin{itemize}
  \itemsep1pt\parskip0pt\parsep0pt
  \item
    1500-1700
  \end{itemize}
\item
  Waar oriënteerde de renaissance zich op? (13.3.2)

  \begin{itemize}
  \itemsep1pt\parskip0pt\parsep0pt
  \item
    Op de kunst en ideeën uit de Klassieke Oudheid
  \end{itemize}
\item
  Op welke wijze werden de werken uit de klassieke oudheid opnieuw
  gebruikt tijdens de renaissance? (13.3.2)

  \begin{itemize}
  \itemsep1pt\parskip0pt\parsep0pt
  \item
    Translatio (vertaling)
  \item
    Imitatio (nabootsing)
  \item
    Aemulatio (overtreffing)
  \end{itemize}
\item
  Welke klassieke literatuurgenres werden er in de renaissanceliteratuur
  overgenomen? (13.3.2)

  \begin{itemize}
  \itemsep1pt\parskip0pt\parsep0pt
  \item
    Toneel
  \item
    Poezie
  \end{itemize}
\item
  Welke klassieke toneelgenres werden opnieuw uitgevoerd gedurende de
  renaissance? (13.3.2)

  \begin{itemize}
  \itemsep1pt\parskip0pt\parsep0pt
  \item
    Komedie (blijspel)
  \item
    Tragedie (treurspel)
  \end{itemize}
\item
  Wat was de nieuwe visie voor renaissanceliteratuur? (13.3.2)

  \begin{itemize}
  \itemsep1pt\parskip0pt\parsep0pt
  \item
    Pronken met taalschoonheden
  \item
    Pronken met mythologie
  \item
    Belerend zijn
  \item
    Diepzinnig zijn
  \end{itemize}
\item
  Welke nieuwe genres ontstonden er in de renaissanceliteratuur?
  (13.3.2)

  \begin{itemize}
  \itemsep1pt\parskip0pt\parsep0pt
  \item
    Emblematiek
  \item
    Sonnet
  \end{itemize}
\item
  Welke concepten stonden centraal in het humanisme? (13.3.3)

  \begin{itemize}
  \itemsep1pt\parskip0pt\parsep0pt
  \item
    Zelfontplooiing
  \item
    Menswaardigheid
  \end{itemize}
\item
  Waar pleitten de humanisten voor? (13.3.3)

  \begin{itemize}
  \itemsep1pt\parskip0pt\parsep0pt
  \item
    Verdraagzaamheid
  \end{itemize}
\item
  Wat waren de twee grote humanistische stromingen binnen de wetenschap?
  (13.3.3)

  \begin{itemize}
  \itemsep1pt\parskip0pt\parsep0pt
  \item
    Menswetenschappen
  \item
    Natuurwetenschappen
  \end{itemize}
\item
  Waardoor maakten de humanisten onderscheid tussen mens en dier?
  (13.3.3)

  \begin{itemize}
  \itemsep1pt\parskip0pt\parsep0pt
  \item
    Taal
  \item
    Ethiek
  \end{itemize}
\item
  Noem twee belangrijke humanisten (13.3.3)

  \begin{itemize}
  \itemsep1pt\parskip0pt\parsep0pt
  \item
    Erasmus
  \item
    D.V. Coornhert
  \end{itemize}
\item
  Wat probeerde het humanisme? (13.3.3)

  \begin{itemize}
  \itemsep1pt\parskip0pt\parsep0pt
  \item
    Het probeerde het gedachtegoed van de klassieke oudheid met het
    christendom te verzoenen
  \end{itemize}
\item
  Welke klassieke filosofische stroming werd herleefd door het
  humanisme? (13.3.3)

  \begin{itemize}
  \itemsep1pt\parskip0pt\parsep0pt
  \item
    Stoïcisme
  \end{itemize}
\item
  Waar moest men standvastig tegen zijn volgens de stoïcisten? (13.3.3)

  \begin{itemize}
  \itemsep1pt\parskip0pt\parsep0pt
  \item
    Tegen tirannie
  \end{itemize}
\item
  Waar pleitte D.V. Coornhert voor? (13.3.3)

  \begin{itemize}
  \itemsep1pt\parskip0pt\parsep0pt
  \item
    Zedelijke verbetering
  \item
    Godsdienstvrijheid
  \item
    Zelfkennis
  \end{itemize}
\item
  Wie waren tegen godsdienstvrijheid? (13.3.3)

  \begin{itemize}
  \itemsep1pt\parskip0pt\parsep0pt
  \item
    Spaanse koningen
  \item
    Calvinisten
  \end{itemize}
\item
  Waardoor werden de ideeën uit de renaissance en het humanisme
  verspreid? (13.3.3)

  \begin{itemize}
  \itemsep1pt\parskip0pt\parsep0pt
  \item
    Reizen
  \item
    Boekdrukkunst
  \end{itemize}
\item
  Van wie waren Nederlandse schrijvers afhankelijk in de middeleeuwen?
  (13.3.4)

  \begin{itemize}
  \itemsep1pt\parskip0pt\parsep0pt
  \item
    Van kerkelijke en adellijke mecenas (opdrachtgevers)
  \end{itemize}
\item
  Wie vervingen de mecenassen? (13.3.4)

  \begin{itemize}
  \itemsep1pt\parskip0pt\parsep0pt
  \item
    De regenten
  \end{itemize}
\item
  Wat waren de maatschappelijke taken van Nederlandse literatuur?
  (13.3.4)

  \begin{itemize}
  \itemsep1pt\parskip0pt\parsep0pt
  \item
    Beleren
  \item
    Moraliseren
  \item
    Kritiek leveren
  \end{itemize}
\item
  Door wie waren de maatschappelijke taken geïnspireerd? (13.3.4)

  \begin{itemize}
  \itemsep1pt\parskip0pt\parsep0pt
  \item
    Door de Romein Horatius
  \end{itemize}
\item
  Wat vond Horatius dat de dichtkunst moest bieden? (13.3.4)

  \begin{itemize}
  \itemsep1pt\parskip0pt\parsep0pt
  \item
    Utile (lering)
  \item
    Dulce (vermaak)
  \end{itemize}
\item
  Wat waren de twee groepen werken binnen de Nederlandse literatuur?
  (13.4)

  \begin{itemize}
  \itemsep1pt\parskip0pt\parsep0pt
  \item
    Strijdliteratuur
  \item
    Renaissancistisch-humanistische literatuur
  \end{itemize}
\item
  Waar verwees de strijdliteratuur naar? (13.4)

  \begin{itemize}
  \itemsep1pt\parskip0pt\parsep0pt
  \item
    Naar de actualiteit
  \end{itemize}
\item
  Waartoe stimuleerde de strijdliteratuur? (13.4)

  \begin{itemize}
  \itemsep1pt\parskip0pt\parsep0pt
  \item
    Tot actie om zich tegen Spanje te verzetten
  \end{itemize}
\item
  Welk strijdlied bevatte Een nieu Geusen Lieden Boecxken? (13.4)

  \begin{itemize}
  \itemsep1pt\parskip0pt\parsep0pt
  \item
    Het Wilhelmus
  \end{itemize}
\item
  Wat waren de onderdelen van een embleem? (13.4.1)

  \begin{itemize}
  \itemsep1pt\parskip0pt\parsep0pt
  \item
    Motto (opschrift)
  \item
    Pictura (afbeelding)
  \item
    Subscriptio (uitleg)
  \end{itemize}
\item
  Wat illustreerde een embleem? (13.4.1)

  \begin{itemize}
  \itemsep1pt\parskip0pt\parsep0pt
  \item
    Een algemene waarheid
  \end{itemize}
\item
  Waarvan maakte emblematiek gebruik? (13.4.1)

  \begin{itemize}
  \itemsep1pt\parskip0pt\parsep0pt
  \item
    Diepere betekenissen
  \item
    Analogiedenken
  \end{itemize}
\item
  Welk argument werd gemaakt voor het gebruik van analogiedenken?
  (13.4.1)

  \begin{itemize}
  \itemsep1pt\parskip0pt\parsep0pt
  \item
    Alles is Gods schepping, en daarom kunnen parallellen nooit
    toevallig zijn
  \end{itemize}
\item
  Noem een populaire auteur binnen de emblematiek (13.4.1)

  \begin{itemize}
  \itemsep1pt\parskip0pt\parsep0pt
  \item
    Jacob Cats
  \end{itemize}
\item
  Welke beroemde embleembundel publiceerde Jacob Cats? (13.4.1)

  \begin{itemize}
  \itemsep1pt\parskip0pt\parsep0pt
  \item
    Sinne- en Minnebeelden
  \end{itemize}
\item
  Wat werd centraal gesteld in de lyriek? (13.4.2)

  \begin{itemize}
  \itemsep1pt\parskip0pt\parsep0pt
  \item
    De liefde
  \end{itemize}
\item
  Welke twee vormen van liefde werden onderscheiden? (13.4.2)

  \begin{itemize}
  \itemsep1pt\parskip0pt\parsep0pt
  \item
    Platonische liefde
  \item
    Lichamelijke liefde
  \end{itemize}
\item
  Wanneer was lichamelijke liefde toegestaan? (13.4.2)

  \begin{itemize}
  \itemsep1pt\parskip0pt\parsep0pt
  \item
    Wanneer er sprake was van platonische liefde
  \end{itemize}
\item
  Op welke mode sloten de inhoud van de liederen en sonnetten aan?
  (13.4.2)

  \begin{itemize}
  \itemsep1pt\parskip0pt\parsep0pt
  \item
    Op het petrarkisme
  \end{itemize}
\item
  Naar wie is het petrarkisme genoemd? (13.4.2)

  \begin{itemize}
  \itemsep1pt\parskip0pt\parsep0pt
  \item
    De Italische renaissancedichter Petrarka
  \end{itemize}
\item
  Waar maakte het petrarkisme gebruik van? (13.4.2)

  \begin{itemize}
  \itemsep1pt\parskip0pt\parsep0pt
  \item
    Van literaire beelden en conventies
  \end{itemize}
\item
  Welke literaire conventies werden er gebruikt in het petrarkisme?
  (13.4.2)

  \begin{itemize}
  \itemsep1pt\parskip0pt\parsep0pt
  \item
    Paradoxen
  \item
    Antithesen
  \end{itemize}
\item
  Wat stond in het petrarkisme centraal? (13.4.2)

  \begin{itemize}
  \itemsep1pt\parskip0pt\parsep0pt
  \item
    De liefdesklacht
  \end{itemize}
\item
  Wat werd door het petrarkisme verheerlijkt? (13.4.2)

  \begin{itemize}
  \itemsep1pt\parskip0pt\parsep0pt
  \item
    De platonische liefde
  \end{itemize}
\item
  Waaraan was de liefdesklacht gericht? (13.4.2)

  \begin{itemize}
  \itemsep1pt\parskip0pt\parsep0pt
  \item
    Aan een onbereikbaar ideaalbeeld
  \end{itemize}
\item
  Door wie werden de literaire beelden en conventies begrepen? (13.4.2)

  \begin{itemize}
  \itemsep1pt\parskip0pt\parsep0pt
  \item
    Door de literaire elite
  \end{itemize}
\item
  Waar pleitte het petrarkisme voor? (13.4.2)

  \begin{itemize}
  \itemsep1pt\parskip0pt\parsep0pt
  \item
    Voor zelfbeheersing ten opzichte van voorechtelijk geslachtsverkeer
  \end{itemize}
\item
  Waaruit bestonden sonnetten? (13.4.3)

  \begin{itemize}
  \itemsep1pt\parskip0pt\parsep0pt
  \item
    Uit veertien versregels
  \end{itemize}
\item
  Waaruit bestonden de veertien versregels van een sonnet? (13.4.3)

  \begin{itemize}
  \itemsep1pt\parskip0pt\parsep0pt
  \item
    Octaaf
  \item
    Volta/wending
  \item
    Sextet
  \end{itemize}
\item
  Waar in het gedicht vond de volta/wending plaats? (13.4.3)

  \begin{itemize}
  \itemsep1pt\parskip0pt\parsep0pt
  \item
    Na het octaaf
  \item
    Voor het sextet
  \end{itemize}
\item
  Wie was de belangrijkste sonnetdichter in de Nederlanden? (13.4.3)

  \begin{itemize}
  \itemsep1pt\parskip0pt\parsep0pt
  \item
    P.C. Hooft
  \end{itemize}
\item
  Welk ethisch-didactisch doel diende het toneel? (13.4.4)

  \begin{itemize}
  \itemsep1pt\parskip0pt\parsep0pt
  \item
    Het uitbeelden van de stedelijke gedragscode
  \end{itemize}
\item
  Waarvoor was het toneel geopend? (13.4.4)

  \begin{itemize}
  \itemsep1pt\parskip0pt\parsep0pt
  \item
    Het publiek
  \end{itemize}
\item
  Waar vond de grootste productie van toneel plaats in de Nederlanden?
  (13.4.4)

  \begin{itemize}
  \itemsep1pt\parskip0pt\parsep0pt
  \item
    In de Nederlanden
  \end{itemize}
\item
  Waar ging de winst van het toneel naartoe? (13.4.4)

  \begin{itemize}
  \itemsep1pt\parskip0pt\parsep0pt
  \item
    Naar liefdadige instellingen
  \end{itemize}
\item
  Wie hadden er baat bij de liefdadige instellingen? (13.4.4)

  \begin{itemize}
  \itemsep1pt\parskip0pt\parsep0pt
  \item
    Regenten
  \end{itemize}
\item
  Welke theaters werden er in Amsterdam geopend? (13.4.4)

  \begin{itemize}
  \itemsep1pt\parskip0pt\parsep0pt
  \item
    Eerste Nederduytsche Academie
  \item
    Amsterdamse Schouwburg
  \end{itemize}
\item
  Wat hield het theater het publiek voor? (13.4.4)

  \begin{itemize}
  \itemsep1pt\parskip0pt\parsep0pt
  \item
    Een morele spiegel
  \end{itemize}
\item
  Wat was het onderwerp van een tragedie? (13.4.4.1)

  \begin{itemize}
  \itemsep1pt\parskip0pt\parsep0pt
  \item
    De ondergang van de hoofdpersoon
  \end{itemize}
\item
  In wat voor positie bevond de hoofdpersoon van een tragedie zich?
  (13.4.4.1)

  \begin{itemize}
  \itemsep1pt\parskip0pt\parsep0pt
  \item
    In een hooggeplaatste positie
  \end{itemize}
\item
  Waaraan verleenden tragedies grotendeels hun inhoud? (13.4.4.1)

  \begin{itemize}
  \itemsep1pt\parskip0pt\parsep0pt
  \item
    Klassieke oudheid
  \item
    Geschiedenis
  \item
    De bijbel
  \end{itemize}
\item
  Welke genres van tragedies waren er? (13.4.4.1)

  \begin{itemize}
  \itemsep1pt\parskip0pt\parsep0pt
  \item
    Retorisch-didactische tragedie
  \item
    Aristotelische tragedie
  \end{itemize}
\item
  Wat waren belangrijke voorbeelden voor de retorisch-didactische
  tragedie? (13.4.4.1)

  \begin{itemize}
  \itemsep1pt\parskip0pt\parsep0pt
  \item
    De tragedies van de Romein Seneca
  \end{itemize}
\item
  Waaruit bestonden retorisch-didactische tragedies? (13.4.4.1)

  \begin{itemize}
  \itemsep1pt\parskip0pt\parsep0pt
  \item
    Uit vijf bedrijven
  \end{itemize}
\item
  Op welke structuur was de aristotelestragedie gebaseerd? (13.4.4.1)

  \begin{itemize}
  \itemsep1pt\parskip0pt\parsep0pt
  \item
    Op de handelingsstructuur
  \end{itemize}
\item
  Wat hield de handelingsstructuur in? (13.4.4.1)

  \begin{itemize}
  \itemsep1pt\parskip0pt\parsep0pt
  \item
    Eenheid van tijd
  \item
    Eenheid van plaats
  \item
    Eenheid van handeling
  \end{itemize}
\item
  Wie heeft de handelingsstructuur bedacht? (13.4.4.1)

  \begin{itemize}
  \itemsep1pt\parskip0pt\parsep0pt
  \item
    Aristoteles
  \end{itemize}
\item
  Noem een bekend voorbeeld van een Nederlandse aristotelische tragedie
  (13.4.4.1)

  \begin{itemize}
  \itemsep1pt\parskip0pt\parsep0pt
  \item
    Gijsbreght van Aemstel
  \item
    Geschreven door J. van den Vondel
  \end{itemize}
\item
  Wat voor personages werden afgebeeld in een komedie? (13.4.4.2)

  \begin{itemize}
  \itemsep1pt\parskip0pt\parsep0pt
  \item
    Personages uit lagere klassen
  \end{itemize}
\item
  Wat voor soort taal werd in de komedie gebruikt? (13.4.4.2)

  \begin{itemize}
  \itemsep1pt\parskip0pt\parsep0pt
  \item
    Spreektaal
  \end{itemize}
\item
  Waarmee eindigde een komedie? (13.4.4.2)

  \begin{itemize}
  \itemsep1pt\parskip0pt\parsep0pt
  \item
    Met een happy end
  \end{itemize}
\item
  Noem een bekend voorbeeld van een komedie (13.4.4.2)

  \begin{itemize}
  \itemsep1pt\parskip0pt\parsep0pt
  \item
    Warenar
  \item
    Geschreven door P.C. Hooft
  \end{itemize}
\item
  Wat was een nieuw genre in het toneel? (13.4.4.2)

  \begin{itemize}
  \itemsep1pt\parskip0pt\parsep0pt
  \item
    De klucht
  \end{itemize}
\item
  Wat was het verschil tussen een klucht en een komedie? (13.4.4.2)

  \begin{itemize}
  \itemsep1pt\parskip0pt\parsep0pt
  \item
    De klucht was korter
  \end{itemize}
\item
  Wat werd er getoond in een komedie? (13.4.4.2)

  \begin{itemize}
  \itemsep1pt\parskip0pt\parsep0pt
  \item
    Grappige situaties
  \end{itemize}
\item
  Waardoor lieten personages in kluchten zich leiden? (13.4.4.2)

  \begin{itemize}
  \itemsep1pt\parskip0pt\parsep0pt
  \item
    Door primaire levensdriften
  \end{itemize}
\item
  Noem een bekend voorbeeld van een klucht (13.4.4.2)

  \begin{itemize}
  \itemsep1pt\parskip0pt\parsep0pt
  \item
    De klucht van de koe
  \item
    Geschreven door G.A. Bredero
  \end{itemize}
\end{itemize}

\subsection{Comprehension questions}\label{comprehension-questions}

\begin{itemize}
\itemsep1pt\parskip0pt\parsep0pt
\item
  Leg uit tegen welke historische gebeurtenis het verzet van de
  Nederlanden gedurende de Tachtigjarige Oorlog gericht was. (13.1)

  \begin{itemize}
  \itemsep1pt\parskip0pt\parsep0pt
  \item
    Noemt dat het verzet tegen Spanje gericht was
  \item
    Noemt Karel V
  \item
    Noemt Fillips II
  \item
    Noemt de centralisatie
  \item
    Noemt centralisatie van katholicisme
  \item
    Noemt de geloofsvervolging
  \item
    Noemt dat protestantisme of calvinisme vervolgd werd
  \end{itemize}
\item
  Leg uit waarom regenten baat hadden bij het financieren van kunst en
  literatuur in de Nederlanden. (13.2)

  \begin{itemize}
  \itemsep1pt\parskip0pt\parsep0pt
  \item
    Legt uit dat deze bijdroegen aan de stedelijke cultuur of
    gedragscode
  \item
    Legt uit dat er anders onmatig gedrag ontstond
  \item
    Noemt voorbeelden van onmatig gedrag (overmatig drankgebruik,
    ongeremde seksualiteit)
  \item
    Legt uit wie de regenten waren
  \item
    Noemt dat de winst van het toneel naar liefdadige instellingen ging
  \end{itemize}
\item
  Welk maatschappelijk probleem probeerde de lyriek op te lossen? (13.2)

  \begin{itemize}
  \itemsep1pt\parskip0pt\parsep0pt
  \item
    Noemt de opkomst van galante jongeren uit de bovenlaag
  \item
    Noemt de gouden eeuw
  \item
    Noemt dat de calvinisten een strenge moraal hadden ten opzichte van
    voorechtelijk geslachtsverkeer
  \item
    Noemt dat jongeren met deze moraal moesten leven
  \item
    Legt uit hoe de lyriek de jongeren onderwees in de verhouding tussen
    platonische en zinnelijke liefde
  \end{itemize}
\item
  Leg uit wat de Rederijkerskamers waren en wat hun functies waren.
  (13.3.1)

  \begin{itemize}
  \itemsep1pt\parskip0pt\parsep0pt
  \item
    Noemt dat de Rederijkerskamers uit vooraanstaande dichters bestond
  \item
    Legt uit dat ze een opinierende rol moesten hebben
  \item
    Noemt hierbij dat deze opinies over de renaissance gingen
  \item
    Noemt hierbij dat deze opinies over het humanisme gingen
  \item
    Noemt hierbij dat deze opinies over het protestantisme gingen
  \item
    Legt uit dat deze rol inhield dat ze moesten beleren
  \item
    Legt uit dat deze rol inhield dat ze moesten moraliseren
  \item
    Legt uit dat deze rol inhield dat ze moesten kritiek moesten leveren
  \end{itemize}
\item
  Leg uit hoe de Klassieke Oudheid opnieuw tot leven gebracht werd in de
  renaissance. (13.3.2)

  \begin{itemize}
  \itemsep1pt\parskip0pt\parsep0pt
  \item
    Noemt dat de verschillende klassieke genres opnieuw toegepast werden
  \item
    Noemt dat de klassieke ideeen opnieuw bediscussieerd werden
  \item
    Noemt dat de thema's uit de klassieke mythologie opnieuw werden
    gebruikt
  \item
    Legt uit dat de klassieke werken eerst vertaald werden (translatio)
  \item
    Legt uit dat deze werken nagebootst werden (imitatio)
  \item
    Legt uit dat deze werken overtroffen moesten worden (aemulatio)
  \end{itemize}
\item
  Hoe maakten de humanisten onderscheid tussen mens en dier? (13.3.3)

  \begin{itemize}
  \itemsep1pt\parskip0pt\parsep0pt
  \item
    Noemt dat mensen zich onderscheidden door taal te gebruiken, waarbij
    dieren dat niet deden
  \item
    Noemt hierbij grammatica
  \item
    Noemt hierbij retorica
  \item
    Noemt dat mensen zich onderscheidden door kennis te hebben over goed
    en kwaad, terwijl dieren dit niet hadden (en handelden op basis van
    instinct)
  \item
    Noemt dat deze kennis geleverd werd door poezie
  \item
    Noemt dat deze kennis geleverd werd door voorbeelden uit de
    geschiedenis
  \end{itemize}
\item
  Waarom maakte de emblematiek gebruik van analogieën? (13.4.1)

  \begin{itemize}
  \itemsep1pt\parskip0pt\parsep0pt
  \item
    Noemt dat volgens de calvinisten de wereld gods schepping is
  \item
    Noemt dat daarom parallellen nooit toevallig kunnen zijn
  \item
    Legt uit dat daarom analogieen belangrijk waren in de emblematiek
  \end{itemize}
\item
  Waarom stelde het petrarkisme de liefdesklacht centraal? Gebruik de
  literaire beelden en conventies in je antwoord. (13.4.2)

  \begin{itemize}
  \itemsep1pt\parskip0pt\parsep0pt
  \item
    Noemt de paradox
  \item
    Noemt de antithese
  \item
    Noemt dat de liefdesklacht gericht was aan een ideaalbeeld
  \item
    Noemt dat dit ideaalbeeld onbereikbaar was
  \item
    Legt uit dat de onbereikbaarheid de liefde juist beter maakt, omdat
    de platonische liefde belangrijker is dan de lichamelijke liefde
  \item
    Legt uit dat dit paradoxaal is
  \end{itemize}
\item
  Leg uit wat een volta is. (13.4.3)

  \begin{itemize}
  \itemsep1pt\parskip0pt\parsep0pt
  \item
    Noemt dat de volta zich tussen het octaaf en het sextet bevindt
  \item
    Legt uit dat de volta en dramatische wending is tussen deze twee
    delen
  \end{itemize}
\item
  Wat zijn de verschillen tussen een tragedie en een komedie? (13.4.4)

  \begin{itemize}
  \itemsep1pt\parskip0pt\parsep0pt
  \item
    Noemt dat een tragedie treurig is en een komedie vrolijk
  \item
    Noemt dat een tragedie hooggeplaatste personages bevat en een
    komedie laaggeplaatste personages
  \item
    Noemt dat een tragedie de ondergang van het hoofdpersonage toont,
    terwijl de komedie eindigt met een happy end
  \end{itemize}
\end{itemize}

\section{Translated questions from the Technology Acceptance Model}
\label{sec:tamquestions}

\subsection{Perceived usefulness
questions}\label{perceived-usefulness-questions}

\begin{itemize}
\itemsep1pt\parskip0pt\parsep0pt
\item
  Het gebruik van het flashcardsysteem maakte het leren sneller.
\item
  Het gebruik van het flashcardsysteem maakte het leren langzamer.
\item
  Het flashcardsysteem was goed voor mijn leerprestaties.
\item
  Het flashcardsysteem was slecht voor mijn leerprestaties.
\item
  Het flashcardsysteem was goed voor mijn productiviteit.
\item
  Het flashcardsysteem was slecht voor mijn productiviteit.
\item
  Door het flashcardsysteem kon ik effectiver leren.
\item
  Door het flashcardsysteem kon ik minder effectief leren.
\item
  Door het flashcardsysteem kon ik makkelijker leren.
\item
  Door het flashcardsysteem kon ik moeilijker leren.
\item
  Ik vond het flashcardsysteem nuttig.
\item
  Ik vond het flashcardsysteem niet nuttig.
\end{itemize}

\subsection{Perceived ease of use
questions}\label{perceived-ease-of-use-questions}

\begin{itemize}
\itemsep1pt\parskip0pt\parsep0pt
\item
  Het flashcardsysteem leren te gebruiken was makkelijk.
\item
  Het flashcardsysteem leren te gebruiken was moeilijk.
\item
  Ik kon het flashcardsysteem makkelijk laten doen wat ik wilde.
\item
  Ik kon het flashcardsysteem moeilijk laten doen wat ik wilde.
\item
  Het was duidelijk hoe het flashcardsysteem gebruikt moest worden.
\item
  Het was onduidelijk hoe het flashcardsysteem gebruikt moest worden.
\item
  Het flashcardsysteem was flexibel.
\item
  Het flashcardsysteem was inflexibel.
\item
  Het was makkelijk om goed te worden in het gebruik van het
  flashcardsysteem.
\item
  Het was moeilijk om goed te worden in het gebruik van het
  flashcardsysteem.
\item
  Ik vond het flashcardsysteem makkelijk om te gebruiken.
\item
  Ik vond het flashcardsysteem moeilijk om te gebruiken.
\end{itemize}
