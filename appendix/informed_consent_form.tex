\chapter{Informed consent form}
\label{app:consentform}

\section{Letter}

Betreft: onderwijskundig onderzoek \\[2ex]
Geachte ouder/verzorger, \\[2ex]
 
 
Als masterstudent Educational Science and Technology aan de Universiteit Twente voer ik vanaf 17 tot en met 25 mei een onderzoek uit binnen de VWO 4 klassen van Het Stedelijk Lyceum. Dit onderzoek loopt binnen het vak Nederlands en is volledig in samenwerking met de docenten. \\[2ex]
Aangezien het welzijn en vrijwillige deelneming van de deelnemer zeer belangijk is is het binnen de universiteit verplicht om voor een onderzoek toestemming te krijgen van de deelnemer, en als deze jonger is dan 18 jaar ook van een ouder of verzorger. Bijgaande deze brief vind u uitgebreide informatie over het onderzoek en een toestemmingsverklaring, dat door u samen met uw (pleeg-)kind ingevuld kan worden. Wanneer u of uw (pleeg-)kind geen toestemming geeft zal laatstgenoemde niet deelnemen aan het onderzoek zonder dat daar negatieve consequenties aan verbonden zijn. Bovendien is deze toestemming volledig vrijblijvend en kan de deelnemer zich op ieder moment terugtrekken. \\[2ex]
Mogen er nog vragen zijn over het onderzoek kunt u mij altijd benaderen via [email address]. Bij voorbaat dank voor uw medewerking. \\[2ex]
 
 
Met vriendelijke groeten, \\[2ex]
Micha van den Enk
 
\cleardoublepage
\cleardoublepage

\section{Informatie over het onderzoek}

\subsection{Beschrijving van het onderzoek}

Veel leerlingen ervaren moeilijkheden bij het leren van teksten voor een toets, bijvoorbeeld omdat ze niet weten hoe ze het aan moeten pakken of dat ze niet vroeg genoeg beginnen. Dit is begrijpelijk, aangezien er weinig aandacht besteed wordt op school hoe dit aangepakt kan worden. Echter, er zijn een aantal hulpmiddelen beschikbaar die hier geschikt voor zijn, waaronder het flashcard systeem. \\[2ex]
Dit systeem bestaat uit een aantal kaarten, de flashcards, waarop aan de ene kant een vraag en aan de andere kant een antwoord geformuleerd staat. Vervolgens besteedt de leerling iedere dag een bepaalde tijd aan het beantwoorden van deze vragen en dit vervolgens te controleren. Als het antwoord fout was wordt de kaart dezelfde dag nog herhaald, en als het goed was schuift de kaart door naar de volgende dag. \\[2ex]
Voor dit onderzoek is er een nieuw digitaal systeem ontwikkeld gebaseerd op het flashcard systeem, dat een aantal extra mogelijkheden introduceert waardoor het gebruik van flashcards efficiënter en betekenisvoller wordt. Het doel van het onderzoek is het evalueren van dit nieuwe systeem in de context van  het leren over 17de eeuwse Nederlandse literatuur. In het experiment worden twee verschillende versies getoetst, echter omwille het belang van het onderzoek zal hier verder niet over worden uitgewijd. \\[2ex]

\subsection{Voordelen voor de participant}

Flashcards zijn al frequent getoetst in de wetenschap en blijken veel bij te dragen aan de leereffectiviteit van leerlingen, zowel op de korte als de lange termijn. Bovendien kan een leerling op deze manier het werk verspreiden over de week en daarmee de werkdruk bij het naderen van het toetsmoment verlagen. Bovendien krijgt de leerling een waardebon van 5 euro voor Van der Poel IJs op het moment dat deze volledig aan het experiment meedoet (zie de voorwaarden in de procedure). \\[2ex]
De gegevens die verzameld worden gedurende het experiment worden volledig anoniem opgeslagen, en er wordt geen cijfer aan het experiment gebonden. \\[2ex]
De deelnemer kan er los voor kiezen om zich beschikbaar te stellen voor een interview, waarbij er gevraagd zal worden naar hoe deze het systeem gebruikt heeft. De informatie verkregen uit dit interview zal opnieuw geanonimiseerd worden en alleen via het onderzoeksrapport gecommuniceerd naar derden. Bovendien kan de leerling inzicht krijgen in zijn eigen resultaten op het moment dat deze beschikbaar zijn. \\[2ex]
 
Zie ommezijde

\clearpage

\subsection{Procedure}

\begin{enumerate}
    \item In de les van 17 mei zal er een korte introductie van het systeem plaatsvinden
    \item De leerling kan vervolgens online inloggen in het systeem
    \item Hier maakt deze een voorkennistoets
    \item Vervolgens besteedt hij 7 dagen lang iedere dag 15 minuten aan het systeem
    \item Aan het einde maakt de leerling nog een kennistoets om te kunnen bepalen wat er geleerd is in de tussenliggende periode
    \item Ook vult deze kort een enquete om zijn ervaringen met het systeem te meten
    \item Als de leerling minstens op 6 van de 7 dagen 15 minuten heeft besteed aan het systeem en de toetsen heeft ingevuld, krijgt deze de waardebon
    \item Ten slotte vindt er eventueel een interview plaats
\end{enumerate}
 
Na het einde van het onderzoek is er nog een week tijd voor de leerling om zichzelf voor te bereiden op de toets. Gedurende deze tijd kan de applicatie nog steeds gebruikt worden. \\[2ex] 
Als de student de waardebon wil ontvangen dient hij de code opgenomen in de toestemmingsverklaring in te vullen in de webapplicatie. Hierdoor kan de onderzoeker de identiteit van de deelnemer achteraf achterhalen, zonder dat identiteitsgegevens opgeslagen hoeven te worden in de gegevensbank. Als de student anoniem wenst te blijven kan hij dit veld in de webapplicatie leeglaten.
 
\cleardoublepage

\section{Toestemmingsverklaring}

Ik verklaar op een voor mij duidelijke wijze te zijn ingelicht over de aard, methode, doel en [indien aanwezig] de risico’s en belasting van het onderzoek. Ik weet dat de gegevens en resultaten van het onderzoek alleen anoniem en vertrouwelijk aan derden bekend gemaakt zullen worden. Mijn vragen zijn naar tevredenheid beantwoord. \\[2ex]
Ik stem geheel vrijwillig in met deelname aan dit onderzoek. Ik behoud me daarbij het recht voor om op elk moment zonder opgaaf van redenen mijn deelname aan dit onderzoek te beëindigen. \\[2ex]
Naam deelnemer: …………………………………………………………………………  \\[2ex]
Handtekening deelnemer: …………………………………… \\[2ex]
Naam ouder of voogd: …………………………………………………………… \\[2ex]
Handtekening ouder of voogd: ………………………………………………………… \\[2ex]
Datum: ………………… \\[2ex]
Code: XXXX 
