\chapter{Methods}
\label{ch:methods}

\section{Research design}
\label{sec:researchdesign}

Research questions~\ref{benefit}\ref{effectiveness}, \ref{efficiency}, \ref{perception}\ref{usefulness} and \ref{ease} will be investigated using intervention-based research. Because of the systems being used for self-study by the students, they can be individually assigned to a condition, and this enables the use of a true experimental design. Since this will provide the most valid and reliable results, this research design is implemented in this experiment.

Additionally, research questions~\ref{usefulness} and \ref{ease} will also be investigated using open questions on the questionnaire form and by conducting interviews with a sample of the participants.

Finally, research question~\ref{howused} will be investigating both quantitatively by logging the user behaviour during the experiment and by the open questions and interviews also used for research questions~\ref{usefulness} and \ref{ease}.

The quantitative and qualitative results will be mixed for the purposes of triangulation and expansion as described by \citeA{mixedmethods}. The interviews and logs could provide insight in the degree of which the systems were used the intended way and in why students had certain perceptions on using the systems. Both triangulation and expansion will be on a partial level of mixing, will take place concurrently, and the quantitative data will be dominant, since the qualitative data exists only to triangulate and expand the quantitative data. 

\section{Respondents}
\label{sec:respondents}

100 15 to 17 year old tenth grade Dutch high school students will be approached. They already have to prepare themselves for an exam on the same topic and thereby have incentive to learn. To increase the response rate, the students will be rewarded with a \euro{} 5 voucher for participation. The participants will be assigned to either the flashcard or the flashmap condition at random when they create a user account within the webapplication.

\section{Procedure}
\label{sec:procedure}

\paragraph{Review concept map, flashcards, and item bank} In order to verify whether the content offered within the system is in alignment with the learning goals set by the teachers, one of the teachers is asked to review the content. The feedback received from the teacher is afterwards incorporated by altering the dataset. This can be seen as the focus evaluation of the product \cite{slo}.

\paragraph{Approval ethical committee} Before the actual experiment can take place, the research setup first has to be approved by the ethical committee of the University of Twente. This takes place before the system is introduced to the students.

\paragraph{Presentation} Within the school curriculum there are two instructions planned for the topic of Dutch renaissance literature. At the end of the instruction, the researcher introduces the experiment and the system to the participants. This is meant both to attract students to participate, but also to provide a briefing next to the written briefing. Within the presentation, the benefits are stated (better preparation for the exam, preview for the exam), it is stressed that participation is voluntary and that the data will be collected anonymously, the informed consent form will be introduced, and finally the reward (icecream vourchers) will be announced, together with the conditions for receiving the reward. Information with regards to the seperate experimental conditions will be limited to the researcher explaining that there are two different versions, without elaborating what these versions entail. This is in order to prevent prejudgements from within the user, making it a double-blind experiment. It could however of course happen that the participants learn about each others version during the experiment, and unfortunately this cannot be prevented.

\paragraph{Informed consent form} The informed consent form, included within the \nameref{app:consentform} appendix on page~\pageref{app:consentform}, contains a letter, the written briefing, and a form which has to be signed by both the parent or caretaker and the participant. It also contains a code with which it can be verified that a user within the system did indeed sign this form. The briefing contains a description of the research, the advantages of participation, and the procedure of the experiment.

\paragraph{Division of respondents} Users will be assigned alternately to the control group or the experimental group. This pseudo-random assignment increases the validity of the experimental design without risking one group becoming larger than the other group. This however only holds up for the initial group, because dropout rates between the group could vary resulting in differently sized finished groups.

\paragraph{Descriptives} Before the experiment itself starts, the gender and the birthdate of the participants are prompted. This provides descriptive statistics necessary for the measure of generalisability of the results.

\paragraph{Pretest} Another form prompted towards the user before the start of the experiment is the pretest, measuring how much the user already knows and understands about the subject. This test is elaborated within the next section.

\paragraph{Experiment} The experiment itself consists of the participants having to review instances for 15 minutes over the course of 6 days. The 15 minutes are estimated to be the amount of time necessary to review one section in the instructional material, and the 6 days are chosen as a balance between having covered a large enough portion of the material to measure a significant learning gain without the participant investment being too large resulting in no student wanting to participate. The 6 days of learning ideally take place subsequently, since then students have a higher chance of retrieving repeating instances correctly. However, it is likely that participants could forget about the experiment or be too busy to invest the 15 minutes, and therefore they are allowed one non active day during the experiment. Each session consists of being presented by questions or incomplete concept maps, trying to retrieve the correct answer or missing concepts from memory, and indicating whether the correct answer or missing concept was successfully retrieved.

\paragraph{Posttest} The seventh day the participant logs in to the system he is prompted with the posttest in order to measure the level of knowledge and comprehension after the experiment. This test uses the same itembank as the pretest, and is thereby also elaborated within the next section.

\paragraph{Questionnaire} After filling in the posttest, the participant is also asked to fill in a questionnaire based on the Technology Acceptance Model, which is further elaborated within the next section. The form also contains two open questions, namely to describe what the participant thought was good about the system, and what he thought could be improved. Finally, he can fill in his email address if he is interested in being interviewed afterwards.

\paragraph{Debriefing} Finally, the participant is presented with the debriefing text from figure~\ref{fig:ui_debriefing} on page~\pageref{fig:ui_debriefing}. This states that the user will soon receive the voucher, that he is allowed to keep using the system, and that he can contact the researcher if he has questions or when he wants to see his personal data. The participant is now finished with the main experiment.

\paragraph{Scoring sample items} After all the results have been gathered, a small sample of the responses are reviewed by both one of the teachers and the researcher in order to establish an inter-rater reliability. This will take place after the school test itself, but before the teacher has scored the test administered by the school itself to remain unbiased. Furthermore, the sample will be a random anonymous selection of responses in order to minimise any halo effect. Finally, the samples are filtered on non-empty items which do also not exactly correspond to the response model, since these can be automatically scored. If the inter-rater reliability is too low, the scoring rubics will be altered in order to differentiate better among correct or incorrect responses, and the procedure is repeated. Otherwise, the rest of the responses is scored by the researcher.

\paragraph{Interview} Those who volunteered for the interview will be sent an invitation by email for an open group interview, also elaborated in the next section.

\paragraph{Icecream vouchers} The vouchers and the list containing the names of students having participated within the experiment will be handed over to the students after the scoring of the test administered within the school in order to avoid influencing the teachers during the marking of the tests.

\section{Instrumentation}
\label{sec:instrumentation}

\subsection{Test}

A test (either pre- or posttest) consists out of a knowledge section and a comprehension section, derived from Bloom's Taxonomy \cite{bloom}. The knowledge section aimed at measuring whether the rote memorisation was effective, and the comprehension section served the purpose of measuring whether the explicit relations within the flashmaps scaffolded the comprehension. Only these two levels were chosen, since higher levels are more time consuming to create and to answer by the students. Furthermore, the first step is to measure whether students can already generalise from only rehearsing questions on the knowledge level to questions on the comprehension level. The questions were randomly chosen from itembanks, where the itembank used for the knowledge questions were the flashcards themselves and the itembank for the comprehension questions a seperate set of 10 items. By randomly selecting the questions, the overall knowledge and comprehension are measured instead of specific subsets measured seperately on the pretest and posttest. This eliminates the specific item difficulty variable from the learning gain, increasing its validity. The random variable however also increases the variability of scores, decreasing the significance when comparing the conditions. Both the pretest and the posttest select 5 questions from each bank with non-overlapping items. Finally, a rubics was created in order to quantisise the openly formulated answers, consisting out of possible answer categories for each item.

\subsection{Questionnaire}

The questionnaire consists out of items based on the Technology Acceptance Model (TAM) \cite{tam}, containing items measuring perceived usefulness and perceived ease of use. \citeA{devellis} describes that the validity of a questionnaire can be increased by formulating the items mixed positively and negatively, and by repeating the items. Therefore, the items from TAM are translated to Dutch in both a positive and a negative phrasing. For each participant, 2 sets of items are created per TAM category: for each item, a phrasing is selected at random after which the set is shuffled; another set of shuffled perceived usefulness items is created with each item having the opposite phrasing of its counterpart in set 1. This results in 4 sets of items encompassing both phrasings of all items. For each item, the participant can indicate whether he completely disagrees (-2), he disagrees (-1), he neither agrees nor disagrees (0), he agrees (1), or he completely agrees (2).

\subsection{Data collection during experiment}

During the experiment itself more data is created in order to provide statistics about the usage of the system. These statistics are mainly contained within the Response objects in the database.

\paragraph{Correctness retrievals} For each response within an instance, information is stored about whether the instance was retrieved correctly by the participant. This is not only useful for the rescheduling of the instance, but also in order to gain more insights in how often a participant was able to retrieve instances correctly, and to verify whether the success rate is indeed around 90\% as stated by \citeA{microlearning}.

\paragraph{Retrieval time} Furthermore, the start time (when an instance is sent from the server to the participant) and the end time (when the response was received by the server) are stored for each response in order to measure how much time was spent on each instance, and on the system in general.

\paragraph{Active days} Finally, a list is stored for each participant containing the dates that the participant was active. An active day is here defined as a day where the participant is active for 15 minutes or where the participant reviewed all of the available instances within the read sections. This data is useful for the system to know when to prompt the posttest, but also serves as a statistic on participation rates.

\subsection{Interview}

The interview is open instead of structured or semi-structured, since the interview should provide the interviewees the opportunity to tell about their own experiences from using the system, and since the data covering specific questions is already gathered within the usage data, the questionnaire and the two open questions. The researcher will use topics based on TAM in order to verify during the interview whether the main area is covered, with an open question at the end asking for general possible improvements. Furthermore, the interview will take place as one group interview, since the school organisation only has limited availability options for providing seperate rooms fit for interviews and the group interview being the most efficient option. At the start of the interview, the interviewees will be disclosed about the aims of the research in order to establish the different versions. Results will however not be disclosed until after the experiment in order to leave the interviewees unbiased in their opinion about the perceived usefulness. Finally, with the interviewees permission, the interview is recorded for later analysis. If not, the researcher takes notes during the interview.

\section{Analysis}
\label{sec:analysis}

\paragraph{Inter-rater reliability}

\paragraph{Classical test theory}

\paragraph{Item response theory}

\paragraph{Learning gain}

%Absolute and relative

\paragraph{Instance statistics}

\paragraph{Questionnaire statistics}

\paragraph{Comparisons}

\paragraph{Hypotheses}

\paragraph{Interviews}
