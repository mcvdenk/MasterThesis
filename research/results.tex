\chapter{Results}
\label{ch:results}

\section{User participation}

The \nameref{app:dropouts} appendix on page~\pageref{app:dropouts} shows statistics on how many days the different used the system. In table~\ref{tab:dropouts_incl} it is shown that in total 63 students made an account within the system. From these students 44 used the system on at least one day for longer than 15 minutes, of which the average student participated on 4.68 days. The usage of the system is also depicted in figures~\ref{fig:dropouts_fc}, \ref{fig:dropouts_fm} and~\ref{fig:dropouts_gen}. Finally, figures~\ref{fig:activedays_fc}, \ref{fig:activedays_fm} and \ref{fig:activedays_gen} display on which days users have been actively using the system, where day 0 is the day the system was introduced within the presentation and day 21 the final day before the exam. Interesting to note here is that there are two subsequent strong increases in finished users around 6 days after the system was introduced, but that there is also a very strong increase the day before the students' exam. This is also reflected within a highly increased activity within the first week, and on the day before the exam. There is even some noticable activity after the exam took place, possibly of students already having invested some time into the system before the exam, but still finishing up in order to be rewarded with the icecream coupon.

This resulted in a total number of 25 finished users, of which 13 users within the flashcard condition and 12 users within the flashmap condition. Strange enough, both in the flashcard and the flashmap condition there is one user which did use the software for 6 days, but then did not partake in the posttest. They could be confused by the posttest (since it looks exactly the same as the pretest), and then decided to not fill it in. Finally, in figure~\ref{fig:dropouts_gen} it can be seen that there are 4 students which used the system for only 5 days. This is also strange, since that way they just missed out on the reward. A sample of 23 divided over two conditions is considerably small, and therefore any results stemming from this experiments are only indicatory and should be further investigated before making any generalisations.

Since the only students usable for the rest of the result section are those finishing the posttest, the other students will be omitted from consideration.

\section{Participant descriptives}

The participant descriptives are included in the \nameref{app:descriptives} appendix on page~\ref{app:descriptives}, containing distributions of student gender and age.

\paragraph{Gender} As can be seen in figure~\ref{tab:gender}, 15 out of the 23 total participants are male and 8 are female, where within the flashcards condition there is a 7 to 5 ratio and within the flashmap a 8 to 3 ratio. This is probably just coincidental due to the small sample size.

\paragraph{Age} All students have an age within the range of 15 to 17 with an average age of 15.75 and a modus of 16, which is to be expected from VWO4 students. There is also no considerable age difference among the conditions, indicated in table~\ref{tab:age_comp} and figure~\ref{fig:age}.

\section{Learning gain}

\section{Efficiency}

\section{Usefulness}

\subsection{Usefulness items}

\subsection{Received comments}

\section{Ease of use}

\subsection{Ease of use items}

\subsection{Received comments}

\section{Usage}

\subsection{Number of responses}

\subsection{Exponents}

\subsection{Correct retrievals}

\subsection{Interview results}
