\chapter{Results}
\label{ch:results}

\section{User participation}

The \nameref{app:dropouts} appendix on page~\pageref{app:dropouts} shows statistics on how many days the different used the system. In table~\ref{tab:dropouts_incl} it is shown that in total 63 students made an account within the system. From these students 44 used the system on at least one day for longer than 15 minutes, of which the average student participated on 4.68 days. The usage of the system is also depicted in figures~\ref{fig:dropouts_fc}, \ref{fig:dropouts_fm} and~\ref{fig:dropouts_gen}. Finally, figures~\ref{fig:activedays_fc}, \ref{fig:activedays_fm} and \ref{fig:activedays_gen} display on which days users have been actively using the system, where day 0 is the day the system was introduced within the presentation and day 21 the final day before the exam. Interesting to note here is that there are two subsequent strong increases in finished users around 6 days after the system was introduced, but that there is also a very strong increase the day before the students' exam. This is also reflected within a highly increased activity within the first week, and on the day before the exam. There is even some noticable activity after the exam took place, possibly of students already having invested some time into the system before the exam, but still finishing up in order to be rewarded with the icecream coupon.

This resulted in a total number of 25 finished users, of which 13 users within the flashcard condition and 12 users within the flashmap condition. Strange enough, both in the flashcard and the flashmap condition there is one user which did use the software for 6 days, but then did not partake in the posttest. They could be confused by the posttest (since it looks exactly the same as the pretest), and then decided to not fill it in. Finally, in figure~\ref{fig:dropouts_gen} it can be seen that there are 4 students which used the system for only 5 days. This is also strange, since that way they just missed out on the reward. A sample of 23 divided over two conditions is considerably small, and therefore any results stemming from this experiments are only indicatory and should be further investigated before making any generalisations.

Since the only students usable for the rest of the result section are those finishing the posttest, the other students will be omitted from consideration.

\section{Participant descriptives}

The participant descriptives are included in the \nameref{app:descriptives} appendix on page~\ref{app:descriptives}, containing distributions of student gender and age.

\paragraph{Gender} As can be seen in figure~\ref{tab:gender}, 15 out of the 23 total participants are male and 8 are female, where within the flashcards condition there is a 7 to 5 ratio and within the flashmap a 8 to 3 ratio. This is probably just coincidental due to the small sample size.

\paragraph{Age} All students have an age within the range of 15 to 17 with an average age of 15.75 and a modus of 16, which is to be expected from VWO4 students. There is also no considerable age difference among the conditions, indicated in table~\ref{tab:age_comp} and figure~\ref{fig:age}.

\section{Learning gain}

The pre- and posttest results are displayed in the \nameref{app:learning_gain} appendix on page~\ref{app:learning_gain}. These results are separately described for the knowledge questions and comprehension scores on the test, since they measure different variables. The different scores described and compared are the pretest scores, posttest scores, total scores, and the absolute (abs\_learn\_gain) and relative learning gains (rel\_learn\_gain). Additionally, they are reported as classical test theory scores (ctt), item response theory person abilities (irt), and person abilities from item response theory using fixed item difficulties from the combined pretest scores (fixed irt). Per category, the sample size, minimum, maximum, and mean values are displayed as descriptive values; the skew, kurtosis, and t and p values from the scipy normaltest are displayed as values for describing the distribution of the results; and $\alpha$ describes the reliability of the test (either Cronbach's alpha for the ctt results or the EAP value for the irt results). These results are described for the flashcard condition, the flashmap condition, and the combined sample of both conditions. The included graphs display histograms depicting the test matrices. Finally, the pre- and posttest scores are compared with each other by means of the non-parametric Mann-Whitney U test and the parametric Welchs' t-test in order to verify that users scored significantly higher on the posttest than on the pretest, and the learning gains between conditions are compared in order to answer research question~\ref{benefit}\ref{effectiveness}.

Remarkable are the low scores on the tests, with on average only 1.3 points on the knowledge questions (0.43 on the pretest and 2.17 on the posttest) and only 1.33 points on the comprehension questions (0.33 on the pretest and 2.33 on the posttest). This would indicate that the tests are exceptionally difficult, even though the questions are directly derived from the textbook itself and the users drilling these questions over the course of 6 days. Especially the low posttest knowledge question scores from the flashcard group are striking, since these questions were literally rehearsed during the experiment.

For both the knowledge questions as for the comprehension questions, the ctt reliability for the combined pre- and posttest score for the combined flashcard and flashmap users is around .7 --- mainly because the omission process of unreliable items ---, whereas the fixed irt reliability is around .6 (see table~\ref{tab:know_gen} and table~\ref{tab:comp_gen}). According to \citeA{devellis}, this means that the results obtained from classical testing are acceptable, whereas the results obtained from the item response theory are questionnable at best. Additionally, both the score outcomes, the figures, and the pre- and posttest comparisons in tables~\ref{tab:know_pp_fc_comp}, \ref{tab:know_pp_fm_comp}, \ref{tab:know_pp_gen_comp}, and tables~\ref{tab:comp_pp_fc_comp}, \ref{tab:comp_pp_fm_comp}, \ref{tab:comp_pp_gen_comp} indicate an average positive learning gain from the classical test theory, but a negative gain from the item response theory. Therefore, the conclusion will be based on the results from the classical test theory only.

Table~\ref{tab:learning_gain_effect} summarises the results related to learning gains. In the rows, the absolute and the relative classical test scores and the absolute item response theory person ability scores with fixed item difficulties are included for both the knowledge and comprehension questions. The item response theory results are only included for reference, and from this only the absolute learning gains are taken into consideration, since the person abilities are already estimated relative to the item difficulties. the columns include the reliability, the p-value of the normality test, the flashcard and flashmap mean score, and the p-values for in this case the Mann-Whitnney U test, since none of the results seem to be normaly distributed.

The flashmap users seem to have a higher learning gain than the flashcard users on the knowledge questions, and that looking at only the ctt results they seem to have a lower gain on the comprehension questions. None of the Mann-Whitney U test p-values for the ctt results seem to be significant however, so no conclusions can be drawn yet. This is highly likely due to the low response rate, and more significant results might be found when using a larger sample, especially since the difference in mean values are relatively high in comparison to the variance in scores.

\begin{table}
    \centering
    \begin{tabular}{lrrrrr}
        \toprule
        & $\alpha$ & $\mu_{fc}$ & $\mu_{fm}$ & $p$ \\
        \midrule
        \multicolumn{5}{l}{\emph{Knowledge questions}} \\
        \midrule
        abs-ctt & .721 & 1.25 & 2.27 & .394 \\
        rel-ctt & .721 & 0.04 & 0.05 & .464 \\
        irt & .671 & -2.67 & 3.17 & .000 \\
        \midrule
        \multicolumn{5}{l}{\emph{Comprehension questions}} \\
        \midrule
        abs-ctt & .714 & 2.00 & 0.91 & .218 \\
        rel-ctt & .714 & 0.07 & 0.04 & .245\\
        irt & .606 & -1.28 & -.97 & .688 \\
        \bottomrule
    \end{tabular}
    \caption{Compact view of the results relevant for answering research question~\protect\ref{benefit}\protect\ref{effectiveness}}
    \label{tab:learning_gain_effect}
\end{table}

\section{Efficiency}

In order to verify whether the users of the different conditions spent the same amount of effort and time on the system, the \nameref{app:instance_stats} appendix on page~\pageref{app:instance_stats} provides different statistics on the rehearsed instances. This entails the number of reviewed instances, the number of responses, the exponent from the instance.get\_exponent() function (indicating how often the instance was retrieved correctly since the last incorrect retrieval), the percentage of correct retrievals in comparison to the total amount of retrievals, and finally the amount of time the users spent on the system. For every category, the descriptives (sample, minimum, maximum, mean, and variance), distribution (skew, kurtosis and normality test t- and p-values), and cronbach's alpha are displayed, both separate for the flashcard and flashmap condition and for the combined sample of users. It should be noted that in most cases the cronbach's alpha is not a sufficient measure for determining the reliability of the test, since there is a natural decrease in most statistics over the course of the instances, since the latter instances are repeated less often than the earlier statistics. The ratio of correct retrievals might be the only exception here, however one would still expect a higher ratio of correct retrievals in earlier instances than in later instances. 

Furthermore, the absolute score, the relative score and the mean score are displayed for each condition, where the relative score is the absolute score divided by the total amount of either flashcards or edges. In the combined condition, the relative score is omitted, since the relative score is only useful for comparing the conditions. The reason why the mean score is included next to the relative score is to provide additional insight in the score per reviewed instance, whereas the other relative score is just meant for being able to compare both groups.

Finally, the flashcard and flashmap conditions are again compared using the Mann-Whitney U test and Welch's t-test. Table~\ref{tab:efficiency} displays all results in a more compact manner.

\begin{table}
    \centering
    \begin{tabular}{lrrrrr}
        \toprule
        & $\alpha$ & $\mu_{fc}$ & $\mu_{fm}$ & $p$ \\
        \midrule
        \multicolumn{5}{l}{\emph{Reviewed instances}} \\
        \midrule
        abs & 0.979 & 72.83 & 131.45 & 0.001 \\
        rel & 0.979 & 0.78 & 0.66 & 0.188 \\
        \midrule
        \multicolumn{5}{l}{\emph{Responses}} \\
        \midrule
        mean & 0.938 & 7.61 & 5.61 & 0.024 \\
        \midrule
        \multicolumn{5}{l}{\emph{Exponents}} \\
        \midrule
        mean & 0.893 & 6.67 & 6.38 & 0.625 \\
        \midrule
        \multicolumn{5}{l}{\emph{Correct retrievals}} \\
        \midrule
        mean & 0.978 & 0.86 & 0.89 & 0.000 \\
        \midrule
        \multicolumn{5}{l}{\emph{Time spent}} \\
        \midrule
        abs & 0.924 & 12374.41 & 14121.58 & 0.000 \\
        mean & 0.924 & 169.77 & 117.30 & 0.000 \\
        \bottomrule
    \end{tabular}
    \caption{Compact view of the results relevant for answering research question~\protect\ref{benefit}\protect\ref{efficiency}}
    \label{tab:efficiency}
\end{table}

\paragraph{Number of reviewed instances} Within these statistics, the mean statistics are left out, since they only make sense when reviewing them over the entire range of available instances. Furthermore, the flashcards cannot simply be compared one on one with the edges, since some of the flashcards encompass multiple edges instead of merely representing one edge. This is most notably the case when multiple sibling edges are presented as one instance to the user, which are thereby also represented by only one flashcard. Therefore, the combined conditions table only shows the relative score, since this is the only score comparable across both conditions. This is also noticable in the large difference in absolute mean values of the flashcard and flashmap condition (72.83 vs 131.45) and the low p-value on the Mann-Whitney U test (0.001). When however purely looking at the relative score, the average user reviewed 70\% of the available cards, where the flashcard users reviewed 78\% of the cards and the flashmap users 66\%. This difference is not yet significant ($p=.188$ on the Mann-Whiney U test). As can be seen in figure~\ref{fig:instance_abil} on page~\pageref{fig:instance_abil}, this difference is mainly due to both groups having reviewed about equal numbers of instances except for 2 users in the flashmap condition having reviewed less than 50\% of the edges.

\paragraph{Number of responses} These first statistics depict the total number of reviews of instances per user. On average, a user has 641.87 responses, where flashcard users have 561.33 and flashmap users have 729.73 responses. Both the Mann-Whitney U test as the Welch's t-test provide a relatively high p-value (.813 and .810) when looking at the difference in relative score, indicating no (significant) difference between the number of responses within the flashcard users and the flashmap users when adjusting for the different numbers of flashcards and edges.

\paragraph{Exponents} These statistics are useful as an indicator of learning progress of the user, since the exponents are used for rescheduling the instance where a higher exponent indicates a longer time interval until the next review. The mean exponent for a reviewed instance was 6.53, where it was 6.67 for flashcard users and 6.38 for flashmap users. This difference is not significant ($p=0.625$).

\paragraph{Correct retrievals} The ratio of correct and total retrievals is also included, since this provides more insight in the effectiveness of the scheduling algorithm and the presentation form. For each reviewed instance, the ratio of correct and total retrievals is 0.87, where 0.86 for flashcard and 0.89 for flashmap users. This results in a small although significant ($p<0.001$) difference in favour of the flashmap users. In the interviews however some students stated to not have noticed the instructions teaching how to mark retrievals to be either correct or incorrect, resulting in a 100\% correct retrieval rate within the database (see also figure~\ref{fig:score_abil} on page~\pageref{fig:score_abil}).

\paragraph{Time spent} The most efficient variable for determining the efficiency of the system is the amount of time spent by the user on the system. In total, the average user spent 13210 seconds (3 hours and 40 minutes) on the system, where the average flashcard user spent 12374 seconds (3 hours and 26 minutes) and the average flashmap user spent 14122 seconds (3 hours and 55 minutes). Per instance, this was on average 144.7 seconds, where 169.8 seconds for flashcard users and 117.3 seconds for flashmap users. Both these differences are significant ($p<0.001$).

\section{Usefulness}

\subsection{Usefulness items}

\subsection{Received comments}

\section{Ease of use}

\subsection{Ease of use items}

\subsection{Received comments}

\section{Usage}

\subsection{Interview results}
