\chapter{Results}
\label{ch:results}

\section{User participation}

The \nameref{app:dropouts} appendix on page~\pageref{app:dropouts} shows statistics on how many days the different used the system. In table~\ref{tab:dropouts_incl} it is shown that in total 63 students made an account within the system. From these students 44 used the system on at least one day for longer than 15 minutes, of which the average student participated on 4.68 days. The usage of the system is also depicted in figures~\ref{fig:dropouts_fc}, \ref{fig:dropouts_fm} and~\ref{fig:dropouts_gen}. Finally, figures~\ref{fig:activedays_fc}, \ref{fig:activedays_fm} and \ref{fig:activedays_gen} display on which days users have been actively using the system, where day 0 is the day the system was introduced within the presentation and day 21 the final day before the exam. Interesting to note here is that there are two subsequent strong increases in finished users around 6 days after the system was introduced, but that there is also a very strong increase the day before the students' exam. This is also reflected within a highly increased activity within the first week, and on the day before the exam. There is even some noticable activity after the exam took place, possibly of students already having invested some time into the system before the exam, but still finishing up in order to be rewarded with the icecream coupon.

This resulted in a total number of 25 finished users, of which 13 users within the flashcard condition and 12 users within the flashmap condition. Strange enough, both in the flashcard and the flashmap condition there is one user which did use the software for 6 days, but then did not partake in the posttest. They could be confused by the posttest (since it looks exactly the same as the pretest), and then decided to not fill it in. Finally, in figure~\ref{fig:dropouts_gen} it can be seen that there are 4 students which used the system for only 5 days. This is also strange, since that way they just missed out on the reward. A sample of 23 divided over two conditions is considerably small, and therefore any results stemming from this experiments are only indicatory and should be further investigated before making any generalisations.

Since the only students usable for the rest of the result section are those finishing the posttest, the other students will be omitted from consideration.

\section{Participant descriptives}

The participant descriptives are included in the \nameref{app:descriptives} appendix on page~\ref{app:descriptives}, containing distributions of student gender and age.

\paragraph{Gender} As can be seen in figure~\ref{tab:gender}, 15 out of the 23 total participants are male and 8 are female, where within the flashcards condition there is a 7 to 5 ratio and within the flashmap a 8 to 3 ratio. This is probably just coincidental due to the small sample size.

\paragraph{Age} All students have an age within the range of 15 to 17 with an average age of 15.75 and a modus of 16, which is to be expected from VWO4 students. There is also no considerable age difference among the conditions, indicated in table~\ref{tab:age_comp} and figure~\ref{fig:age}.

\section{Learning gain}

The pre- and posttest results are displayed in the \nameref{app:learning_gain} appendix on page~\ref{app:learning_gain}. These results are separately described for the knowledge questions and comprehension scores on the test, since they measure different variables. The different scores described and compared are the pretest scores, posttest scores, total scores, and the absolute (abs\_learn\_gain) and relative learning gains (rel\_learn\_gain). Additionally, they are reported as classical test theory scores (ctt), item response theory person abilities (irt), and person abilities from item response theory using fixed item difficulties from the combined pretest scores (fixed irt). Per category, the sample size, minimum, maximum, and mean values are displayed as descriptive values; the skew, kurtosis, and t and p values from the scipy normaltest are displayed as values for describing the distribution of the results; and $\alpha$ describes the reliability of the test (either Cronbach's alpha for the ctt results or the EAP value for the irt results). These results are described for the flashcard condition, the flashmap condition, and the combined sample of both conditions. The included graphs display histograms depicting the test matrices. Finally, the pre- and posttest scores are compared with each other by means of the non-parametric Mann-Whitney U test and the parametric Welchs' t-test in order to verify that users scored significantly higher on the posttest than on the pretest, and the learning gains between conditions are compared in order to answer research question~\ref{benefit}\ref{effectiveness}.

For both the knowledge questions as for the comprehension questions, the ctt reliability for the combined pre- and posttest score for the combined flashcard and flashmap users is around .7 --- mainly because the omission process of unreliable items ---, whereas the fixed irt reliability is around .6 (see table~\ref{tab:know_gen} and table~\ref{tab:comp_gen}). According to \citeA{devellis}, this means that the results obtained from classical testing are acceptable, whereas the results obtained from the item response theory are questionnable at best. Additionally, both the score outcomes, the figures, and the pre- and posttest comparisons in tables~\ref{tab:know_pp_fc_comp}, \ref{tab:know_pp_fm_comp}, \ref{tab:know_pp_gen_comp}, and tables~\ref{tab:comp_pp_fc_comp}, \ref{tab:comp_pp_fm_comp}, \ref{tab:comp_pp_gen_comp} indicate an average positive learning gain from the classical test theory, but a negative gain from the item response theory. Therefore, the conclusion will be based on the results from the classical test theory only.

Table~\ref{tab:learning_gain_effect} summarises the results related to learning gains. In the rows, the absolute and the relative classical test scores and the absolute item response theory person ability scores with fixed item difficulties are included for both the knowledge and comprehension questions. The item response theory results are only included for reference, and from this only the absolute learning gains are taken into consideration, since the person abilities are already estimated relative to the item difficulties. the columns include the reliability, the p-value of the normality test, the flashcard and flashmap mean score, and the p-values for in this case the Mann-Whitnney U test, since none of the results seem to be normaly distributed.

The flashmap users seem to have a higher learning gain than the flashcard users on the knowledge questions, and that looking at only the ctt results they seem to have a lower gain on the comprehension questions. None of the Mann-Whitney U test p-values for the ctt results seem to be significant however, so no conclusions can be drawn yet. This is highly likely due to the low response rate, and more significant results might be found when using a larger sample, especially since the difference in mean values are relatively high in comparison to the variance in scores.

\begin{table}
    \centering
    \begin{tabular}{lrrrrr}
        \toprule
        & $\alpha$ & norm-p & fc-$\mu$ & fm-$\mu$ & p-value \\
        \midrule
        \multicolumn{5}{l}{\emph{Knowledge questions}} \\
        \midrule
        abs-ctt & .721 & .364 & 1.25 & 2.27 & .394 \\
        rel-ctt & .721 & .372 & 0.04 & 0.05 & .464 \\
        irt & .671 & .690 & -2.67 & 3.17 & .000 \\
        \midrule
        \multicolumn{5}{l}{\emph{Comprehension questions}} \\
        \midrule
        abs-ctt & .714 & .110 & 2.00 & 0.91 & .218 \\
        rel-ctt & .714 & .118 & 0.07 & 0.04 & .245\\
        irt & .606 & .536 & -1.28 & -.97 & .688 \\
        \bottomrule
    \end{tabular}
    \caption{Compact view of the results relevant for answering research question~\protect\ref{benefit}\protect\ref{effectiveness}}
    \label{tab:learning_gain_effect}
\end{table}

\section{Efficiency}

\section{Usefulness}

\subsection{Usefulness items}

\subsection{Received comments}

\section{Ease of use}

\subsection{Ease of use items}

\subsection{Received comments}

\section{Usage}

\subsection{Number of responses}

\subsection{Exponents}

\subsection{Correct retrievals}

\subsection{Interview results}
