\chapter{Discussion}
\label{ch:discussion}

\section{Conclusions}

\paragraph{Learning gain} The only significant difference in learning gain was that the flashmap respondents scored higher on the knowledge questions than the flashcard respondents using irt analysis ($p < 0.001$). The test reliability of this analysis was relatively low ($\alpha=0.67$), however was confirmed by the non significant outcomes of the ctt analysis ($p=0.464$) with higher test reliability ($\alpha=7.21$).

\paragraph{Time spent}

\paragraph{Perceived usefulness}

\paragraph{Perceived ease of use}

\section{Limitations}



\section{Future work}

The outcomes for the comprehension questions were both insignificant and contradictory, with a higher learning gain for the flashcard condition using ctt analysis ($p=0.245$, $\alpha=0.714$) and a lower learning gain using irt analysis ($p=0.688$, $\alpha=0.606$). Since the former result has a lower p-value and higher validity, it is likely that with a larger sample size the flashcard user's learning gain could become significantly larger than that of the flashmap users. Possibly, this could be the result of the flashcard users having to actively think of the connections between the concepts themselves, as already hypothesised by one of the interviewees. The items being phrased could also have an effect. Further research using larger sample sizes could therefore be conducted.

