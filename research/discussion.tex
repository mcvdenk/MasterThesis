\chapter{Discussion}
\label{ch:discussion}

\section{Conclusions}

The research described within this thesis aimed at evaluating a new tool for learning called the flashmap system by comparing it to the already existing flashcard system. The variables used within this comparison are the learning gain over the use of the systems, the efficiency of the system, and the perceived usefulness and ease of use of the system by the users. These three variables are expounded individually within the following paragraphs.

\paragraph{Learning gain} The most important variable for solving the need for knowing and understanding countless facts expressed by \citeA{glaserfield} is the learning gain, divided into knowledge gain and comprehension gain from the taxonomy by \citeA{bloom}. The flashcard users were expected to have a higher knowledge gain, since they already practised the exact questions. The flashmap users however were expected to have a higher comprehension gain, since the flashmap system provides more meaningful relations between the concepts, instead of only providing segregated information as the flashcard system was criticised for by \citeA{hulstijn} and \citeA{mccullough}. The results however seem to indicate that the flashmap respondents scored higher on the knowledge questions than the flashcard respondents using irt analysis. The test reliability of this analysis was relatively low, however was confirmed by the non significant outcomes of the ctt analysis with higher test reliability. One explanatory hypothesis might be that there are more explicit retrieval routes in the students' memory for the learned concepts, resulting in a higher correct retrieval rate. The differences for the comprehension gain were insignificant, with a higher learning gain for the flashcard condition using ctt analysis and a lower learning gain using irt analysis. Since the former result has a higher reliability, it is likely that with a larger sample size the flashcard user's learning gain could become significantly larger than that of the flashmap users. Possibly, this could be the result of the flashcard users having to actively think of the connections between the concepts themselves, as already hypothesised by one of the interviewees. This hypothesis would confirm the statement by \citeA{canas} that 'fill-in-the-gap' uses of concept-mapping does not lead to meaningful learning. Furthermore, the items being phrased as a question within the flashcard system could also have an effect for the students' active meaningful processing, leading to a higher comprehension rate. Finally, the flashmaps presented towards the end of the 6 days were rather elaborate, perhaps leading to map-shock \cite{moore}.

\paragraph{Efficiency} Reviewing the literature provided within the introduction, no specific differences in efficiency were expected, as these results are mostly included in order to measure the fairness for the comparisons of the other variables. The efficiency of the flashcard and flashmap system are described by how much of the material was covered by the respondents, and how much time the respondents spent on the system. A non-significant lower percentage of material covered was found by the flashmap users than by the flashcard users. Looking at figure~\ref{fig:instance_abil} on page~\pageref{fig:instance_abil}, this is likely due to the 2 outliers on the lower spectrum of the flashmap users, since the other users have comparable material coverage. This could also explain the fewer responses per learned item by flashmap users than by flashcard users, although this result is significant and can thereby not be directly dismissed as random outcome. Finally, the flashmap users spent a significant larger amount of time on the system than the flashcard users. This difference could be due to the more cumbersome navigation indicated by the flashmap users. However, a significantly smaller amount of time was spent by flashmap users per item. One explanatory factor is that many flashcards asked for the retrieval of multiple concepts per items, explaining a longer retrieval time in comparison to the single concept per relation retrievals for the flashmap condition. It could also be that flashmap users had a quicker retrieval rate because of the more explicit retrieval routes within memory.

\paragraph{User perceptions} Finally, the perceived usefulness and ease of use were measured in order to provide some information on how the users generally perceived the software, indicating whether users would like to use the system or prefer using one system over the other, and providing formative feedback on how the system could be improved. The most direct variable for determining the participants' perception are the participation rates, which do not significantly differ between the experimental conditions. This could either amount to an equal perceived usefulness or equal perceived ease of user, or a combination of both factors. No significant difference has been found separately for the perceived usefulness items and the perceived ease of use items on the questionnaire between the conditions as well. Within the interviews, students commented that they liked the structure provided by the questions or flashmaps, the distribution of learning over time by the scheduling algorithm, and the repetition of questions. However, they also indicated that certain questions or flashmaps are repeated too often. They also commented that the instructions provided within the software were not always very clear and could be improved, and that the system was rigid in what they had to rehearse. Finally, the correct retrieval rate within the use of the system was around 0.87, which is indicated by \citeA{microlearning} as a desirable balance between overlearning and spacing. Herein, the flashmap users had a significant higher retrieval rate, however this can be explained by certain flashmap users not knowing how to indicate whether their retrieval was correct or incorrect, and thereby indicating all items as correct (which is the default value). 

\section{Limitations}

The largest limitation within this study is the small sample size of only 23 students divided over two groups, which is too small to come to any definite conclusions. Another limitation is that the students are a homogeneous group, studying only one subject and enrolled within the same school, making the results not fit for generalisation to any other fields before further replication studies within other groups or with different subject material are conducted.

\section{Future work}

In order to confirm the higher comprehension gain in flashcard users compared to flashmap users, replication studies using larger sample sizes should be conducted. These studies could then also be tailored towards measuring specific explanatory hypotheses. Furthermore, the research could be repeated with the participant creating their own concept maps or flashcards, providing insights in the learning gain when the system is used more generatively.

Finally, the suggestions provided within the interviews could be implemented to improve the user experience. The scheduling system could be made more adaptive, for example by including the additions suggested by \citeA{microlearning}. Furthermore, the instructions could be improved, or a tutorial could be included for learning how to use the system. Additionally, the flashmaps could show a smaller portion of the concept map, making the navigation easier (especially on smaller devices such as smartphones). On the same note, a way could be found to render the graphs hierarchically instead of cyclical, as was the original intention. Finally, options could be included for the users in order to select the sections they want to learn. The system could then be formatively evaluated, for example observing students using the system, in order to tune the system to the needs of the average user.
