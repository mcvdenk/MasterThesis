\chapter{Discussion}
\label{ch:discussion}

\section{Conclusions}

Within the following sections, answers for the research questions stated on page~\pageref{sec:intro_evaluation} are provided based on the findings stated in the \nameref{ch:results} chapter, together with explanatory hypotheses.

\paragraph{Learning gain} The only significant difference in learning gain between the flashcard and flashmap condition was that the flashmap respondents scored higher on the knowledge questions than the flashcard respondents using irt analysis ($p < 0.001$). The test reliability of this analysis was relatively low ($\alpha=0.67$), however was confirmed by the non significant outcomes of the ctt analysis ($p=0.464$) with higher test reliability ($\alpha=7.21$). One explanatory hypothesis might be that there are more explicit retrieval routes in the students' memory for the learned concepts, resulting in a higher correct retrieval rate.

\paragraph{Learning progress} The efficiency of the flashcard and flashmap system can be described by how much of the material was covered by the respondents, and how much time the respondents spent on the system. A non-significant lower percentage of material covered was found by the flashmap users than by the flashcard users ($p=0.188$). Looking at figure~\ref{fig:instance_abil} on page~\pageref{fig:instance_abil}, this is likely due to the 2 outliers on the lower spectrum of the flashmap users, since the other users have comparable material coverage. This could also explain the fewer responses per learned item by flashmap users than by flashcard users, although this result is significant ($p=0.024$) and can thereby not be directly dismissed as random outcome. Finally, the flashmap users spent a significant larger amount of time on the system than the flashcard users ($p<0.001$). This difference could be due to the more cumbersome navigation indicated by the flashmap users. However, a significantly smaller amount of time was spent by flashmap users per item ($p<0.001$). One explanatory factor is that many flashcards asked for the retrieval of multiple concepts per items, explaining a longer retrieval time in comparison to the single concept per relation retrievals for the flashmap condition. It could also be that flashmap users had a quicker retrieval rate because of the more explicit retrieval routes within memory.

\paragraph{Perceived usefulness} No significant difference has been found in perceived usefulness between flashmap and flashcard users ($p=0.245$). Within the interviews, students commented that they liked the structure provided by the questions or flashmaps, the distribution of learning over time by the scheduling algorithm, and the repetition of quesitons. However, they also indicated that certain questions or flashmaps are repeated too often. Finally, the correct retrieval rate within the use of the system was around 0.87, which is indicated by \citeA{microlearning} as a desirable balance between overlearning and spacing. Herein, the flashmap users had a significant higher retrieval rate ($p<0.001$), however this can be explained by certain flashmap users not knowing how to indicate whether their retrieval was correct or incorrect, and thereby indicating all items as correct (which is the default value).

\paragraph{Perceived ease of use} Flashmap and flashcard users did also not significantly differ in perceived ease of use ($p=0.482$). Students did comment that the instructions provided within the software could be improved, and that the system was rigid in what they had to rehearse. 

\section{Limitations}

The largest limitation within this study is the small sample size of only 23 students divided over two groups, which is too small to come to any definite conclusions. Another limitation is that the students are a homogeneous group, studying only one subject and enrolled within the same school, making the results not fit for generalisation to any other fields before further replication studies within other groups or with different subject material are conducted.

\section{Future work}

The outcomes for the comprehension questions were both insignificant and contradictory, with a higher learning gain for the flashcard condition using ctt analysis ($p=0.245$, $\alpha=0.714$) and a lower learning gain using irt analysis ($p=0.688$, $\alpha=0.606$). Since the former result has a lower p-value and higher validity, it is likely that with a larger sample size the flashcard user's learning gain could become significantly larger than that of the flashmap users. Possibly, this could be the result of the flashcard users having to actively think of the connections between the concepts themselves, as already hypothesised by one of the interviewees. The items being phrased could also have an effect. In order to confirm this hypothesis, replication studies using larger sample sizes should be conducted.

Furthermore, the research could be repeated with the participant creating their own concept maps or flashcards, providing insights in the learnign gain when the system is used more generatively.

Finally, the suggestions provided within the interviews could be implemented to improve the user experience:
%
\begin{itemize}
    \item the scheduling system could be made more adaptive, for example by including the additions suggested by \citeA{microlearning}
    \item the instructions could be improved, or a tutorial could be included for learning how to use the system
    \item the flashmaps could show a smaller portion of the concept map, making the navigation easier (especially on smaller devices such as smartphones)
    \item options could be included for the users in order to select the sections they want to learn
\end{itemize}
%
The system could then be formatively evaluated, for example observing students using the system, in order to tune the system to the needs of the average user.
