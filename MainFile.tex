% !TEX TS-program = pdflatex
% !TEX encoding = UTF-8 Unicode

\documentclass[twoside]{report} % use larger type; default would be 10pt

\usepackage[utf8]{inputenc} % set input encoding (not needed with XeLaTeX)

%%% Examples of Article customizations
% These packages are optional, depending whether you want the features they provide.
% See the LaTeX Companion or other references for full information.

%%% PAGE DIMENSIONS
\usepackage{geometry} % to change the page dimensions
\geometry{a4paper} % or letterpaper (US) or a5paper or....
% \geometry{margin=2in} % for example, change the margins to 2 inches all round
% \geometry{landscape} % set up the page for landscape
%   read geometry.pdf for detailed page layout information

\usepackage{graphicx} % support the \includegraphics command and options
\usepackage{caption}
\usepackage{subcaption}
\usepackage{longtable}
\usepackage[section]{placeins}
\makeatletter
\AtBeginDocument{%
  \expandafter\renewcommand\expandafter\subsection\expandafter{%
    \expandafter\@fb@secFB\subsection
  }%
}
\makeatother

% \usepackage[parfill]{parskip} % Activate to begin paragraphs with an empty line rather than an indent

%%% PACKAGES
\usepackage{booktabs} % for much better looking tables
\usepackage{array} % for better arrays (eg matrices) in maths
\usepackage{paralist} % very flexible & customisable lists (eg. enumerate/itemize, etc.)
\usepackage{verbatim} % adds environment for commenting out blocks of text & for better verbatim
\usepackage[font=small,labelfont=it]{caption}
\usepackage{pdfpages}
\usepackage{hyperref}
%\usepackage{bibtopic}
\usepackage{wrapfig}
\usepackage{titling}
\usepackage{michatitle}
\usepackage{gitinfo2}
\usepackage[gen]{eurosym}
\usepackage{textcomp}
\usepackage{listings}
\usepackage[toc,page]{appendix}
\usepackage{microtype}

\lstset{
    breaklines=true,
    numbers=left,
    basicstyle=\footnotesize,
    frame=single,
    showspaces=false,
    title=\lstname
}

%%% HEADERS & FOOTERS
\usepackage{fancyhdr} % This should be set AFTER setting up the page geometry
\pagestyle{fancy} % options: empty , plain , fancy
\renewcommand{\headrulewidth}{0pt} % customise the layout...
%\lhead{\gitAbbrevHash}\chead{}\rhead{Micha van den Enk {[}s1004654{]}}
%\lfoot{\today}\cfoot{}\rfoot{\thepage}

\let\Oldpart\part
\newcommand{\parttitle}{}
\renewcommand{\part}[1]{\Oldpart{#1}\renewcommand{\parttitle}{#1}}

%%%FIX PARTS!!!!!!!!!!

\fancyhead[LO,RE]{\parttitle}
\fancyhead[C]{}
\fancyhead[RO,LE]{\gitAbbrevHash}
\fancyfoot[LO,RE]{\today}
\fancyfoot[RO,LE]{\thepage}
\fancyfoot[C]{Micha van den Enk {[}s1004654{]}}

%%% SECTION TITLE APPEARANCE
\usepackage{sectsty}
\allsectionsfont{\sffamily\mdseries\upshape} % (See the fntguide.pdf for font help)
\setcounter{secnumdepth}{-1} 
% (This matches ConTeXt defaults)

%%% RULE

\newcommand{\HRule}{\rule{\linewidth}{0.5mm}}

%%% BIBLIOGRAPHY

\usepackage{apacite}                           %bibliography in apa-style

%%% ToC (table of contents) APPEARANCE
\usepackage[titles,subfigure]{tocloft} % Alter the style of the Table of Contents
\renewcommand{\cftsecfont}{\rmfamily\mdseries\upshape}
\renewcommand{\cftsecpagefont}{\rmfamily\mdseries\upshape} % No bold!
\setcounter{tocdepth}{1}

\makeatletter
\renewcommand\chapter{\if@openright\cleardoublepage\else\clearpage\fi
                    \thispagestyle{fancy}%
                    \global\@topnum\z@
                    \@afterindentfalse
                    \secdef\@chapter\@schapter}
\makeatother

%%% CUSTOM DEFINITION STYLING

\newenvironment{definition} {
    \vspace{2ex}
    \begin{tabular}{p{0.8\textwidth}}
        \centering
        \large
        \em
}
{
    \end{tabular}
    \vspace{2ex}
}

%%% END Article customizations

%%% The "real" document content comes below...

\begin{document}

\hyphenation{re-con-struc-ti-vism sta-ten-bij-bel}

\supervisora{dr. A.H. Gijlers}
\supervisoremaila{a.h.gijlers@utwente.nl}
\supervisorb{dr. L. Bollen}
\supervisoremailb{l.bollen@utwente.nl}
\title{Developing a Tool for Learning Concept Maps}
\coursename{Final Project Thesis}

\maketitle
\tableofcontents
\thispagestyle{fancy}
\bibliographystyle{apacite}

%\part{Research Proposal}
%\chapter{Summary}

Here follows a summary of maximum 250 words.


\chapter{Project Description}

\section{Problem Statement}

%Describe a rationale for the focus and aim of this study; why is the topic of your research important? This can be done from a theoretical, as well as from a practical point of view (or both). For example, what gaps exist in current literature, what problems need to be solved, What tool/intervention do we need, what societal changes have led to a need for this research?

%IDEAL AFFAIRS

Within the history of educational psychology, three major distinct perspectives of the learning process have been proposed, namely the behavioural, the cognitive and the constructivist perspective \cite{ertmer}. Learning theory first shifted from behaviourist models to cognitivist models, resulting in a change in focus on observable performance to what is happening within the learner itself. However, both these theories still assume a primarily objectivistic world view, whereas the final perspective of constructivism offer an alternative, relativistic world view. Within this model, the student is not supplied with information, but instead has to construct his own model of reality. \citeA{glaserfield} however criticised constructivism for its lack of rote memorisation, and argues for a need to train students so that they permanently possess facts and are able to repeat them flawlessly whenever they are needed, while also understanding what is placed into their memory.

%introduction flashcards

%n1.1.7, n1.1.3
A currently existing method to efficiently rote memorise information is the flashcard system, where declarative knowledge is studied in a paired-associate format. Within this format, learners are asked to associate terms with other terms outside meaning-focused tasks, for example by associating a definition with a presented concept \cite{nakata}. With flashcards, large numbers of words can be memorised in a very short time, and are more resistant to decay \cite{nakata, joseph}. \citeA{macquarrie} adds to this by stating that increasing the amount of drill or pracice is the most effective device that can be applied to learning. Finally, when evaluating flashcards in a psychology setting, it was found that students who use flashcards have a significantly higher final average than those who do not \cite{burgess, golding}.

%EXPLAIN THE PROBLEM

%n1.1.2, n1.1.3.2, n1.1.3.8, n1.5.9, n1.2.1.2
Per contra, not all research favours using flashcards for textual comprehension. \citeA{zirkle} states that flashcards are especially useful for learning declarative knowledge, while learning from a textbook is a form of learning for intellectual skills \cite{instructionaldesign}. This problem is also emphasised by \citeA{mccullough}, who states that the use of flashcards is helpful for language learning but the main emphasis of flashcards is memorisation, not comprehension. \citeA{zirkle} points out the overemphasis placed upon the rote memorisation of disconnected facts, whereas whatever it is that students are to place into memory they should, more importantly, understand. Furthermore, \citeA{hulstijn} describes flashcards as a relic of the old-fashioned behaviourist learning model, and states that we have to look for more modern constructivist models.

%EXPLAIN WHY THE PROBLEM IS IMPORTANT

%n1.1.1.7
Solving the aforementioned problem could lead to better understanding of memory, and could lead to better utilisation by teachers and students with the intent to produce a store of knowledge that remains flexibly retrievable in a variety of contexts over a period of time, in contrast to only segregated paired associations which depend on specific cues in order to be retrieved. Furthermore, it could pave the way for the design of new educational activities based on consideration of retrieval processes. Furthermore, using computer-based flashcards have been used very widely \cite{nakata}, and more recently textbooks have started making flashcards available on their websites \cite{burgess, golding}. \citeA{kornell} stated that "Perhaps no memorisation technique is more widely used than flashcards" (p. 125). Improving currently existing flashcards therefore has the potential of reaching a wide audience of future users of flashcard systems. Finally, it might be a solution to the need expressed by \citeA{glaserfield} for more meaningful rote memorisation.

%PROPOSE A SOLUTION/IDEA AND ITS BENEFITS

%introduction concept maps

%n1.2.1, n1.(2.1),(1.3).1
An instructional tool more in line with constructivistic approaches is the concept map, which is defined by \citeA{eppler} as a hierarchical diagram showing the relationship between concepts, including cross connections among concepts, and their manifestations (see figure~\ref{fig:conceptmap}. Multiple researchers have found by means of both qualitative and quantitative studies that concept maps can promote meaningful learning leading to positive effects on students \cite{hwang2, subramaniam, canas}. This has been demonstrated in comparison to activities such as reading text passages, attending lectures, and participating in class discussions \cite{singh, nesbit2}. \citeA{canas} describes the process of concept mapping as the only effective way of using the concept map, which refers to students constructing their own concept maps. This is why the concept map is generally viewed as a tool in alignment with the constructivist perspective. Because of this, the concept map might seem as a solution to the need asked by \citeA{glaserfield} and his peers. However, a recent article by \citeA{karpicke2} reveals that paired associate learning produced better performance than elaborative concept mapping for meaningful learning, even on the short-term.

%introduction flashmaps

%n1.2.6.8 and n1.2.6.9 (Counterarguments), n1.2.5, n1.2.10
Therefore, another solution might be the development of a new tool, namely the flashmap. The intention behind the flashmap is to combine the paired associate mechanism of the flashcard system with the visual represenation of the concept map, and is a new tool designed and developed for this research project. This tool might have the potential to bridge the gap between the two systems and therefore make meaningful and effective rote memorisation possible, for it makes the relations between the concepts explicit to the student.

%SUMMARISE, END WITH RESEARCH GOALS

In conlusion, flashcards systems are an effective tool for meaningful learning, but could be enhanced by visualising it with concept maps. The objective of this research is therefore to evaluate whether learning with a flashmap is a more effective or efficient for meaningful learning than flashcards, and whether it might be more affective.

\section{Theoretical Conceptual Framework}

%Describe theoretical framework where you introduce the most important concepts in your study and their relation

%FLASHCARDS

%n1.1.6

%CONCEPT MAPS

%n1.2.4
%n1.2.8
%n1.4.1.4

%FLASHMAPS

\citeA{canas} describes that fill-in-the-cmap or memorise the concept map conditions are not recommended, for meaningful learning does not work this way. However, they do not provide statistics or literature in order to support this claim, and furthermore the findings from \citeA{karpicke2} about paired associate learning being more effective for meaningful learning than concept mapping also puts this claim into doubt. Finally, the flashmap creates the opportunity for a more interactive concept map that starts with a parsimonious and theme-oriented structure which gradually expand the details along with the instruction, advised by \citeA{tzeng} to mitigate map schock. This phenomenon occurs when students view the kind of larger concept maps that might more fully capture textbook knowledge structures, but is a type of cognitive overload that prevents students from effectively processing the concept map and thereby inhibiting their ability to learn from it \cite{moore}.

\section{Research Question and Model}

%The research questions (hypotheses if applicable) and model are described here. A figure on your research model is optional.

\section{Scientific and Practical Relevance}

%Describe the expected contribution of your study (how can scientists, practitioners benefit)

%n1.1.3.1
%n2.1


\chapter{Research Design and Methods}

\section{Research design}

%What kind of research is this? (e.g. exploratory, confirmatory, intervention-based, evaluationbased, design-based, etc.). Justify the type of research design(s) you intend to use :descriptive (e.g., case study) ;correlational (longitudinal) ; (quasi-)experimental; review etc.

\section{Respondents}

%This section is where you describe the who will be approached to participate in your study and how many. Explain how they will be selected (sampling method). Make sure this justification fits your research design and questions.

\section{Instrumentation}

%Describe the instruments you will use; make a link between the research variables and the instruments explicit (operationalization process) and describe their measurement level if applicable. If you are using existing instruments, provide references.

\section{Procedure}

%Describe the procedure for your data collection; what will respondents in your study do? Here you can also address any constraints you may need to cope with in the research and the actions to guard its quality and validity. Potential ethical concerns can be addressed here too.

\section{Data Analysis}

%Describe the type of data you will generate (qualitative, quantitative, mixed-method) and the methods you intend to use to analyze your data. Make sure this fits with your research design and questions. Here you may also address reliability checks for your instrumentation.


\chapter{Planning}

\section{Timeline}

%Include an overview for whole project

\section{Outputs}

%Include descriptions and target dates for final outputs (e.g., advice reports, delivery of products, scientific article) as well as those along the way (e.g. literature review; instruments; data collection).



%\bibliography{references}

\input{./preface/acknowledgements.tex} %Not started
\input{./preface/preamble.tex} %Not started
\chapter{Abstract}

Modern day society requires students to memorise and understand a large number of facts. Currently, a powerful learning tool for comprehension is concept mapping, which entails drawing meaningful relations between concepts in an associative network. For rote memorisation, flashcard systems are widely employed, which entails students going repeatedly through a set of questions and recalling their answers from memory. Critics have found concept mapping not entailing any method for retaining facts in memory, where others found flashcard learning extracting all meaning and context from the learning process. Therefore, a new learning tool is developed within this study, aiming to bridge the gap between aforementioned learning tools by integrating the visualisation of concept maps within the retrieval mechanism of flashcard learning. The new tool is an augmentation on the flashcard system, which asks the students to fill in empty concepts within given concept maps instead of to answer provided questions. Since retrieval practices --- such as flashcard learning --- have already been found to provide more meaninful learning than concept mapping in a recent study, the new tool is compared with a generic flashcard system for knowledge retention and comprehension. Furthermore, the usefulness and ease of use perceived by the participants are compared. Because of a low response rate, the results from the comparisons are only indicatory for further research.
 %Not started

\part{Introduction}
    \chapter{Project Description}

\label{ch:problem}

Over the centuries, knowledge has been fundamental to any learning process. Socrates already stated that knowledge is the only true virtue, and the tragedian Aeschylus regarded memory as the mother of all knowledge. Moreover, it was not only regarded as important by ancient thinkers, but is still regarded as such by modern scholars on education. Both the taxonomy of learning by \citeA{bloom} as a revision of this taxonomy by \citeA{krathwohl}, as well as the three stages of skill acquisition by \citeA{skillacquisition}, propose that all learning should start with memorising factual knowledge. Furthermore, \citeA{glaserfield}, one of the main founders for critical constructivism, expresses a need for training students so that they permanently possess facts and are able to repeat them flawlessly whenever they are needed, while also understanding what is placed into their memory. \citeA{ltwm} adds to this by stating that in order to perform complex tasks, people must maintain access to large amounts of information, and that solely encoding knowledge is not sufficient. Despite all of this, \citeA{karpicke4} argues that "[r]etrieval processes, the processes involved in using available cues to actively reconstruct knowledge, have received less attention" (p. 158), whereas basic research on learning and memory has emphasised that retrieval must be considered in any analysis of learning.

Traditionally, when students have to gain complex and meaningful knowledge -- for example knowledge about a historical event or a chapter in a psychology textbook, they are asked to read the relevant chapter from a provided textbook. However, \citeA{learninginstruction} states that many students have difficulty gaining knowledge in this manner. He breaks reading for comprehension down into four separate skills, which are integrating, organising, elaborating, and monitoring. Integrating refers to relating a text to one's prior knowledge, for which exists evidence that rich background knowledge leads to better inferences about the text, and thereby better comprehension. After integration, the reader has to organise the text, so that the important ideas and the relatinoships among them are identified. This is mainly a problem for less experienced readers, which will spend too much time on reading the unimportant information. At the same time, the student has to make necessary inferences while reading, or has to elaborate, which is quite difficult for readers when not prompted. Finally, students have to monitor their comprehension, which refers to evaluating comprehension of the text and if necessary adjusting the reading strategy. This is again quite difficult for the average reader, however can be trained.

While intergrating is something more dependent on the curriculum design, organising and elaborating can be facilitated by concept mapping, and monitoring by flashcard systems. Furthermore, this research aims to develop a tool which combines these tools, called the flashmap. In this chapter, concept mapping, flashcard systems, and the flashmap will be explored on a practical level in order to establish their definitions together with a summary of arguments in favour or opposition of using them as tools for studying textual material.

\section{Concept mapping}

%Definition

%Summary of arguments in favour
    %Reference to context chapter
    %Reference to theoretical framework
    %Practical studies

\section{Flashcard system}

\subsection{Definition}

In the context of language learning, \citeA{nakata} define flashcard systems where ``target items are presented outside meaning-focused tasks, and learners are asked to associate the L2 [foreign language] word form with its meaning, usually in the form of a first language translation, L2 synonym, or L2 definition'' (p. 17). This form of learning is also refered to as a \emph{paired-associate format}, which refers to learning by being presented by cues and the learner having to recall an associated counterpart. Therefore, the following general definition is proposed:

\begin{definition}
A flashcard system refers to any system in which a learner is presented by cues and has to recall answers from an paired-associate format.
\end{definition}

The most simple form of a flashcard system is a system where the learner has a stack of cards, with each containing a retrieval cue on one side and the correct associated response on the other side. He can then study by going through the whole stack each day, trying to come up with the correct answer. He can then increase its efficiency by repeating difficult cards more often, or skipping reviewing certain easy cards for multiple days. This way he focuses only on the pairs which are more needy of retrieval. Finally, he can choose to not start with the whole stack on the first day, but increase the size of the stack over multiple days in order to improve the spreading of cognitive load. Next to these paper flashcards there is also a multitude of digital flashcard systems available \cite{hwang2, nakata, microlearning}, which allows for automating the rescheduling of flashcards, making it possible to create more advanced mechanisms.

\subsection{Effectiveness}

Flashcard systems have not been completely free from criticism by other researchers. \citeA{hulstijn} for example describes flashcards as a relic of the old-fashioned behaviourist learning model, and \citeA{mccullough} states that the main emphasis of flashcards is memorisation, not comprehension. However, \citeA{zirkle} states that it is still important for teachers and students to understand and utilise memory in such a way that a store of knowledge is produced that remains flexibly retrievable in a variety of contexts over a period of time, even more so because even though it is deemed useless to learn without comprehension, students still should learn by heart many conventional facts \cite{glaserfield}. Flashcards have found to be both a time efficient tool for learning large numbers of facts and an effective tool for these facts to be more resistant to decay in comparison to traditional teaching methods \cite{nakata}. They have also been found to be effective accross studies in different contexts, for example that of language learning \cite{chien, nakata, macquarrie, mccullough}, word recognition \cite{joseph}, psychology courses \cite{burgess, golding}, and geography \cite{zirkle}. Therefore, many authors support pursuing research into flashcards and its effective application into classrooms.

\subsection{Design features}

\citeA{nakata} also describe general design features of flashcard software, which are seperated in terms of creation and editing of flashcards, and learning of flashcards. These entail features whether learners are able to create their own flashcards or flashcard sets, whether learners merely have to recall an answer or have to produce an answer, how big a learning session is and how repetitions are scheduled. Partly, these features are also applicable on paper flashcards. The features will be further elaborated later on page INSERT REFERENCE TO DESIGN CHAPTER, but for now it is sufficient to state that there do not exist commonly accepted guidelines for how flaschard software should be designed. This mainly is due to the fact that not a lot of research is conducted on specific design-features, research reviewing mostly the same program, and there being discrepancies in the way they are designed, and further research is needed in order to establish these guidelines.

%TODO: add reference to design chapter

\subsection{Application of flashcards}

Multiple sources describe an increase in the use of flashcards in education: \citeA{kornell} states that ``perhaps no memorisation technique is more widely used than flashcards'' (p. 125), and more recently textbooks have also started making them available \cite{burgess, golding}. \citeA{golding} provides two reasons for the popularity of flashcards: students can generate flashcards for themselves, they feel taht they are `doing' something when they study. Most of the studies found is based around flashcard usage in language courses \cite{nakata, joseph, chien}, but there also exists a study by \citeA{golding} describing that 70\% of general psychology students used flashcards for at least one exam.

\citeA{chien} and \citeA{nakata} describe that multimedia and digital flashcards are used very widely, because they can be easily programmed to keep track of performance and better control the sequency, which is cumbersome if done manually, and that students might be more motivated using digital flashcards because of the enhanced presentation of materials due to their multimedia capabilities. However, \citeA{golding} still found the majority of students using written flashcards. These findings surprised \citeA{burgess}, since many students have their smart phones with them most of the time -- 75\% of students report using smartphones during breaks, meetings etc, 55\% while waiting, and 45\% for school related uses -- and phones are more portable than large stacks of traditional flashcards. However, when he pursued the study by providing students with either written or digital flashcards, students used the digital flashcards less frequently than the traditional flashcards, even when the students had to make their own flashcards. Reasons students provided were technical issues such as battery consumption, simply forgetting about it, using entertainment apps instead of studying, and preference for traditional flashcards.


\section{Comparison of the two tools}

%Criticism flashcard systems
%Criticism concept mapping

%Study karpicke4

%Critical constructivism

\section{Flashmap}

%%%%%%%%%%%%%%%%%%%%%%%%%%%% FROM THE RESEARCH PROPOSAL (TODO: DELETE LATER) %%%%%%%%%%%%%%%%%%%%%%%%%%%%%%%%%%%%%%%%%%%%%%%%%%%%%

%In the context of language learning, \citeA{nakata} defines a flashcard system as ``Target items [...] presented outside meaning-focused tasks, [where] learners are asked to associate the L2 [foreign] word form with its meaning, usually in the form of a first language translation, L2 synonym, or L2 definition'' (p. 17). In a generalised form, a flashcard system contains pairs by association, where the student is presented by one member of the pair and has to recall the other. This pair can indeed consist out of words in different languages, but can also consist out of a picture and a word for learning spelling, or a question with an answer for learning history. Traditionally, these pairs were presented as small cards (hence the name), and students would spend each day going through the stack of cards, preparing themselves for their exams.

%\section{Problem statement}

%n1.1.7, n1.1.3

%Both the taxonomy of learning by \citeA{bloom} as a revision of this taxonomy by \citeA{krathwohl}, as well as the three stages of skill acquisition by \citeA{skillacquisition}, propose that all learning should start with memorising factual knowledge. Furthermore, \citeA{glaserfield}, one of the main founders for critical constructivism, expresses a need for training students so that they permanently possess facts and are able to repeat them flawlessly whenever they are needed, while also understanding what is placed into their memory. \citeA{ltwm} adds to this by stating that in order to perform complex tasks, people must maintain access to large amounts of information, and that solely encoding knowledge is not sufficient. Despite this, \citeA{karpicke4} argues that "[r]etrieval processes, the processes involved in using available cues to actively reconstruct knowledge, have received less attention" (p. 158), whereas basic research on learning and memory has emphasised that retrieval must be considered in any analysis of learning. Therefore, this project aims to research a tool for meaningfully enhancing the retrieval process. 

%\citeA{karpicke4} also states that meaningful learning often is defined in contrast to rote learning, and that active retrieval is thought of as an example of the latter leading to poorly organised knowledge that lacks coherence and integration. However, in another study they found active retrieval to enhance learning of meaningful educational materials and that these effects are long-lasting, not short-lived \cite{karpicke2}. In this study, he compared the effects of active retrieval using measures of meaningful learning contrasting to a popular learning strategy known as concept mapping. The latter involves a graph consisting of nodes representing concepts and labeled lines denoting the relation between a pair of nodes \cite{ruiz1} (see figure~\ref{fig:conceptmap}). Multiple researchers have found by means of both qualitative and quantitative studies that concept maps can promote meaningful learning leading to positive effects on students \cite{hwang2, subramaniam, canas}. This has been demonstrated in comparison to activities such as reading text passages, attending lectures, and participating in class discussions \cite{singh, nesbit2}. \citeA{canas} describes the process of concept mapping as the only effective way of using the concept map, which refers to students constructing their own concept maps. This is why the concept map is generally viewed as a tool in alignment with the constructivist perspective. Because of this, the concept map might seem as a solution to the need asked by \citeA{glaserfield} and his peers. However, the aforementioned article by \citeA{karpicke2} reveals that retrieval practices produced better performance than elaborative concept mapping for meaningful learning.

%\begin{figure}
%    \centering
%    \includegraphics[width=\textwidth]{img/conceptmap}
%    \caption{An example of a concept map}
%    \label{fig:conceptmap}
%\end{figure}

%One of the currently existing methods for efficiently rote memorising information is the flashcard system, which entails studying declarative knowledge using active retrieval in a so-called paired-associate format. Within this format, learners are asked to associate terms with other terms outside meaning-focused tasks \cite{nakata}, for example by associating a definition with a presented concept. With flashcards, large numbers of words can be memorised in a very short time, and are more resistant to decay \cite{nakata, joseph}. Furthermore, when evaluating flashcards in a psychology setting, it was found that students who use flashcards have a significantly higher final average than those who do not \cite{burgess, golding}.

%EXPLAIN THE PROBLEM

%n1.1.2, n1.1.3.2, n1.1.3.8, n1.5.9, n1.2.1.2
%Per contra, not all research favours using flashcards for textual comprehension. \citeA{zirkle} and \citeA{mccullough} state that flashcards are especially useful for learning declarative knowledge but not for textual comprehension. \citeA{zirkle} points out the overemphasis placed upon the rote memorisation of disconnected facts, whereas whatever it is that students are to place into memory they should, more importantly, understand. Furthermore, \citeA{hulstijn} describes flashcards as a relic of the old-fashioned behaviourist learning model, and states that we have to look for more modern constructivist models.

%EXPLAIN WHY THE PROBLEM IS IMPORTANT

%n1.1.1.7
%Solving these problems could lead to better utilisation by teachers and students of producing a store of knowledge that remains flexibly retrievable, in contrast to only segregated paired associations which depend on specific cues in order to be retrieved. Furthermore, using computer-based flashcards have been used very widely \cite{nakata,burgess, golding,kornell}, and improving currently existing flashcards could reach a wide audience of future users of flashcard systems.

%PROPOSE A SOLUTION/IDEA AND ITS BENEFITS

%introduction flashmaps

%n1.2.6.8 and n1.2.6.9 (Counterarguments), n1.2.5, n1.2.10
%Therefore, another solution might be the development of a new tool, which will from henceforth be referred to as the flashmap system. The intention behind the flashmap system is to combine the paired associate mechanism of the flashcard system with the visual representation of the concept map, and is a new tool designed and developed for this research project. This tool might have the potential to bridge the gap between the two systems and therefore make meaningful and effective rote memorisation possible, for it makes the relations between the concepts explicit to the student and thereby increasing the organisation of the knowledge and reducing the segregation of facts. Thereby, it might provide a solution for the problems by \citeA{zirkle} described before.
 %Reviewed by Tobias TODO: missing references to labels
    \chapter{Context}

As can be read in the previous chapter, the aim of this study is to develop and evaluate a tool designed for the purpose of meaningful memorisation. However, why is it actually important to memorise? This question has historically been debated since the days of the early Greek philosophers, and still remains relevant today. Therefore, before delving into the effectivity and specifications of the tool itself, it seems important to briefly reflect on this question first. This chapter does not aim to answer this age-old question, but rather tries to provide both some philosophical and historical context, for better understanding of the relevance of a better memorisation tool, and what `better' generally entails. Furthermore, it more specifically will relate these questions specifically to the tools investigated within this study.

\section{Five educational philosophies}

Curriculum theorisers have proposed many different systems of categories \cite{curriculumtheory}, of which the aim is to investigate which goals people involved with education have, and which aspects they therefore regard as being important. \citeA{educationalphilosophy5} differentiates between the five philosophies of education \emph{Perennialism}, \emph{Essentialism}, \emph{Progressivism}, \emph{Reconstructionalism}, and \emph{Existentionalist education}, which have also (at least partly) been acknowledged by other authors \cite{educationalphilosophy, educationalphilosophy2, educationalphilosophy3, educationalphilosophy4}. Furthermore, \citeA{educationalphilosophy3} have found these categories to be sufficiently valid and reliable upon measuring their prevalence among teachers. Therefore, these categories will be discussed further individually in order to provide philosophical context towards the function of knowledge.

\section{Perennialism}

According to perennialism, there is no alteriar motive for attaining knowledge, but rather that attaining knowledge is a purpose on itself. This is along the words of Socrates, who concluded that knowledge is the only virtue. This he concluded based on that wisdom is the same as knowledge \cite{wisdomknowledge}, that wisdom is one of the five cardinal virtues, and that all other virtues (e.g. justice) are merely derived from the virtue of wisdom.

The perennialists are mainly based on either the general philosophy of idealism or of realism. The most notable idealist perennialist are the scholastics, who focused on teaching the great classical and religious works in order to better understand their supreme being. Realist perennialists believe the classic works still have much implications today, and therefore should be taught to the next generation.

Methods generally practiced by are considered to be rather traditional, example of these are memorisation, reading, writing, drill, and recitation. It is also the only philosophy which has many of its followers believing that education should be directed towards the intellectually gifted, and that other students should only receive vocational education.

Perennialism has been the leading philosophy in academics before the enlightenment. In the classical era, Greek students had to memorise and recite famous poetry, such as the Iliad and Odyssey by Homer, because these were believed to ``provide great moral lessons and taught them what it meant to be a Greek'' \cite[p.139]{searchgreeks}. This academic tradition was then perpetuated throughout the middle ages by the scholastics, who used the rationalism of the Greek philosophers to defend christian doctrine -- most notably in the \emph{Summa Theologica} by Thomas Aquinas. Scholastic instruction consisted of four elements: \emph{lectio}, the reading of an authoritative text; \emph{mediatio}, a reflection on the text; \emph{quaestio}, questions from students about the text; and \emph{disputationes}, a discussion about controversial \emph{quaestiones}. With the coming of the enlightenment, academics transisted from using classical idealism as a source of truth and instead used experimentalism as a source of learning about the material world and verifying truth claims, and humanism as a means to a better understanding of the human endevour. Nonetheless, perennialism remained a prominent philosophy in education until the industrial revolution in the 19th century, and still has a place in modern society in the form of for example the Great Book program proposed by Hutchins, albeit in a far lesser degree than before the enlightenment.

\section{Essentialism}

Essentialism is generally seen as a child philosophy of perennialism, and is more goal oriented than its parent. Its purpose is to pass on knowledge to new generations in order for them to be able to function in society, and focuses on subject matter. It is also a very teacher oriented approach to education.

This philosophy also is based on both idealism and realism, whereas the idealists think the content comes from history, language and the classics, and the realists think it comes from the physical world, including mathematics and the natural sciences.

Just like perennialism, essentialist teaching methods are rather traditional, and include returning to the three R's, reading, lectures, memorisation, repetition, audio-visual materials, and examinations.

The earliest form recognisable as essentialist is the factory model of education \cite{honours}, which was a means to deliver education to the general public for the benefit of the whole society. This model was improved upon by introducing aspects of behaviourism with the introduction of reinforcement and repetition in order to shape the behaviour the teacher wanted. Furthermore, it introduced the audio-lingual method, where the whole class as a group chanted correct answers or key phrases. Furthermore, because of the importance of high-quality instruction, cognitivism contributed towards a better understanding of how to present materials more effectively. Essentialism still remains a popular philosophy in the form of people wanting to go `back to basics' or wanting more order in the classroom.

\section{Progressivism}

Progressivism go one step further than essentialists in a sense that new students should not only be taught to function in society, but to go beyond and improve society. This might seem like a small step, but is rather involved for it has its base in opposing authorianism instead of conforming to it.

It also has its root philosophy in experimentalism, where truth is not constant such as in idealism or realism, but rather is constantly in transition to a better understanding. Therefore, a progressivist curriculum focuses itself not on teaching already existing knowledge, but rather on the methods existing to discover knowledge such as the scientific method. This does not mean however that knowledge has become irrelevant. Students still have to be brought up to date with the newest developments in their field of interest, and thereby there is still some knowledge transfer necessary. The only difference is that this knowledge is never taught to be final, and the focus still lies within the transition and the parts still unknown.

Progressivists generally use more generative methods for instruction, such as enquiry learning, the scientific method and problem solving skills.

Starting from the philosophy of pragmatism of Peirce and James, progressivism became a serious contender for perennialism and essentialism in the 1920's, opposing their extreme authoritarian positions. As an educational practice, they grew larger with cognitivism and constructionism, where enquiry learning developed further and proved to be a more meaningful way of education. Yet, this approach was also criticised by the traditionalists, because it lacked rote learning and therefore could not be controlled, and was deemed highly inefficient for the students had to find out the wheel over and over again. However, progressivists argued that discovering truth is a very important part of learning, for it makes it meaningful and independent of an authoritarian truth. This idea of knowledge transmission also sprouted the idea of constructivism, a movement very close to progressivism.

\section{Reconstructivism}

There are a lot of similarities between progressivism and reconstructivism, such as both subscribing to experimentalism, moral and epistemological relativism, and the goal of improving society instead of conforming to society. Yet, reconstructivists differ from progressivists in the sense that they are more concerned with the ends than the means. Their goal is not to teach problem solving, but rather problem solving itself, and that society should be repaired. This emphasises the idea that the current society is broken, and focuses on social problems such as inequalities.

One might conclude that reconstructivism is thereby not different from the traditional perennialism and essentialism, because these philosophies also focus on the ends rather than the means. However, these philosophies still assume that the truth is absolute, unchanging, and provided by previous generations, whereas reconstructivism is still rooted in experimentalism and thereby states that the truth has to be discovered using the scientific method.

Reconstructivism stems from critical pedagogy, which is again based on postmodernism, anti-racism, feminism, and queer theories. This was first described by Paulo Freire, who was an educator and philosopher fighting for the less fortunate against the Brazilian dictatorship. Critical pedagogy was also applied in other countries with problems of social injustice and poverty, such as the Philippines and South-Africa during the apartheid. Reconstructivism was then created by Theodore Brameld, who advocated for using it in the US for avoiding facism and fighting the still prevalent institutionalised racism.

\section{Existentionalism}

Out of all described educational philosophies, existentionalism differentiates itself the most. Its core direction is towards individual self-fulfillment, and views education as an instrument for encouraging individual choice and autonomy. Not only does it oppose current authority, but it even goes far enough to state that there should be no authority, and that nobody should decide for students what to learn. It also states that what a person is capable of knowing and experiencing is more important than what he knows.

The main method of existentionalism is to put students into situations where they have to make meaningful choices, and to let them confront them alone in order to overcome personal crises so he develops selfreliance and overcomes despair. These are completely different from the methods used by other philosophies, since they do not rely on values preexistent to actions and thereby merely waiting to be discovered.

Existentionalism has seen the least progress in comparison to the aforementioned philosophies, both because of its relative novelty and its radical difference in methodology. It is also the philosophy which is most difficult to implement in current schools. One could even argue that existentialists are opposed to institutionalised education, since it revolves around self discovery and has a very anti-authoritarian viewpoint in the sense that no one should have the authority on deciding what students have to learn. One might argue that democratic schools are a form of an existentionalist curriculum, since here the students get to vote on the content they get to learn, and this school teaches democracy not from theory, but by experience. However, it is not a full realisation, for students do not learn by overcoming personal crises. Another form could be the Dutch \emph{Iederwijs}, a school where students are placed together in a learn-friendly environment and are allowed to do whatever they please. However, this \emph{laissez-faire} method of education still does not challenge the students in any way, which still would be part of existentionalism.

\section{Discussion}

\begin{table}[]
    \centering
    \resizebox{\textwidth}{!}{%
        \begin{tabular}{|p{2.5cm}|p{2.5cm}|p{2.5cm}|p{2.5cm}|p{3cm}|p{2.5cm}|}
            \hline
            \textbf{Educational Philosophy} & \textbf{Perennialism}                             & \textbf{Essentialism}                                                             & \textbf{Progressivism}                    & \textbf{Reconstructivism}                    & \textbf{Existentialism}                            \\ \hline
            \textbf{Function of knowledge}  & As a purpose on itself                            & In order to function in society                                                   & In order to improve society               & In order to change society                   & In order to discover oneself                       \\ \hline
            \textbf{Purpose of education}   & Preserving knowledge                              & Supplying knowledge                                                               & Supplying tools for discovering knowledge & Supplying tools for discovering inequalities & Encouraging maximum individual choice and autonomy \\ \hline
            \textbf{Philosophies}           & Classical idealism, realism                       & Idealism, realism                                                                 & Experimentalism                           & Experimentalism                              & Existentialism                                     \\ \hline
            \textbf{Subject matter}         & Classical literature                              & Three R's                                                                         & Scientific method                         & Social problems                              & Personal reflection                                \\ \hline
            \textbf{Methodology}            & Memorisation, reading, writing, drill, recitation & Reading, lectures, memorisation, repetition, audio-visual materials, examinations & Problem solving                           & Problem solving                              & Subjecting students to crises                      \\ \hline
            \textbf{Authority}              & Ancient works                                     & Teacher                                                                           & Science                                   & Socialists                                   & Student                                            \\ \hline
        \end{tabular}%
    }
    \caption{A comparative summary on the five educational philosophies \protect\cite{educationalphilosophy5}}
    \label{philosophies}
\end{table}

Table~\ref{philosophies} shows a comparitive summary on all above mentioned philosophies, giving an indication on the growing perspective on knowledge and learning methodology throughout history. In general the older philosophies, perennialism and essentialism, are labeled as the traditional philosophies, whereas the other three, progressivism, reconstructivism, and existentionalism, are often labeled as the modern philosophies. These two groups have the most apparent clashes: traditionalists place most trust in the current authorities where the modernists oppose them; traditionalists emphasise rote memorisation where modernists emphasise enquiry; and traditionalists want students to conform to society where modernists want students to change it.

Comparing these two general paradigms with the tools investigated within this thesis, the drill and practice used by the flashcards is most advocated for by the traditionalists, whereas the constructionist concept mapping technique fits mostly to the enquiry practice of the modernists. Flashcards are used by perennialists to memorise data such as dates and reproduction questions, and even more so by essentialists for drilling facts such as multiplication tables and spelling. Concept maps however would be used to shift the attention towards the meaning behind the surface concepts: progressivists use them to discover the ever expanding scientific body of knowledge, reconstructivists for demonstrating historical causality behind social inequalities and how these could be countered, and existentialists to let students map out their own experience and knowledge. However, this preference is not absolute, perennialists could for example also use concept mapping in order to let students figure out the arguments of Socrates in a philosophy assignment (an argument map), and a modernist could still use flashcards for drilling vocabulary.

It is important to consider the five educational philosophies when attempting to succesfully develop the new learning tool flashmaps which combines the flashcards and concept maps. For example, one might ask themselves the questions `what are the benefits of concept map visualisation of flashcards for essentialists' or `why would an existentialist want to memorise the concept map', but also more practical questions such as `should the concept map be provided to or constructed by the students' or `in which order shoud the student traverse through the map'. These are questions which have to be addressed during the design and development of the new tool.
 %Reviewed by Tobias
    \chapter{Cognitive theories}

\label{ch:theory}

This chapter aims to explain the effectivity and inner workings of both concept mapping and flashcard systems by elaborating on the physiology of the relevant parts of the brain, and the relevant cognitive theories. It is important however that these theories mainly focus on a certain type of learning only. According to \citeA{squire}, there are multiple varieties of memory, which can mainly be categorised into declarative and nondeclarative knowledge, sometimes also referred to as respectively explicit and implicit knowledge \citeA{cognitivepsychology}. Declarative knowledge also refers to memories that can be explicitly recalled, entailing facts such as definitions, paired associations etc., but also the events where these facts were acquired. Nondeclarative memory involves every memory which can be demonstrated in action, but not in conscious recall per se. Subcategories of these memories are procedural skills, priming, conditioning, and nonassociative memories. Because of the nature of this study, the cognitive theories discussed below are mainly focused on declarative knowledge, although most theories also are relevant to nondeclarative memory in some degree.

Furthermore, \citeA{instructionaldesign} describes declarative knowledge as one of Gagné's types of learning outcomes, and relates declarative knowledge to Bloom's levels of recall and understanding, meaning that declarative knowledge does not only encompass rote memorisation of facts, but also understanding the meaning behind this fact. This is also in line with the essay written by \citeA{glaserfield} on radical constructivism, in which it is stated that whatever it is that students are to place into memory they should also understand. Another category of learning outcomes applicable to this context is that of intellectual skills, mainly that of concepts. These, according to \citeA{instructionaldesign}, help the learners simplify the world and can make them into more efficient thinkers. From a cognitive perspective however, there is not a great difference in dealing with declarative knowledge or concepts, because both relate to explicitly recallable memories and thereby can both be considered as being explicit \cite{squire}.

\section{Storage and retrieval}

\begin{figure}
    \centering
    \includegraphics[width=0.5\textwidth]{img/brainareas.png}
    \caption{The brain areas mainly involved in storing and retrieving declarative knowledge \protect\cite{amnesia}}
    \label{fig:brainareas}
\end{figure}

Although the whole brain is involved in storing memories, the most prominent areas facilitating the process of memorising are the frontal lobes, medial septum and the hippocampus \cite{cognitivepsychology} (see figure~\ref{fig:brainareas}). The prefrontal regions are responsible for the creation and retrieval of memories, whereas the hippocampal and surrounding areas are responsible for permanent storage of these memories. Because of this dynamic, \citeA{modalmemory} conceived a modal theory of memory, displayed in figure~\ref{fig:modalmemory}. In this model, information is perceived as sensory input, and is then shortly stored in the sensory memory. If the perceiver has paid enough attention to the input, it is then transfered (or encoded) into short-term memory. When the input is strong enough, that is, rehearsed often enough within short term memory, it can be more permanently stored in long-term memory. If not, the input fades away from memory and is forgotten. When a memory exists in long-term memory, it has to be retrieved into short-term memory in order to be remembered and used.

This model was heavily influenced by developments in electrical engineering and computer sciences, and can be thought of as functioning like a complex computer, where data is written on a hard drive (the long-term memory), and can be used by first retrieving it into working memory (or short-term memory) and later be transferred to the hard drive again. However, the way the brain works is different from a computer in the sense that a brain has to put effort into memorising data, and that a brain forgets data over time. Therefore, instead of merely inputting the data, learning requires a more rigid approach.

\begin{figure}
    \centering
    \includegraphics[width=0.5\textwidth]{img/modalmemory.png}
    \caption{The modal model of memory proposed by \protect\citeA{modalmemory}}
    \label{fig:modalmemory}
\end{figure}

\citeA{karpicke4} describes two seperate learning practices based on the modal model of memory, namely encoding and retrieval practices, where encoding practices are focused on meaningful encoding or construction of knowledge, and retrieval practiced are more focused on the reconstruction and rehearsal of knowledge. He states that both practices are essential to enhancing learning. Flashcards are a famous retrieval practice, which emphasises drilling the same facts over and over again by means of pairs by association, whereas concept maps are known to be an encoding practice where the student has to connect diverse concepts within one topic by meaningful relations.

The following sections will elaborate on cognitive effects with regard to both encoding and retrieval practices, and relating them with their relevance to the effectiveness of concept mapping and flashcard systems respectively.

\section{Cognitive effects with regard to encoding practices}

The first step of memorisation always is encoding, because (logically speaking) it first has to be processed and encoded in either Short-Term or Long-Term Memory in order to be retrieved or used later on. After all, one cannot retrieve a memory which is not already there. It therefore is important to first acknowledge by which means knowledge is encoded, and in what kind of structure it is then stored.

\subsection{Early metaphors for the brain}

For centuries, a lot of metaphors describing memory characterised the brain as a room where a person could store physical things, such as a library filled with books or a storehouse with items \cite{roediger}. The fact that this metaphor seems intuitive and is easy to understand, it is one of the most popular metaphors, and is still prevalent today. It is for example explicitly used in many popular depictions of memory, such as in the film Inside Out (2015), where memories are stored as physical balls. There even exists a widely-used memorisation technique called the \emph{loci method}, which lets students envision a house where they have to store memories physically in the room. They can then later retrieve the memories by walking through the house and walking along the places they stored the memories at \cite{cognitivepsychology}.

Yet, there are certain flaws with this model. Firstly, with regards to retrieval practices, it depicts memories as static objects which only have to be stored there and consequently being remembered forever (although inside out already addressed this partly by showing the decay theory of memory by the balls being thrown away when never used, and illustrated strong flashbulb memories as being `core memories' stored seperately), misleading students, teachers and scientists into focusing more on encoding practices than on retrieval practices \cite{karpicke4}. Furthermore, it treats memories as existing as separate objects, which does not correspond with how memories are encoded in the brain. As a matter of fact, already in the 19th century, Cajal discovered that memories were patterns of electricity through neurons by means of synapses \cite{longtermpotentiation}. This enabled another spatial metaphor, namely that of a switchboard, where the synapses were represented by electrical wires \cite{roediger}. Later on, when the field of computer science begun to emerge, this metaphor transformed to that of a computer, enabling the conception of the modal model of memory. This is already a more useful metaphor than the physical space metaphor, since it is more biologically accurate, and it emphasises the need of communication between certain nodes (encoding and retrieval between the different memory systems).

However, the metaphor of a computer still has its flaws. A computer stores information on certain independent addresses in the form of binary data, and thereby implies that one can store data for later use without any need for comprehension of the data, and that the data can be formatted in any way the user would like to. Yet, the brain is differently structured, which has consequences for succesfull encoding.

\subsection{The brain as an associative network}

Unlike a computer, the brain is not organised with bits with physical addresses, but rather structured as an associative network. This entails the data being stored and retrieved by means of associated peers. In the brain, the neurons function as the nodes, and the synapses function as the edges. When information is encoded, new neurons are marked, and these are connected to other relevant, already marked neurons in the network. When something then has to be retrieved from memory, neurons signal relevant neighbouring neurons in order to activate the relevant parts of the brain. More generally speaking, when stimulated with a retrieval cue, the brain can then use neural pathways to find a corresponding item in the brain. These networks are sometimes referred to as \emph{semantic networks}, and the implication for retrieval as \emph{spreading activation} \cite{cognitivepsychology}. This effect has also been found on a cognitive level, for example \citeA{kintsch} has found that material is often not literally encoded, but rather as a set of abstract meaning units representing certain associations between concepts.

\subsection{Elaborative processing}

Because information is retrieved in the brain via related nodes and edges in the semantic network, strong neural pathways facilitate the retrieval process. One way of creating these pathways is elaborative processing \cite{karpicke4, cognitivepsychology}, which focuses on meaningful processing of the content. \citeA{craik} conducted an experiment where students were to freely recall from a list of words, where the students had to train the words by one of the following techniques: answering questions about structural details (e.g. is it in capital letters); about phonemical details (e.g. the word rhyming on another word); whether the word fits into a certain category; and whether the word fits in a certain sentence. They found that students had higher retrieval rates in ascending order of these techniques. Furthermore, research conducted by \citeA{nelson} presented students with paired associates that where either semantic or phonetic (in this case rhymes), and students showed a significantly higher recall of semantic associates. Both of these studies demonstrate the importance of meaningful processing for retention.

\subsection{Implications for concept mapping}

Reflecting on the previously described theory of associated networks, it appears that a semantic network is very similar in structure to concept maps, and thereby the maps provide an accurate representation of the way information is retrieved from the brain. \citeA{nesbit2} speculate that because of this, more and better retrieval cues are created when learning from or generating a concept map. Furthermore, a concept map displays the relations between certain concepts, and thereby focuses more on the meaning behind the content, rather than just the content itself. However, \citeA{karpicke4} also states that although it might seem to be an effective tool for elaborative processing, there has not been enough randomised and controlled experiments which examine the most effective ways to use is as a learning activity.

\section{Cognitive effects with regard to retrieval practices}

According to \citeA{karpicke4}, a lot of educational practices have placed an emphasis on finding optimal ways to encode knowledge and experiences, but that retrieval practices have received less attention. Nevertheless, basic research has indicated that retrieval is still important to consider in any analysis of learning. This is mainly due to the fact that information is not stored exactly and indefinitely, but rather that memories are forgotten over time. Two theories have been proposed and debated over explaining why forgetting occurs, namely by interference of other redundant memories and by decay of existing memories.

\subsection{Interference and Decay}

The theory of interference being responsible for forgetting has been demonstrated by an experiment by \citeA{interference}. The participants were asked to memorise sentences in the form \emph{A \textless person\textgreater{} is in the \textless location\textgreater}, where sometimes multiple persons where associated with only one location, and some locations with only one person. They found that if a sentence contained locations or persons with multiple associations this had an impact on the recognition time for that sentence, and even more so if both the location and the person had multiple associations. The explanation for this phenomenon is that since memories are retrieved by means of spreading activation and only limited activation can spread from one source \cite{cognitivepsychology}, the activity has to be divided over different branches in the semantic network, increasing the retrieval difficulty of the correct node. The increase in difficulty is also related to as the \emph{fan effect}.

The effect of decaying memories takes place in the connections between neurons, and therefore it is important to first examine how neurons communicate signals. Figure~\ref{fig:neuron} displays a schematic representation of a neuron in which it can be seen how the soma (cell body) is connected via an axon to the dendritic tree of other cells. The neuron can transmit stimuli by creating an action potential in the nucleus, transmitting this signal through the axon to the terminal button in the connected telodendrion (in the image refered to as the terminal aborization). There, neurotransmitters are released from vesicles, and after they have crossed the synaptic cleft there is a certain chance of being received by postsynaptic receptors. When this is the case, the nucleus of the receiving cell is triggered via the connected dendrite to also create an action potential, where the whole process is repeated \cite{longtermpotentiation}. The strength of a certain connection between neurons is therefore dependent on the action potential generated by a nucleus, the amount of telodendria over which the action potential has to be distributed (hence the aforementioned fan effect), the amount of neurotransmitters in the terminal button, and the amount of postsynaptic receptors in the dendrite of the next neuron.

\begin{figure}
    \centering
    \includegraphics[width=0.8\textwidth]{img/neuron.png}
    \caption{A schematic image of a neuron with a closeup of a synapse \protect\cite{website:neuron}}
    \label{fig:neuron}
\end{figure}

One widely studied effect with regard to the increase and decrease of action potential and strength of memory traces is called long-term potentiation (LTP) \cite{cognitivepsychology, longtermpotentiation, activationbasedmodel, amnesia}. Whenever a neurotransmitter is received by by a receptor, not only is the next nucleus activated to release its action potential, but also more receptors are activated, so that the postsynaptic membrane is able to receive more neurotransmitters at the next activation. Furthermore, another process is activated altering the dna in the neuron, causing it to create proteins for more stable increased sensitivity towards stimuli. It is also speculated that there might be a retrograde effect, causing presynaptci modifications such as the creation of more neurotransmitted in the presenaptic vessicles \cite{longtermpotentiation}. This all results in an increased sensitivity in the postsynaptic neuron towards action potential in the presynaptic neuron, which then again increases the stength of this particular memory trace. Over time, if a specific neural pathway is not used, the effects of LTP decrease again, causing its strength to decrease and thereby causing decay. This also is a predictor for the \emph{testing effect}, the effect of retrieval strengthening memory more than extra opportunities for further encoding, even when the retrieval is only carried out internally without any outward response \cite{microlearning}.

Although both the effect of interference and decay have been proposed as separate theories and have been debated, they are still mutually inclusive, and \citeA{cognitivepsychology} therefore conludes that forgetting results both from decay and from interference.

\subsection{Power laws of forgetting and learning}

Now that the relevant theories for learning and forgetting have been discussed, it is important to investigate with which rate people learn and forget. Already in 1885, Ebbinghaus discovered the power law of learning, referred to as the inversal exponential nature of forgetting \cite{microlearning, activationbasedmodel}. The implication of this model is that memory not only systematically deteriorates with delay, but also that this loss is negatively accelerated, meaning that the rate of change gets smaller with increasing delay \cite{cognitivepsychology}. \citeA{wickelgren} already proposed the formula $m = \lambda (1 + \beta t)^{-\psi}$, where $m$ is memory strength (the probability of recognition), $t$ is time, $\lambda$ is the state of long-term memory at $t = 0$, $\psi$ is the rate of forgetting, and $\beta$ is the scaling parameter (see figure~\ref{fig:powerlawforgetting}). This formula has also found to be accurate by \citeA{wixted}. Finally, the effect has been directly related to LTP in the rat hippocampus by stimulating neural pathways directly with electrical signals \cite{raymond}.

\begin{figure}
    \centering
    \begin{subfigure}{0.7\textwidth}
        \includegraphics[width=\textwidth]{img/powerlawforgetting}
        \caption{The power law of forgetting, with m as the probability of recognition and t as the time passed since learning}
        \label{fig:powerlawforgetting}
    \end{subfigure}
    \par\bigskip
    \begin{subfigure}{0.7\textwidth}
        \includegraphics[width=\textwidth]{img/powerlawlearning}
        \caption{The power law of forgetting, with p(t) as the probability of recognition and t as the iterations of learning}
        \label{fig:powerlawlearning}
    \end{subfigure}
    \caption{The power laws}
\end{figure}

A similar effect has been found for the effectiveness of repetition: \citeA{powerlaw1} have proposed a power law of learning, stating that a learning curve is inversal exponential (see also \citeA{powerlaw2} and \citeA{powerlaw3}). \citeA{murre} propose $P = p(t) = 1-e^{-\mu_{i}t}$ as a function describing this power law, where $P$ or $p$ is the probability of recognition after $t$ iterations and $\mu$ is the learning rate of student $i$ (see figure~\ref{fig:powerlawlearning}). The power law describes the effect that repetitions have a positive effect on retrieval probability, however that it diminishes with more repetitions. Again, this effect has also been demonstrated in the context of LTP in rat hippocampi \cite{barnes}. The stronger memory trace from a higher repetition rate does not only result in a higher recall probability, but also in a more gradual retention curve, allowing memories to persist longer.

\subsection{Spacing effect}

The spacing effect is a well known effect occuring within paired-associate learning, and demonstrates that repeated items are better remembered when both occurences are separated by other events or items than when they are presented in immediate succession \cite{verkoeijen, logan, siegel, xue, karpicke2}, which is demonstrated with diverse populations \cite{verkoeijen, logan}, under various learning conditions \cite{verkoeijen, logan}, and in both explicit and implicit memory tasks \cite{verkoeijen}. Items in immediate succession are called massed items, and items in separated succession are called spaced items.

One can test the spacing effect either by using pure lists or mixed lists. When using pure lists, one compares the effect of learning a list containing only massed items with a list containing only spaced items, and using mixed lists one measures the effect of learning both massed items and spaced items in one list, comparing their individual retentions. \citeA{verkoeijen} states that the vast majority of studies are conducted using mixed lists and found that spaced items where consistenly better recalled than massed items, yet studies using pure lists are relatively rare and have produced contradictory outcomes. They conducted a study providing participants first with an all-massed list, then letting them write down as many words as they could remember, and repeat an identical procedure for an all-spaced list with a 2 minute break inbetween. They conducted this experiment with short-lagged spaced items (with 1-4 items in between) and long-lagged spaced items (with 4-13), and found only a spacing effect in the latter experiment. However, \citeA{wahlheim} adds to this that repetition is only increases when a student detects the repetition of an item, and therefore the lag should not be too long.

Two theories have been presented explaining this phenomenon, namely the contextual variability theory and the study-phase retrieval theory \cite{siegel}. The first theory entails that because context is not static but continuous, and that therefore spaced items are studied in a greater variety of contexts and as such are easier to recall in yet other contexts than massed items due to the so-called encoding-specificity principle \cite{cognitivepsychology}. This principle entails that the probability of recalling an item depends on the similarity of the context during the encoding. The study-phase retrieval theory entails that additional retrieval cues for the repetition of an item are generated by earlier occurences and their associated contexts being associated with the repeated item. These theories are not mutually exclusive \cite{siegel}.

Inspired by the power laws of learning and forgetting, \citeA{karpicke} conducted an experiment to test the effect of constant or varying lags between items have a significant effect on learning. They tested this by conducting a similar experiment to \citeA{verkoeijen}, however in this experiment they only tested pure lists with three different lag intervals to test for an absolute spacing effect, and for each lag interval category they tested for an expanding lag condition (where the lag would increase for the repetition of each next item), an equal lag condition (where the lag would remain constant) and a contracting lag condition (where the lag would decrease for the repetition of each next item) in order to test for a relative spacing effect. From their findings they confirmed the effect of absolute spacing, namely that longer gaps between items do have an effect on long-term retention, yet they did not find a relative spacing effect. However, this has not been tested for spacing for longer intervals, such as intervals spanning multiple days or weeks.

\subsection{Implications for the flashcard system}

It can be concluded that the flashcard system derives its effects mainly from the testing effect by having students actively retrieve information instead of simply encoding it, and from the spacing effect by students going through the items interspersally instead of by immediate succession. The key question however is how often a single card has to be repeated. Herein one has to balance \emph{overlearning} -- the student repeating an item too often resulting only in diminished learning effects because of the power law of learning, and also only on the short term \cite{rohrer} -- because its inefficiency, and forgetting items in between intervals, since then the spacing effect does not apply anymore. In order to solve this problem, most modern digital flashcard systems apply a system called \emph{adaptive spaced-repetition learning} (e.g. the Pimsleur system, the Leitner system, Supermemo, and Anki \cite{microlearning}. In this system, exponentially expanding intervals are used, not because of a relative spacing effect which does not exist according to the previously mentioned literature, but rather to increase the average (absolute) spacing with each new repetition. This creates a stronger memory trace every time, but also takes into account the further decreasing risk of forgetting because of the slower declining retention curve (see figure~\ref{fig:spacedrepetition}).

\begin{figure}
    \centering
    \includegraphics[width=0.5\textwidth]{img/spacedrepetition}
    \caption{Adaptive spaced-repetition learning (taken from \protect\citeA{microlearning})}
    \label{fig:spacedrepetition}
\end{figure}

\section{Conclusion}

Overall, this chapter has discussed several cognitive theories related to the storage and retrieval of explicit (or declarative) knowledge in and from the hippocampus. Related to encoding practices, it has now been established that the brain works as an associative or semantic network, and that meaningful or elaborative processing is important for the later retrieval of memories. This seems to fit with the structure and process of concept mapping, although more research is needed in this area. Furthermore, the theories of interference and decay have been discussed in order to explain forgetting of memories, together with Long-Term Potentiation and its effects on the rate of forgetting and learning. In addition, articles were discussed demonstrating that spaced rehearsal is more effective than massed rehearsal. This has finally led to the conclusion that adaptive spaced-repetition learning is an effective method to expand absolute spacing, which entails that items are repeated with exponentially increasing intervals.
 %Reviewed by Tobias, Paulina

\part{Design Report}
    The \nameref{ch:problem} mainly described the needs which the Flashmap System might be able to accomodate, and on page~\pageref{sec:intro_flashmap} generic features of such a system are described. Although the term Flashmap System is intended for describing any system including these features, when having to evaluate the idea one has to evaluate one or multiple specific implementations of that idea. Therefore, this part specifies the design features of the specific tool developed within this project, along with arguments in favour of and against these choices and their considerations, and the process with which they are incorporated within the tool itself. The design process is based on the Generic Model \cite{genericmodel}, which is displayed in figure~\ref{fig:genericmodel}.

\begin{figure}[h]
    \centering
    \includegraphics[width=0.5\textwidth]{img/genericmodel}
    \caption{The generic model by \protect\citeA{genericmodel}\label{fig:genericmodel}}
\end{figure}

The first chapter, \nameref{ch:analysis} on page~\pageref{ch:analysis}, describes the design implications stemming from the context, learner, and learning task of this project. The next chapter, \nameref{ch:frameworks} on page~\pageref{ch:frameworks}, describes implications from theoretical design frameworks relevant to the design of the flashcard and flashmap system. Finally, the \nameref{ch:software} chapter on page~\pageref{ch:software} describes the actual implementation of the software following from the design implications of the previous chapters.
 %Reviewed by Tobias TODO: Enlist separate aspects of the product
    \chapter{Analyses}

\label{ch:analysis}

Before designing an educational product, it is important that the designer first acquaints himself with the extrinsic factors important to this product. In order to discover the important characteristics of these factors, \citeA{instructionaldesign} enlist three types of analyses to be conducted, together with steps for conducting them. These are an analysis of the context, of the learner, and of the learning task. Although these analyses are more targeted towards instructional design, and therefore more focused on a specific group being taught specific content, these analyses still provide relevant information for the design choices and the evaluation. However, the steps are adjusted and generalised or even omitted in order to fit the design of the more generic learning tool. The information gathered in order to conduct these analyses mainly stems from meetings with one of the teachers. This might not be the most reliable source of information because of the lack of triangulation, and should therefore not be taken as insight in the curriculum of Dutch Literature courses in secondary education, but rather as context information relevant to the design. The most important findings are described within this chapter, along with their implications for the design of the software.

\section{Analysis of the learning context}
\label{sec:contextanalysis}

As already stated in the \nameref{sec:intro_evaluation} section on page~\pageref{sec:intro_evaluation}, the evaluation of the flashmap system will be evaluated within the Dutch secondary school Stedelijk Lyceum, with students having to learn about the Renaissance genres in Dutch Literature. Although the general needs for a flashmap system are shortly provided within the \nameref{sec:intro_evaluation} on page~\pageref{sec:intro_evaluation}, it is still important to investigate the specific needs of the context where the programm will be implemented. Therefore, this context will be further investigated within this section, starting with the Needs Assessment \cite{instructionaldesign}. 

\subsection{Needs assessment}
\label{subsec:needsassessment}

There are multiple reason why teachers think it is important to learn about Dutch renaissance literature: one could argue that this way the knowledge is passed onto a new generation, keeping it relevant; understanding the history of literature is important for understanding modern literature; and literature can provide certain insights to the individual reading it \cite{opinion1, opinion2}. Furthermore, the school is extrinsically motivated to teach the subject matter, since subdomain E2 and E3 within the Dutch national exam programm state that a student has to recognise and distinct between literary text genres, apply literary concepts in the interpretation of literary work, provide the outlines of the literature history, and place literary works in this historical perspective \cite{eindexamen}.

The teacher did confirm the need for better retention and comprehension of the content already stated within the \nameref{ch:problem}, since she indicated that most of the time the students only learned the night before the exam in order to get a high (enough) grade and consequently forget everything again. Both using a flashcard system and the flashmap system should accomodate this need. Furthermore, she deemed them to become more familiar with the Dutch Renaissance writers or work to be the most important, entailing that students recognise impotant names or that they can distinguish between different genres. Based on these statements, the goals within the context are in line with students memorising and understanding all of the facts, without them being too ambitious. Finally, the teacher provided a test from the previous year to offer some more concrete examples of what she wanted the students to know, of which an English translation is included in the appendix on page~\pageref{app:exampletest}. From this test, more goals can be extrapolated, such as students having to not only distinguish different genres, but also having to define them or provide characteristics, and recognise the application of these features in both examples of the time periods as well as modern examples. Furthermore, they have to be able to relate the famous writers and writings to the genres.

\subsection{The learning environment}
\label{subsec:learningenvironment}

The Stedelijk Lyceum is an open denominated school organisation, consisting of 7 schools on different locations. The school approached within this project has been approved by the Dutch Inspection of Education \cite{inspectierapport}. The course on Dutch renaissance literature consists of two different types of learning activities, which are classroom instruction, and individual learning at home by the students. There are two sessions of classroom instruction, both lasting 50 minutes, in which the 100 students are divided over the three teachers in static groups on separate locations. These lessons take place over the course of two weeks, with one lesson provided in one week. Within these lessons, the teachers transfer knowledge and provide excersises for the students. Outside of the lessons, the students still have to study the textbook Laagland individually \cite{laagland}, which contains all of the materials which will be prompted on a final written assessment. As already stated before, the teacher indicated this activity mostly to take place on the evening before the assessment, and only on a superficial level. Finally, this assessment takes place in the second week after the final instruction, and will be similar to the example test included in the appendix on page~\pageref{app:exampletest}.

The teacher stated that the course mainly consisted of the rote memorisation of facts, and that she was still doubtful whether the students would actually be willing to participate in the evaluation of the Flashmap system. Yet, she did see the general use of the tool for achieving the learning goals, and therefore still seemed to be enthusiastic in cooperation and encouraging the students to participate. The only two technical problems are that there is not too much time for extra activities within the lesson plan and the teachers being quite busy themselves, and that the technological possibilities within the classroom are limited. Within the classroom, only a couple of computers are available for use, and still run relatively old software. Therefore, the activities envolved in using the flashmap have to target the individual learning of students, since they have more time outside of the lesson plan, and mostly do possess the hardware and software necessary to run the software.

\section{Analysis of the learner}
\label{sec:learneranalysis}

\subsection{Physiological characteristics}

The physiological characteristics of the respondent's brain provide important implications for the design of the software. They are enrolled in grade 4 of Dutch secondary education, and therefore should be around the age of 16-17, with some deviations due to students either having skipped or repeated a grade. Therefore, the students are generally considered to be either at the end of puberty, or the beginning of young adolescence. The \nameref{ch:theory} chapter on page~\pageref{ch:theory} already provides general theories about the learning process within the brain. However, during late puberty and early adolesence, the brain is still heavily in development, especially the prefrontal cortex. \cite{blakemore}. In order to map out the changes in the adolescent brain, \citeA{giedd} performed a longitudinal MRI study of the brain development during this period, where three themes emerged within the adolescent development of the brain:

\begin{enumerate}
    \item After a peak in growth of both brain cells, connections and neurotransmitters during childhood, one can see a decline in adolescence;
    \item The connectivity between different regions of the brain increases;
    \item A new balance is formed among frontal and limbic lobes.
\end{enumerate}

The first theme is also known as peak plasticity, after which a decrease can be observed. \citeA{powell} describes this phenomenon as \emph{Use it or lose it}, since the brain rigorously selects the specific memories which are activated during this time. The second theme refers to the strengthening of specific memories, which are enhanced during that period. Here the flashcard system proves to be a useful tool, since it focuses on repeating specific associated pairs that the learner wants to remember.

Finally, during adolescence a shift is made from ``cold'' to ``hot'' cognition, where the former relates to hypothetical, low-emotion reactions, and the latter to high arousal decision making, strongly influenced by peer pressure and real, direct consequences. This is highly related to the prefrontal cortex being heavily developed, resulting in the teenage brain to rely more on the amygdala which is the more emotional, impulsive area of the brain. This means that for students to be motivated to learn, they either need a strong intrinsic motivator, or they have to rely on what \citeA{powell} describes as an ``external prefrontal cortex'', which can be either a reward or a person reminding them to study (e.g. the teacher or a parent). Therefore, extrinsic factors such as the usefulness of the system to passing the school test, a voucher for icecream, and the teacher are used to motivate the students to use the system.

\subsection{Cognitive characteristics}

All students should have learned about the relevant time period in their history classes prior to this course (e.g. the Spanish War, the Lutheran reformation etc.), providing the relevant knowledge to understand the context of Dutch renaissance literature. Additionally, the students have received a similar instruction on Dutch medieval literature, which is also relvant for concepts in the renaissance literature, such as the \emph{Mecenas}, the \emph{Lyriek} and \emph{Rederijkers}. Therefore, these concepts form the root concepts from which to start within the concept map.

However, one difference between students is whether they chose technical or society-related subjects. This makes up for different specific aptitudes within this specific Dutch literature course. Furthermore, some of the students are also enrolled in classical subjects, and because the Dutch renaissance literature has a lot of connections with classical genres, these students might have an advantage in prior knowledge. Both technical and society-oriented profiles, and classical and non-classical profiles are therefore accommodated for within the concept map.

\subsection{Social characteristics}

\citeA{grever} provide information on the perspectives on learning history by Dutch, English and French high school students. Within this study, students were asked several questions about what kinds of history, which periods of history are important or interesting for the students, and what the meaning of history is for their personal lives and what they believe to be its relevance for society. For Dutch students, this study found that the history of ones own family generally ranks high, and after that the history of the country where the parents come from (both for natives and non-natives). This means that native students might be more interested in learning about the subject than non-native students. Furthermore, the history of ones own religion is mostly important for Moroccan and Turkish students (which are mostly muslim), so the history of christianity is generally not that interesting towards most students. The study also found that the time period of early modern history is the least interesting for students, no matter the gender or nationality, despite that in the Netherlands the most important topic is the rise of the Dutch republic and the Golden Age (the content of the subject used within this study). Finally, the study states that there were no significant differences in perceptions of pre-vocational students and HAVO/VWO students in these respects, although one might expect Gymnasium students to be more interested in the classical revival of art during the renaissance than the Atheneum students.

\section{Analysis of the learning task}
\label{sec:taskanalysis}

Finally, the characteristics of the task itself will be investigated in order to learn how to design a tool in such a way that it facilitates or augments the learning process. \citeA{instructionaldesign} enlist primary steps for performing a learning task analysis, which are writing a learning goal, determining the types of learning of the goal, conducting an information-processing analysis of that goal, conducting a prerequisite analysis and determining the type of learning of the prerequisites, writing the learning objectives for the learning goal and each of the prerequisites, and writing the test specifications. However, within this project the instruction has already been written \cite{laagland}, and only has to be used to create a concept map. Still, knowledge of the underlying structure of the instruction might prove to be helpful for finding the relevant elements, and it is also useful to investigate the specific uses of the instruction within the context of this project.

\subsection{Learning goals}

The direct learning goals of the instruction can be found in paragraph 13.4 of Laagland, the specific instruction for the Dutch renaissance literature, where the previous paragraphs only provide the prerequisite knowledge necessary to understand this paragraph. The different chapters describe the \emph{emblematiek}, the \emph{lyriek}, the sonnet, and the different theatrical genres (the tragedy, the comedy, and the \emph{klucht}). One of the goals of this instruction is that the students are able to describe these genres, and are able to differentiate between the subgenres or terminology within these genres. However, the students also have to be able to relate these genres to the general context described in the previous chapters, which consist out of the political, the socioeconomic, and the cultural backgrounds.

\subsection{Types of learning}

Attaining these skills are mainly intellectual in the typology defined by \citeA{instructionaldesign}, because the students mainly have to be able to describe and discriminate between defined concepts. However, there is also a certain amount of declarative knowledge learning involved because students have to first learn and memorise certain definitions or conceptual organisations. Furthermore, within the book there are not only abstract concepts being defined, but also declarative knowledge such as names of important authors (Vondel, Bredero etc.), books or plays (e.g. the \emph{Klucht van de koe}), and certain historic events such as the migration of calvinists from Antwerpen to Amsterdam in 1585.

\subsection{Concept map}

Because the information has already been defined within the textbook (both the new content and prerequisite content), the information-processing and prerequisite analysis activities have been replaced by translating the content of the instruction within the textbook to a concept map. Within this map, not only the relevant concepts, names and events are presented, but also the relations between them, providing a more meaningful representation. Furthermore, the concept map also contains information about the order in which the concepts have to be learned, because of the direction of the relations. The data used for the concept map is uploaded on github\footnote{\url{https://github.com/mcvdenk/MasterThesis-Software/blob/master/database/concept_map.json}}. A direct visualisation is too extensive to be feasibly included within this report, however a digital visualisation is available\footnote{\url{http://www.mvdenk.com/thesis/concept_map/}} (after a short initial rendering time due to its size). The \nameref{sec:cmapframework} section on page~\pageref{sec:cmapframework} will elaborate further on the design choices for the concept map. Finally, this map is directly shown to the students within the flashmap condition during the experiment.

\subsection{Flashcards}

The activity of specifying the learning objectives is replaced by formulating the flashcards, because the flashcards already form the specific knowledge-based learning objectives. They already contain the most important types of information which should be included in an objective, namely the statement of the terminal behaviour (the answer itself), the conditions of demonstration (given this question, the student can reproduce the correct associated answer). The standards or criteria for these objectives are globally defined, namely that the student has to be able to demonstrate that he knows the correct concept corresponding to a parent node and edge label. The flashcards are directly based on the previously defined concept map. Within this activity, edges and their corresponding parent nodes were transformed to a question, and the child node formed the answer to that question. For example, the nodes \emph{Strijdliteratuur} and \emph{Actualiteit}, respectively connected by the edge \emph{verwees naar}, is translated to a flashcard "Q: Waar verwees de Strijdliteratuur naar?" $\rightarrow$ "A: Naar de actualiteit" (\emph{Translation:} To which did the war literature refer? To actual events). Sometimes, multiple edges from one node to several child nodes having the same label or falling within the same category were translated to only one single flashcard. The data for the flashcards can be found again on github\footnote{\url{https://github.com/mcvdenk/MasterThesis-Software/blob/master/database/flashcards.json}}.

\subsection{Test specifications}

The assessments conducted before and after the students have used the learning tool consist partly out of the questions from the flashcards for measuring knowledge reproduction, but also partly of questions targeted to measure the comprehension levels of the students (see \citeNP{bloom}). On both assessments for all questions, the students are asked to fill in an answer in a textbox. In order to answer the questions for comprehension, a student has to be able to draw relations between not directly linked nodes, and thereby requires a higher degree of mastery of the content. It does however not yet contain any questions where students have to apply the content within different context, or have to think outside of the content directly taught, since these questions would rate on even higher levels on the taxonomy of Bloom. Finally, the questions are phrased according to the specified action verbs related to the level of learning. A more detailed elaboration of the test construction and analysis can be found in the \nameref{sec:instrumentation} section on page~\pageref{sec:instrumentation}, and all of the comprehension level questions are included on github\footnote{\url{https://github.com/mcvdenk/MasterThesis-Software/blob/master/database/itembank.json}}.
 %Reviewed by Tobias, Paulina TODO: comments about social characteristics by Tobias
    \chapter{Defining the general use cases}
\label{ch:usecases}

\section{Supplantive or generative}

The first important design choice which has to be made is whether the students are supplied with a concept map or flashcards, or that they generate the content themselves. The dichotomy of generative versus supplantive instruction is described in further detail by \citeA{instructionaldesign}, where the implications of both sides are enlisted for the learner, the task and the context.

One of the aspects of generative strategies is that the learner requires a higher amount of prior knowledge, a higher aptitude, and a wider and more flexible range of cognitive strategies, because the content still has to be (partly) researched and constructed. This can be a disadvantage, because the learner might not possess these skills and therefore the instruction may not be suitable or highly inefficient using generative strategies. On the other hand, greater mental effort generally leads to greater depth of processing and therefore better, more meaningful learning, which was also stressed by \citeA{canas} and \citeA{nesbit}. Furthermore, learners experience a higher motivation and a lower amount of anxiety when using generative strategies, and their attribution of success is internal rather than external. 

Furthermore, when using more generative strategies, the learning task becomes more complex and ill-structured, and therefore requires more instruction and time to complete. It also leads to a higher focus on cognitive strategies, but less so on the learning goals. These goals can also not become universal, since each student creates their own flashcards or concept map, and therefore decides on their own learning content.

The most important factor for this design choice is feasibility. The teacher already stated that there is only limited time available during the lesson to introduce them to the software, so there is no time for extensive instruction on how to create concept maps, let alone creating the maps within the classroom. Additionally, students do not have much time at home to spend on creating the maps, and it is also known from both interviews with the teacher as literature that they will probably have only a low amount of intrinsic motivation. Finally, when the students have to create their own maps, it cannot be guaranted that they will include the nodes relevant for the goals of the instruction, and might become either to narrow or to extensive in certain branches. The same arguments are valid for letting students create their own flashcards. Therefore, despite of the benefits that a more generative approach may have for the learning process, the content will be supplied to the students instead.

\section{Choice of platform}

The next design decision is centered around the choice of platform or medium going to be used in order to support the learning tool. In the section \nameref{subsec:fcapplication} on page~\pageref{subsec:fcapplication} it is described that students generally prefer to use traditional or written flashcards, despite the many advantages of digital flashcards. However, with the flashmap tool this is not a feasible option, since the tool has to dynamically generate different graphs based on the general concept map and the profile of the students. Of course it would be possible to provide the concept map digitally and the flashcards in written form, however this would introduce an extra variable to the research design. Finally, with written flashcards one can only use rather crude methods for rescheduling the cards, instead of using the more precise algorithms possible within a digital tool.

There are various options for the specific implementation, for example a computer program or an app. Of these options, a web application is the most convenient, since it is accessible for any device with a modern web browser, and immediately stores the usage data on a centralised server so that it is immediately accessible for the researcher. Furthermore, adjustments or fixes can be applied during the research, without all users having to update to the newest version.

For the client, HTML, CSS, and Javascript are used, importing the vis.js library for visualising the concept map dynamically\footnote{\url{http://www.visjs.org/}}. Furthermore, Python is used for the server logic, communicating with the webclient through a websocket using JSON messages. The choice for Python is mainly based on preference by the programmer. Finally, MongoDB was used as a database engine since it stores data in a format very similar to JSON, which is also used by vis.js to represent concept maps.

The server implementation will be further elaborated in the \nameref{ch:server} chapter on page~\pageref{ch:server}, and the client implementation in the \nameref{ch:client} chapter on page~\pageref{ch:client}.

\section{Supported user actions}

The final design decision related to the general ideation of software is deciding which use cases should be supported, which are generally displayed within a UML use case diagram \cite{uml}. For the flashmap software, the use cases are divided in cases related to the registering and login process (see figure~\ref{fig:loginusecase}), and the cases related to the main use of the software (see figure~\ref{fig:mainusecase}).

\begin{figure}[h!]
\centering
\includegraphics[width=\textwidth]{img/loginusecase.png}
\caption{Use case diagram for registering and logging in}
\label{fig:loginusecase}
\end{figure}

\paragraph{Login use cases} When opening the webapplication, the user is first prompted with a login screen. Here, th e user can either enter an already existing username to continue this session, or he can enter a new name in order to register as a new user. When the user is registering as a new user, a form is presented asking for information on gender and birthdate as descriptive information, and asking for the specific code the user received on the informed consent form in order to validate that the user indeed signed this form before partaking in the research (see section \nameref{sec:procedure} on page~\pageref{sec:procedure}). After that, another form will be prompted for the pretest (section \nameref{sec:instrumentation} on page~\pageref{sec:instrumentation}). When the user has met certain criteria, a posttest similar to the pretest will be prompted, followed by a questionnaire and a debriefing text. When none of these criteria are met, the user can access he main use cases.

\begin{figure}[h!]
\centering
\includegraphics[width=\textwidth]{img/mainusecase.png}
\caption{Use case diagram for main purposes}
\label{fig:mainusecase}
\end{figure}

\paragraph{Learning use cases} The main use cases entail requesting items for review, requesting the learning progress, or logging out. When requesting items for review, the user can receive a due or new flashcard or flashmap, depending whether there are any old items due for review and the experimental group the user is in. Alternatively, the user can also be prompted whether a certain section of the instruction material has been read, since the rehearsal of items cannot be meaningful when the user is not familiar with the content. These prompts take often place at the beginning of a session so that the user does not have to interrupt a session. Furthermore, they prompt two sections ahead of the material currently being learned or reviewed by the user from the flashmap or flashcards in order to guarantee that the user is familiar with the material before learning the items. The user is prompted to read a section at most once per day. After the user has submitted a response, he can undo this response if he is not content with it. For example, the user could after seeing the correct response decide that he thought of a similar enough answer, but after deeper reflection still decide that his answer was not sufficient. In this case, he could use the undo option in order to be presented with the previous response again and select the 'incorrect' option.

\paragraph{Learning progress} When requesting the learning progress, the user is presented of an overview of what has already been learned and what is still left as either unseen items or items due for review. This provides an indication for the user so that he can estimate how much time he still needs to invest into the software, but also could stimulate the user by seeing the number of new or due items lowering and learned items increasing.

Finally, the user can return to the login screen by logging out.

\section{Detailed description of the client server interaction}

Based on the previous description of use cases, there are two sets of complex interaction between the client and server, which are again the interaction for the login and registering process, and the interaction for the learning process. These are described as activity diagrams in figure~\ref{fig:loginactivity} and figure~\ref{fig:learningactivitygen}. These diagrams are elaborated on below together with the specific network messages belonging to the interaction step. Each network message is a simple JSON message consisting of a \emph{keyword} field --- containing the main function which has to be performed by the other party --- and a \emph{data} field containing a dictionary with necessary supplementary data.

\subsection{Login activities}

\begin{figure}[h!]
\centering
\includegraphics[height=\textheight-10ex]{img/loginactivity.png}
\caption{An activity diagram displaying the server-client interactions when a user logs in}
\label{fig:loginactivity}
\end{figure}

The exact reasoning behind the different activities can be found in the \nameref{sec:procedure} and \nameref{sec:instrumentation} sections of the \nameref{ch:methods} chapter on pages~\pageref{sec:procedure} and \pageref{sec:instrumentation}.

\paragraph{Authenticate} When the user logs in, the client sends a message with keyword "AUTHENTICATE-REQUEST" and data containing a name field with the username. When a user with this username does not exist yet, the server creates a new user with a randomly assigned condition (either flashcard or flashmap, also known as control or experimental). When this user already exists, the server fetches this user from the database.

\paragraph{Descriptives} The server then checks whether the user already has set description fields. If not, the server returns a message with keyword "DESCRIPTIVES-REQUEST", on which the client responds with keyword "DESCRIPTIVES-RESPONSE" with the data fields gender, birthdate, and code.

\paragraph{Pretest} When the previous condition is met, the server will check whether the user has a registered pretest. If this is not the case, it will create a new test by randomly selecting 5 items from the flashcard dataset and 5 items from the itembank, which it will then send to the user with the keyword "TEST-REQUEST". After the user has answered the questions, the client sends the responses to the server with the keyword "TEST-RESPONSE".

\paragraph{After the experiment} If the previous condition is also met, the user will be directed towards the main application with an "AUTHENTICATE-RESPONSE" message from the server, unless he has used the software for at least 15 minutes on 6 days. In that case the checks described below will be performed.

\paragraph{Posttest} In this step the server checks whether the user also has a posttest entry. When this is not the case, it will send a similar test message to the pretest message, with the exception that the flashcards and test items from the pretest are excluded from the random selection in the posttest.

\paragraph{Questionnaire} Sequentially, if the user has no questionnaire entry, the server will construct a new questionnaire by shuffling the Perceived Usefulness item, randomly selecting for each item whether it is positively phrased or negatively, copying this item set but with the opposite phrasing, and finally shuffeling the second set. The same is done for the Perceived Ease of Use items. This questionnaire is then sent to the client with the "QUESTIONNAIRE-REQUEST" keyword. The client sends a filled in version back with the "QUESTIONNAIRE-RESPONSE" keyword to the server with an extra textfield for what was good about the software, what could improved about the software, and an (optional) emailadress of the user for an interview at a later time.

\paragraph{Debriefing} Finally, if the user has not debriefed before, the server sends a message with the keyword "DEBRIEFING-REQUEST" to the client, which will show a debriefing message to the user and returning a message with the keyword "DEBRIEFING-RESPONSE". When all the checks are met, the user will be directed to the main application with the "AUTHENTICATE-RESPONSE" message.

\subsection{Learning activities}

\begin{figure}
\centering
\includegraphics[width=.9\textwidth]{img/learningactivitygen.png}
    \caption{An activity diagram describing the general server-client interactions related to reviewing items. The \protect\emph{Provide answers} activity is described more detailed in figure~\protect\ref{fig:learningclient} on page~\protect\pageref{fig:learningclient}, and the \protect\emph{Store answers} and \protect\emph{Reschedule} activities in figure~\protect\ref{fig:learningserver} on page~\protect\pageref{fig:learningserver}}
\label{fig:learningactivitygen}
\end{figure}

In the main application view, the user can either review items or view his learning progress. If he chooses the latter, a message will be sent to the server with the "LEARNED\_ITEMS-REQUEST" keyword, to which the server will respond with a "LEARNED\_ITEMS-RESPONSE" message containing information on the learning progress (see the \nameref{sec:learningprogress} section on page~\pageref{sec:learningprogress}). If the user wants to review items, the client will send a "LEARN-REQUEST" message to the server and the process described below is performed.

\paragraph{Aimed time reached} First, the server checks whether the user already spent 15 minutes learning today. If this is the case, the client will display a message that the user has spent an sufficient amount of time on learning for today. This will not be directly be displayed as the activity diagram suggests, but rather it will show this message together with the next item.

\paragraph{Selecting an item} After this, the server will check whether there is any item already due for review. If this is the case, the server will sent the item which is due for the longest time to the client with the keyword "LEARN-RESPONSE". If not, the server selects a new item from the database. It is then checked whether the user already read the section in the book related to this item. If not, the server sends a "READ\_SOURCE-REQUEST" to the client, which prompts the user whether he has read the source supplied in the source field of the message. If so, the client sends a "READ\_SOURCE-RESPONSE" message back to the server, which adds the supplied source to the list of read sources for the user. When the user has read the section , the server will sent a "LEARN-RESPONSE" message with a new item from the database. If there is no new item left, the server sends a message to the client with the keyword "NO\_MORE\_INSTANCES".


\paragraph{Validating the item} When the user has reviewed the item, the client sents a "VALIDATE" message to the server with an 'id' field with the item id and a 'correct' field with information whether the user was able to think of the correct response for the item. This response will then be stored in the database by the server, and the item will be rescheduled for when it is due for the next review. After this, the server will repeat the learning cycle as if the client just sent a "LEARN-REQUEST" message.

\paragraph{Undo previous response} When the supplied item is not the first item within the current learning section, the user can choose to undo the response of the previous item. The client will then send back a "UNDO" message to the server, which will then remove the previous response and also go back to the beginning of the learning cycle.
 %Outlined
    \chapter{Server design and development}
\label{ch:server}

In this chapter, the general model behind the server will be elaborated. The formal documentation with descriptions of the server classes and their methods can be found on page~\pageref{app:documentation}, which also includes an UML class diagram \cite{uml}. An HTML version is also available on \url{mvdenk.com/thesis/doc/}. The description of the model is divided in generic data entries --- entailing the supplanted data by the developer such as the concept map or test items --- and the user attributes, objects and methods --- entailing user specific data such as birthdate or how often a user responded correctly or incorrectly to a certain instance. Only the conceptually relevant methods will be described below, since this keeps the thesis more accessible for readers not familiar with programming paradigms, and these methods are already described within the documentation. For example, the to\_dict() method is prevalent in a large amount of classes, but merely serves the purpose of converting the class into an object that can be sent over the network connection to and rendered within the client and therefore is not conceptually relevant and is not elaborated. The two classes left out of the description are the Controller class, which parses client messages, operates the server and translates the server output to a new network message, and the LogEntry, which represents a single incoming or outgoing network message. These classes are also not conceptually relevant, and their functionality can already be derived from the previously described UML Activity Diagrams in figure~\ref{fig:loginactivity} on page~\pageref{fig:loginactivity} and figure~\ref{fig:learningactivitygen} on page~\pageref{fig:learningactivitygen}.

\section{Generic data entries}

There are four classes of generic data: the concept map (containing nodes and edges), the flashcards, the test items and the questionnaire items.

\subsection{Concept map}

The ConceptMap class is mainly a container class consisting of nodes and edges, and certain useful methods for performing standard queries on the concept map. As described in the general definition of the concept map by \citeA{canas}, a Node object represents a concept, and an Edge object represents a relation between two concepts. The Node and Edge were originally intended to be embedded within the ConceptMap class, however this makes them impossible to refer to by other classes in MongoEngine (used for interacting with the MongoDB database). Furthermore, it could theoretically be possible for certain Nodes to exist in different concept maps. Therefore, they are implemented as separate classes in this server model. The example concept map in figure~\ref{fig:examplemap} is used to demonstrate the different functions in of the class.

\begin{figure}
    \centering
    \begin{subfigure}{0.4\textwidth}
        \centering
        \includegraphics[height=.5\textheight]{img/examplemap.png}
        \caption{An abstract example of a concept map}
        \label{fig:examplemap}
    \end{subfigure}
    \qquad
    \begin{subfigure}{0.4\textwidth}
        \centering
        \includegraphics[height=.5\textheight]{img/examplemap_objects.png}
        \caption{The different attributes demonstrated within the example map from figure~\protect\ref{fig:examplemap}}
        \label{fig:examplemap_objects}
    \end{subfigure}
    \caption{An example illustrating the ConceptMap class}
    \label{fig:examplemaps}
\end{figure}

\paragraph{Nodes} The concepts represented by the nodes are not only an abstract ideas (such as Renaissance Literature), but also more concrete concepts such as a time period (e.g. the Golden Age), a person (e.g. P.C. Hooft) or objects or inventions (e.g. the printing press). Nodes are simple classes only containing a label describing what it should represent and a unique identifyer string.

\paragraph{Edges} An Edge also contains a label describing the specific relation between two concepts and an id, but also contains the references to the id's of the nodes it refers to, one being the `from' node and the other the `to' node, and a list of sources (usually only containing one source), which are the sections of the instructional material the edge is derived from. The model does not make any destinction between a hierarchical and cross-link, since they are the same from a graph rendering perspective. However, this destinction might still be considered for more sophisticated hierarchical rendering or searching algorithms.

Figure~\ref{fig:examplemap_objects} demonstrates how the different classes and attributes represent the concept map from figure~\ref{fig:examplemap}.

\paragraph{Methods} The most important method of the ConceptMap class is the get\_partial\_map() function, which will provide a new ConceptMap object containing all the parent nodes and edges which directly or indirectly link to a given Node (the parent nodes and edges), plus the nodes and edges linked to by the direct parent nodes (the sibling nodes and edges). This concept map can then be displayed to the user when showing a specific flashmap instance for review. The reason why the parent nodes are returned rather than the child nodes is that in the instructional material the concepts are introduced top-down rather than bottom up, so building up the concept map from parent to child alligns better with the order in which the students read about the concepts. Additionally, the sibling nodes are also returned so that they can be prompted at the same time, and that the user has more context for deciding which concept should be filled in the missing node. Figure~\ref{fig:examplemap_partial_d} and figure~\ref{fig:examplemap_partial_g} demonstrate the get\_partial\_map() function with Node D and Node G from the example map in figure~\ref{fig:examplemap}.

\begin{figure}
    \centering
    \begin{subfigure}{0.4\textwidth}
        \centering
        \includegraphics[height=.4\textheight]{img/examplemap_partial_d.png}
        \caption{The result of get\_partial\_map() function with Node D from the example map from figure~\protect\ref{fig:examplemap}}
        \label{fig:examplemap_partial_d}
    \end{subfigure}
    \qquad
    \begin{subfigure}{0.4\textwidth}
        \centering
        \includegraphics[height=.4\textheight]{img/examplemap_partial_g.png}
        \caption{The result of get\_partial\_map() function with Node G from the example map from figure~\protect\ref{fig:examplemap}}
        \label{fig:examplemap_partial_g}
    \end{subfigure}
    \caption{Example for using the get\_partial\_map() function of the ConceptMap class}
    \label{fig:examplemap_partial}
\end{figure}

\subsection{Flashcards}

A Flashcard class represents a traditional flashcard by simply having a question and answer entry. It addtitionally has an response model entry in order to also function as a test item. In most cases, this is a list only containing the answer entry, however in some cases the answer entry is split into multiple response entries. Finally, since each flashmap is based on the concept map, each Flashcard object also contains a list with Edges the card is derived from. This has the advantage of being able to compare the flashcards with the concept map, but also indirectly relates the flashcards to the sections within the instructional material.

\subsection{Test items}

A TestItem object represents an item on the pre- or posttest. It is very similar to a Flashcard object, with the exception that it directly links to the text sources, and does not contain an answer field since this is never displayed to the user.

\subsection{Questionnaire items}

The QuestionnaireItem class represents items from the Technology Acceptance Model questionnaire \cite{tamq}, and contains a usefulness entry categorising the item as either a Perceived Usefulness item or as a Perceived Ease of Use item, and a positive and a negative phrasing entry. Both phrasings are included instead of only the standard positive phrasing, so that one of these phrasings can be selected when presenting the item to the user, avoiding only one type of phrasing causing a bias within the user towards that specific phrasing.

\section{User attributes, objects and methods}

The main user attributes are mainly based on the aforementioned generic data entries, namely the FlashcardInstance or FlashmapInstance objects --- storing the learning progression of certain Flashcards or Edges from the ConceptMap ---, the Test objects --- representing a pre- or posttest and containing the responses to sets of TestItems ---, and the Questionnaire object --- containing the responses to the QuestionnaireItems. Next to these objects, the user also has certain descriptive attributes, and objects related to logged in sessions.

\subsection{Descriptive attributes}

The descriptive attributes are either used for the program to function, or for generating descriptive statistics as controll variables in the results section. They contain the username, the condition, the gender and birthdate of the user, the code he received on his informed consent form, an email address, and a debriefed field.

\paragraph{Username} First and foremost, every user has a unique username which can be used to log in to the application. This name is chosen by the user himself, so he can decide to use his proper name or use an alias to remain anonymous. Even so, within the released data this field is removed in order to safeguard the identities of the user.

\paragraph{Condition} This field determines whether the user is partaking in the "FLASHCARD" group (the control group) or the "FLASHMAP" group (the experimental group). This field is set when the user registers a new account by the formula $len(users) \bmod 2$, which entails that every new user is assigned to the opposite group relative to the user before. This is in order to ensure that the initial users are equally divided over both groups.

\paragraph{Gender} The gender field of the user is included in order to check after the experiment whether the distribution of genders is equal within both experimental conditions. This variable is prompted before the pretest.

\paragraph{Birthdate} The birthdate field of the user is included in order to check after the experiment whether the age distributions of both experimental conditions are the same. This variable is prompted before the pretest.

\paragraph{Code} Before the start of the experiment, every student participating had to sign an informed consent form together with a parent or caretaker in accordance with the Ethics Committee at the University of Twente. To ensure that every user in the system corresponded with students who handed in a double signed form without the user having to enter his own name into the system, every form contained a code which was prompted towards the user before the pretest.

\paragraph{Email} At the end of the questionnaire, the user was prompted whether he wanted to participate in an interview. If this was the case, he could fill in his email address here in order to be contacted at a later date. This variable is also ommitted in the published data.

\paragraph{Desbriefed} This field is a simple boolean value indicating whether the user has been debriefed. It initiated with a False value, and was set to True when the server received a "DEBRIEFING-RESPONSE" message from the client (see figure~\ref{fig:loginactivity} on page~\pageref{fig:loginactivity}).

\subsection{Sessions}

\subsection{Instances}

\begin{figure}
\centering
    \includegraphics[height=.5\textheight]{img/learningserver.png}
\caption{An UML activity diagram showing the scheduling and saving of a list of responses within instances}
\label{fig:learningserver}
\end{figure}

\subsection{Tests}

\subsection{Questionnaire}
 %Not started
    \chapter{Client design and development}
\label{ch:client}

\section{Learning progress}
\label{sec:learningprogress}

\begin{figure}[h!]
\centering
\includegraphics[width=\textwidth]{img/learningclient.png}
\caption{Prompting an instance on the client}
\label{fig:learningclient}
\end{figure}
 %Not started

\part{Research}
    \chapter{Aims and goals for the research}
\label{ch:aimsgoals}
 %Not started
    \chapter{Methods}
\label{ch:methods}

\section{Research design}
\label{sec:researchdesign}

Research questions~\ref{benefit}\ref{effectiveness}, \ref{efficiency}, \ref{perception}\ref{usefulness} and \ref{ease} will be investigated using intervention-based research. Because of the systems being used for self-study by the students, they can be individually assigned to a condition, and this enables the use of a true experimental design. Since this will provide the most valid and reliable results, this research design is implemented in this experiment.

Additionally, research questions~\ref{usefulness} and \ref{ease} will also be investigated using open questions on the questionnaire form and by conducting interviews with a sample of the participants.

Finally, research question~\ref{howused} will be investigating both quantitatively by logging the user behaviour during the experiment and by the open questions and interviews also used for research questions~\ref{usefulness} and \ref{ease}.

The quantitative and qualitative results will be mixed for the purposes of triangulation and expansion as described by \citeA{mixedmethods}. The interviews and logs could provide insight in the degree of which the systems were used the intended way and in why students had certain perceptions on using the systems. Both triangulation and expansion will be on a partial level of mixing, will take place concurrently, and the quantitative data will be dominant, since the qualitative data exists only to triangulate and expand the quantitative data. 

\section{Respondents}
\label{sec:respondents}

100 15 to 17 year old tenth grade Dutch high school students will be approached. They already have to prepare themselves for an exam on the same topic and thereby have incentive to learn. To increase the response rate, the students will be rewarded with a \euro{} 5 voucher for participation. The participants will be assigned to either the flashcard or the flashmap condition at random when they create a user account within the webapplication.

\section{Procedure}
\label{sec:procedure}

\paragraph{Review concept map, flashcards, and item bank} In order to verify whether the content offered within the system is in alignment with the learning goals set by the teachers, one of the teachers is asked to review the content. The feedback received from the teacher is afterwards incorporated by altering the dataset. This can be seen as the focus evaluation of the product \cite{slo}.

\paragraph{Approval ethical committee} Before the actual experiment can take place, the research setup first has to be approved by the ethical committee of the University of Twente. This takes place before the system is introduced to the students.

\paragraph{Presentation} Within the school curriculum there are two instructions planned for the topic of Dutch renaissance literature. At the end of the instruction, the researcher introduces the experiment and the system to the participants. This is meant both to attract students to participate, but also to provide a briefing next to the written briefing. Within the presentation, the benefits are stated (better preparation for the exam, preview for the exam), it is stressed that participation is voluntary and that the data will be collected anonymously, the informed consent form will be introduced, and finally the reward (icecream vourchers) will be announced, together with the conditions for receiving the reward. Information with regards to the seperate experimental conditions will be limited to the researcher explaining that there are two different versions, without elaborating what these versions entail. This is in order to prevent prejudgements from within the user, making it a double-blind experiment. It could however of course happen that the participants learn about each others version during the experiment, and unfortunately this cannot be prevented.

\paragraph{Informed consent form} The informed consent form, included within the \nameref{app:consentform} appendix on page~\pageref{app:consentform}, contains a letter, the written briefing, and a form which has to be signed by both the parent or caretaker and the participant. It also contains a code with which it can be verified that a user within the system did indeed sign this form. The briefing contains a description of the research, the advantages of participation, and the procedure of the experiment.

\paragraph{Division of respondents} Users will be assigned alternately to the control group or the experimental group. This pseudo-random assignment increases the validity of the experimental design without risking one group becoming larger than the other group. This however only holds up for the initial group, because dropout rates between the group could vary resulting in differently sized finished groups.

\paragraph{Descriptives} Before the experiment itself starts, the gender and the birthdate of the participants are prompted. This provides descriptive statistics necessary for the measure of generalisability of the results.

\paragraph{Pretest} Another form prompted towards the user before the start of the experiment is the pretest, measuring how much the user already knows and understands about the subject. This test is elaborated within the next section.

\paragraph{Experiment} The experiment itself consists of the participants having to review instances for 15 minutes over the course of 6 days. The 15 minutes are estimated to be the amount of time necessary to review one section in the instructional material, and the 6 days are chosen as a balance between having covered a large enough portion of the material to measure a significant learning gain without the participant investment being too large resulting in no student wanting to participate. The 6 days of learning ideally take place subsequently, since then students have a higher chance of retrieving repeating instances correctly. However, it is likely that participants could forget about the experiment or be too busy to invest the 15 minutes, and therefore they are allowed one non active day during the experiment. Each session consists of being presented by questions or incomplete concept maps, trying to retrieve the correct answer or missing concepts from memory, and indicating whether the correct answer or missing concept was successfully retrieved.

\paragraph{Posttest} The seventh day the participant logs in to the system he is prompted with the posttest in order to measure the level of knowledge and comprehension after the experiment. This test uses the same itembank as the pretest, and is thereby also elaborated within the next section.

\paragraph{Questionnaire} After filling in the posttest, the participant is also asked to fill in a questionnaire based on the Technology Acceptance Model, which is further elaborated within the next section. The form also contains two open questions, namely to describe what the participant thought was good about the system, and what he thought could be improved. Finally, he can fill in his email address if he is interested in being interviewed afterwards.

\paragraph{Debriefing} Finally, the participant is presented with the debriefing text from figure~\ref{fig:ui_debriefing} on page~\pageref{fig:ui_debriefing}. This states that the user will soon receive the voucher, that he is allowed to keep using the system, and that he can contact the researcher if he has questions or when he wants to see his personal data. The participant is now finished with the main experiment.

\paragraph{Scoring sample items} After all the results have been gathered, a small sample of the responses are reviewed by both one of the teachers and the researcher in order to establish an inter-rater reliability. This will take place after the school test itself, but before the teacher has scored the test administered by the school itself to remain unbiased. Furthermore, the sample will be a random anonymous selection of responses in order to minimise any halo effect. Finally, the samples are filtered on non-empty items which do also not exactly correspond to the response model, since these can be automatically scored. If the inter-rater reliability is too low, the scoring rubics will be altered in order to differentiate better among correct or incorrect responses, and the procedure is repeated. Otherwise, the rest of the responses is scored by the researcher.

\paragraph{Interview} Those who volunteered for the interview will be sent an invitation by email for an open group interview, also elaborated in the next section.

\paragraph{Icecream vouchers} The vouchers and the list containing the names of students having participated within the experiment will be handed over to the students after the scoring of the test administered within the school in order to avoid influencing the teachers during the marking of the tests.

\section{Instrumentation}
\label{sec:instrumentation}

\subsection{Test}

A test (either pre- or posttest) consists out of a knowledge section and a comprehension section, derived from Bloom's Taxonomy \cite{bloom}. The knowledge section aimed at measuring whether the rote memorisation was effective, and the comprehension section served the purpose of measuring whether the explicit relations within the flashmaps scaffolded the comprehension. Only these two levels were chosen, since higher levels are more time consuming to create and to answer by the students. Furthermore, the first step is to measure whether students can already generalise from only rehearsing questions on the knowledge level to questions on the comprehension level. The questions were randomly chosen from itembanks, where the itembank used for the knowledge questions were the flashcards themselves and the itembank for the comprehension questions a seperate set of 10 items. By randomly selecting the questions, the overall knowledge and comprehension are measured instead of specific subsets measured seperately on the pretest and posttest. This eliminates the specific item difficulty variable from the learning gain, increasing its validity. The random variable however also increases the variability of scores, decreasing the significance when comparing the conditions. Both the pretest and the posttest select 5 questions from each bank with non-overlapping items. Finally, a rubics was created in order to quantisise the openly formulated answers, consisting out of possible answer categories for each item.

\subsection{Questionnaire}

The questionnaire consists out of items based on the Technology Acceptance Model (TAM) \cite{tam}, containing items measuring perceived usefulness and perceived ease of use. \citeA{devellis} describes that the validity of a questionnaire can be increased by formulating the items mixed positively and negatively, and by repeating the items. Therefore, the items from TAM are translated to Dutch in both a positive and a negative phrasing. For each participant, 2 sets of items are created per TAM category: for each item, a phrasing is selected at random after which the set is shuffled; another set of shuffled perceived usefulness items is created with each item having the opposite phrasing of its counterpart in set 1. This results in 4 sets of items encompassing both phrasings of all items. For each item, the participant can indicate whether he completely disagrees (-2), he disagrees (-1), he neither agrees nor disagrees (0), he agrees (1), or he completely agrees (2).

\subsection{Data collection during experiment}

During the experiment itself more data is created in order to provide statistics about the usage of the system. These statistics are mainly contained within the Response objects in the database.

\paragraph{Correctness retrievals} For each response within an instance, information is stored about whether the instance was retrieved correctly by the participant. This is not only useful for the rescheduling of the instance, but also in order to gain more insights in how often a participant was able to retrieve instances correctly, and to verify whether the success rate is indeed around 90\% as stated by \citeA{microlearning}.

\paragraph{Retrieval time} Furthermore, the start time (when an instance is sent from the server to the participant) and the end time (when the response was received by the server) are stored for each response in order to measure how much time was spent on each instance, and on the system in general.

\paragraph{Active days} Finally, a list is stored for each participant containing the dates that the participant was active. An active day is here defined as a day where the participant is active for 15 minutes or where the participant reviewed all of the available instances within the read sections. This data is useful for the system to know when to prompt the posttest, but also serves as a statistic on participation rates.

\subsection{Interview}

The interview is open instead of structured or semi-structured, since the interview should provide the interviewees the opportunity to tell about their own experiences from using the system, and since the data covering specific questions is already gathered within the usage data, the questionnaire and the two open questions. The researcher will use topics based on TAM in order to verify during the interview whether the main area is covered, with an open question at the end asking for general possible improvements. Furthermore, the interview will take place as one group interview, since the school organisation only has limited availability options for providing seperate rooms fit for interviews and the group interview being the most efficient option. At the start of the interview, the interviewees will be disclosed about the aims of the research in order to establish the different versions. Results will however not be disclosed until after the experiment in order to leave the interviewees unbiased in their opinion about the perceived usefulness. Finally, with the interviewees permission, the interview is recorded for later analysis. If not, the researcher takes notes during the interview.

\section{Analysis}
\label{sec:analysis}

\paragraph{Inter-rater reliability}

\paragraph{Classical test theory}

\paragraph{Item response theory}

\paragraph{Learning gain}

%Absolute and relative

\paragraph{Instance statistics}

\paragraph{Questionnaire statistics}

\paragraph{Comparisons}

\paragraph{Hypotheses}

\paragraph{Interviews}
 %Not started
    \chapter{Results}
\label{ch:results}

\section{User participation}

The \nameref{app:dropouts} appendix on page~\pageref{app:dropouts} shows statistics on how many days the different used the system. In table~\ref{tab:dropouts_incl} it is shown that in total 63 students made an account within the system. From these students 44 used the system on at least one day for longer than 15 minutes, of which the average student participated on 4.68 days. The usage of the system is also depicted in figures~\ref{fig:dropouts_fc}, \ref{fig:dropouts_fm} and~\ref{fig:dropouts_gen}. Finally, figures~\ref{fig:activedays_fc}, \ref{fig:activedays_fm} and \ref{fig:activedays_gen} display on which days users have been actively using the system, where day 0 is the day the system was introduced within the presentation and day 21 the final day before the exam. Interesting to note here is that there are two subsequent strong increases in finished users around 6 days after the system was introduced, but that there is also a very strong increase the day before the students' exam. This is also reflected within a highly increased activity within the first week, and on the day before the exam. There is even some noticable activity after the exam took place, possibly of students already having invested some time into the system before the exam, but still finishing up in order to be rewarded with the icecream coupon.

This resulted in a total number of 25 finished users, of which 13 users within the flashcard condition and 12 users within the flashmap condition. Strange enough, both in the flashcard and the flashmap condition there is one user which did use the software for 6 days, but then did not partake in the posttest. They could be confused by the posttest (since it looks exactly the same as the pretest), and then decided to not fill it in. Finally, in figure~\ref{fig:dropouts_gen} it can be seen that there are 4 students which used the system for only 5 days. This is also strange, since that way they just missed out on the reward. A sample of 23 divided over two conditions is considerably small, and therefore any results stemming from this experiments are only indicatory and should be further investigated before making any generalisations.

Since the only students usable for the rest of the result section are those finishing the posttest, the other students will be omitted from consideration.

\section{Participant descriptives}

The participant descriptives are included in the \nameref{app:descriptives} appendix on page~\ref{app:descriptives}, containing distributions of student gender and age.

\paragraph{Gender} As can be seen in figure~\ref{tab:gender}, 15 out of the 23 total participants are male and 8 are female, where within the flashcards condition there is a 7 to 5 ratio and within the flashmap a 8 to 3 ratio. This is probably just coincidental due to the small sample size.

\paragraph{Age} All students have an age within the range of 15 to 17 with an average age of 15.75 and a modus of 16, which is to be expected from VWO4 students. There is also no considerable age difference among the conditions, indicated in table~\ref{tab:age_comp} and figure~\ref{fig:age}.

\section{Learning gain}

\section{Efficiency}

\section{Usefulness}

\subsection{Usefulness items}

\subsection{Received comments}

\section{Ease of use}

\subsection{Ease of use items}

\subsection{Received comments}

\section{Usage}

\subsection{Number of responses}

\subsection{Exponents}

\subsection{Correct retrievals}

\subsection{Interview results}
 %Not started
    \chapter{Discussion}
\label{ch:discussion}

\section{Conclusions}

\paragraph{Learning gain} The only significant difference in learning gain was that the flashmap respondents scored higher on the knowledge questions than the flashcard respondents using irt analysis ($p < 0.001$). The test reliability of this analysis was relatively low ($\alpha=0.67$), however was confirmed by the non significant outcomes of the ctt analysis ($p=0.464$) with higher test reliability ($\alpha=7.21$). One explanatory hypothesis might be that there are more explicit retrieval routes in the students' memory for the learned concepts, resulting in a higher correct retrieval rate.

\paragraph{Learning progress} A non-significant lower percentage of material covered was found by the flashmap users than by the flashcard users ($p=0.188$). Looking at figure~\ref{fig:instance_abil} on page~\pageref{fig:instance_abil}, this is likely due to the 2 outliers on the lower spectrum of the flashmap users, since the other users have comparable material coverage. This could also explain the fewer responses per learned item by flashmap users than by flashcard users, although this result is significant ($p=0.024$) and can thereby not be directly dismissed as random outcome. Finally, the flashmap users spent a significant larger amount of time on the system than the flashcard users ($p<0.001$). This difference could be due to the more cumbersome navigation indicated by the flashmap users. However, a significantly smaller amount of time was spent by flashmap users per item ($p<0.001$). One explanatory factor is that many flashcards asked for the retrieval of multiple concepts per items, explaining a longer retrieval time in comparison to the single concept per relation retrievals for the flashmap condition. It could also be that flashmap users had a quicker retrieval rate because of the more explicit retrieval routes within memory.

\paragraph{Perceived usefulness} No significant difference has been found in perceived usefulness between flashmap and flashcard users ($p=0.245$). Within the interviews, students commented that they liked the structure provided by the questions or flashmaps, the distribution of learning over time by the scheduling algorithm, and the repetition of quesitons. However, they also indicated that certain questions or flashmaps are repeated too often. Finally, the correct retrieval rate within the use of the system was around 0.87, which is indicated by \citeA{microlearning} as a desirable balance between overlearning and spacing. Herein, the flashmap users had a significant higher retrieval rate ($p<0.001$), however this can be explained by certain flashmap users not knowing how to indicate whether their retrieval was correct or incorrect, and thereby indicating all items as correct (which is the default value).

\paragraph{Perceived ease of use} Flashmap and flashcard users did also not significantly differ in perceived ease of use ($p=0.482$). Students did comment that the instructions provided within the software could be improved, and that the system was rigid in what they had to rehearse. 

\section{Limitations}

The largest limitation within this study is the small sample size of only 23 students divided over two groups, which is too small to come to any definite conclusions. The other limitation is that the students are a homogeneous group, studying only one subject and enrolled within the same school, making the results not fit for generalisation to any other fields before further replication studies within other groups or with different subject material are conducted.

\section{Future work}

The outcomes for the comprehension questions were both insignificant and contradictory, with a higher learning gain for the flashcard condition using ctt analysis ($p=0.245$, $\alpha=0.714$) and a lower learning gain using irt analysis ($p=0.688$, $\alpha=0.606$). Since the former result has a lower p-value and higher validity, it is likely that with a larger sample size the flashcard user's learning gain could become significantly larger than that of the flashmap users. Possibly, this could be the result of the flashcard users having to actively think of the connections between the concepts themselves, as already hypothesised by one of the interviewees. The items being phrased could also have an effect. In order to confirm this hypothesis, replication studies using larger sample sizes should be conducted.

Furthermore, the research could be repeated with the participant creating their own concept maps or flashcards, providing insights in the learnign gain when the system is used more generatively.

Finally, the suggestions provided within the interviews could be implemented to improve the user experience:
%
\begin{itemize}
    \item the scheduling system could be made more adaptive, for example by including the additions suggested by \citeA{microlearning}
    \item the instructions could be improved, or a tutorial could be included for learning how to use the system
    \item the flashmaps could show a smaller portion of the concept map, making the navigation easier (especially on smaller devices such as smartphones)
    \item options could be included for the users in order to select the sections they want to learn
\end{itemize}
%
The system could then be formatively evaluated, for example observing students using the system, in order to tune the system to the needs of the average user.
 %Not started

\part{Recommendations}
    \input{./recommendations/features.tex} %Not started
    \input{./recommendations/research.tex} %Not started
    \input{./recommendations/implementation.tex} %Not started

\part{}

\chapter{Epilogue}

\input{./epilogue/epilogue.tex} %Not started

\bibliography{references}

\begin{appendices}
    \chapter{Test literature history 16th and 17th century}

\label{app:exampletest}

This is not the test used as pre- and posttest within the research, but a test provided to the previous generation of students provided by the teacher.

\begin{enumerate}
    \item Provide a Dutch word for the term `renaissance'. Furthermore, explain the central idea of the renaissansistic body of thoughts.
    \item Indicate whether the following statements are `true' or `false':
        \begin{enumerate}
            \item The renaissance originated in the Northern and Central Italian republican citystates.
            \item The literature from the renaissance is only a revival of classical genres.
            \item The eventual goal of renaissance writers is imitatio.
            \item An amount of great playwriters from the renaissance is literarily schooled within a chamber of rhetoric [\emph{rederijkerskamer}].
        \end{enumerate}
    \item What is the essence of humanism?
    \item Read the citation below: [\ldots]
        \begin{enumerate}
            \item What is the title of the book from which this citation originates and who wrote this book?
            \item What was the goal of writing this book and why is the book still attractive to read?
        \end{enumerate}
    \item \begin{enumerate} \item What was the reason for writing the Dutch Authorised Version of the bible [\emph{Statenbijbel}]?
            \item Some of the expressions we are still using come from the Dutch Authorised Version of the bible. Why was this bible, generally stated, so important for language in that time?
        \end{enumerate}
    \item Except of imitatio writers used two other methods. Enlist the three methods in the right order and provide a description for each.
    \item \begin{enumerate} \item Which poet is the great example for those who write lovepoetry in this era? \label{itm:lovepoet}
            \item Explain what platonic love is and what this has to do with lovepoetry from question~\ref{itm:lovepoet} and with the adjoining picture [see figure~\ref{fig:laura}].\label{itm:platoniclove}
        \end{enumerate}
    \item Which combination of terms best displays the central ideas from renaissance literature?
        \begin{enumerate}
            \item Learning and pleasure
            \item Antiquity and church
            \item Love and antiquity
            \item Church and pleasure
        \end{enumerate}
    \item Provide for each of the genres of theater (tragedy, comedy, and farce [\emph{klucht}]) a name of a matching writer and the title of a matching play.
    \item Provide two differences between a tragedy and a farce.
    \item Quickly after Willem-Alexander became king of the Netherlands, he visited different provinces together with M\'{a}xima. The province of Drenthe gave a small book on the occasion of this visit, containing among else the following poem: [\ldots]
        \begin{enumerate}
            \item This poem is a sonnet. Enlist the characteristics of a sonnet regarding the form and content.
            \item Explain how the content-related characteristic is included in the poem above.
            \item Who was our most important sonnet writer in the 17th century?
        \end{enumerate}
    \item \begin{enumerate} \item What is the goal of \emph{emblematiek}? Include the term `analogical thinking' in your answer.
            \item Of which three parts does an emblem exist? Use the original terms/names.
        \end{enumerate}
    \item View and read the emblem below [see figure~\ref{fig:emblem}] and conduct the following assignments:\label{itm:emblem}
        \begin{enumerate}
            \item Explain in your own words which analogy is made in the emblem and which lesson the writer wants to teach the reader.
            \item With which word from the original emblem does the analogy start?
        \end{enumerate}
    \item In the children series `Dappere Dodo', 75 episodes were broadcasted on the Dutch TV between 1955 and 1964. The programm revolved around \emph{Dappere Dodo}, who together with his friends Kees, Uncle Harrie, the captain, Grandfather Buiswater and Mrs Vulpen sailed around the world and experienced all kinds of adventure. `Dodo' is in this series an appropriate name for the main person. Provide a good explanation for this.
    \item The shipsjournal of Bontekoe went up in flames during a shipboard fire. Why did he write the journal again after the sea journey?
    \item What does the Meertens Institute? It occupies itself with:
        \begin{enumerate}
            \item the study and documentation of Dutch language varation and folk culture
            \item research into and documentation of European language and culture
            \item collecting and documenting songs, specific from the period of the 16th and 17th century
            \item research into dialects and socilects in European context.
        \end{enumerate}
\end{enumerate}

\begin{figure}
    \centering
    \includegraphics[width=0.3\textwidth]{img/laura.jpg}
    \caption{The figure accompanying question~\protect\ref{itm:platoniclove}}
    \label{fig:laura}
\end{figure}

\begin{figure}
    \centering
    \includegraphics[width=0.3\textwidth]{img/emblem.jpg}
    \caption{The figure accompanying question~\protect\ref{itm:emblem}. This figure was accompanied by the following text: ``Soo lang de Roe wanckt. Veel mensche zijn deughdelijck, soo langh zy onder het kruys en verdruckinghe leven: maer als de Roede van den eers is, soo worden zy luy in den dienste Goods. Ghelijuck enen Drijf-tol, die niet meer gheslaghen of ghegispt en wort, die valt haest in onmacht ende blijft ligghen. Uit: Roemer Visscher, Sinnepoppen.'' In the original test a modern Dutch translation was also provided.}
    \label{fig:emblem}
\end{figure}

%    \chapter{Concept map}

\label{ch:concept_map}

\lstinputlisting{../software/database/concept_map.json}

%    \chapter{Flashcards}

\label{ch:flashcards}

\lstinputlisting{../software/database/flashcards.json}

%    \chapter{Questions measuring comprehension levels}

\label{ch:test}

\lstinputlisting{../software/database/itembank.json}

    \clearpage
    \section{Use case diagram for registering and logging in}
\label{app:loginusecase}
\begin{figure}[h!]
\centering
\includegraphics[width=\textwidth]{img/loginusecase.png}
\end{figure}

    \clearpage
    \section{Use case diagram for main purposes}
\label{app:mainusecase}
\begin{figure}[h!]
\centering
\includegraphics[width=\textwidth]{img/mainusecase.png}
\end{figure}

    \clearpage
%    \section{Class diagram for the datamodel}
\label{app:classdiagram}
\begin{figure}[h!]
\centering
\includegraphics[height=\textheight-10ex]{img/classdiagram.png}
\end{figure}

%    \clearpage
    \chapter{Activity diagram displaying the login process}
\label{app:loginactivity}
\begin{figure}[h!]
\centering
\includegraphics[height=\textheight]{img/loginactivity.png}
\end{figure}

    \clearpage
    \section{Activity diagram learning functionality}
\label{app:learningactivity}
\begin{figure}[h!]
\centering
\includegraphics[width=\textwidth]{img/learningactivitygen.png}
\caption{General schema}
\end{figure}
\begin{figure}[h!]
\centering
\includegraphics[width=\textwidth]{img/learningserver.png}
\caption{Instance scheduling on the server}
\end{figure}
\begin{figure}[h!]
\centering
\includegraphics[width=\textwidth]{img/learningclient.png}
\caption{Prompting an instance on the client}
\end{figure}

    \clearpage
    \includepdf[pages={1-}]{../software/server/doc/build/latex/Flashmapserver.pdf}
\end{appendices}
    \input{./appendix/learning_gain.tex}
    \chapter{Questionnaire statistics}
\label{app:questionnaire}

\FloatBarrier
\section{Descriptives of Perceived Usefulness questions}

\begin{longtable}[c]{@{}lrrrrrrrrrr@{}}
\caption{Flashcard condition}
\endfirsthead
\endhead
\toprule\addlinespace
& N & min & max & mean & variance & skew & kurt & norm-t &
norm-p & $\alpha$
\\\addlinespace
\midrule
\textbf{ctt} & 12 & -4 & 14 & 6.50 & 27.36 & -0.57 & -0.52 & 1.144 &
0.5643 & 0.6432
\\\addlinespace
\textbf{irt} & 12 & -5 & 1 & -0.16 & 4.40 & -1.89 & 3.15 & 17.284 &
0.0002 & 0.6263
\\\addlinespace
\bottomrule
    \label{tab:usefulness_fc}
\end{longtable}

\begin{figure}
    \centering
    \includegraphics[width=.7\textwidth]{img/usefulness_fc_diff.png}
    \caption{A histogram depicting the scores per Perceived Usefulness item by flashcard users}
    \label{fig:usefulness_fc_diff}
\end{figure}
\begin{figure}
    \centering
    \includegraphics[width=.7\textwidth]{img/usefulness_fc_abil.png}
    \caption{A histogram depicting the scores on the Perceived Usefulness items per flashcard user}
    \label{fig:usefulness_fc_abil}
\end{figure}

\begin{longtable}[c]{@{}lrrrrrrrrrr@{}}
\caption{Flashmap condition}
\endfirsthead
\toprule\addlinespace
& N & min & max & mean & variance & skew & kurt & norm-t &
norm-p & $\alpha$
\\\addlinespace
\midrule
\textbf{ctt} & 11 & 0 & 13 & 8.82 & 15.16 & -1.05 & 0.32 & 4.698 &
0.0955 & 0.6777
\\\addlinespace
\textbf{irt} & 11 & -3 & 1 & -0.15 & 1.89 & -1.41 & 1.44 & 9.670 &
0.0079 & 0.5298
\\\addlinespace
\bottomrule
    \label{tab:usefulness_fm}
\end{longtable}

\begin{figure}
    \centering
    \includegraphics[width=.7\textwidth]{img/usefulness_fm_diff.png}
    \caption{A histogram depicting the scores per Perceived Usefulness item by flashmap users}
    \label{fig:usefulness_fm_diff}
\end{figure}
\begin{figure}
    \centering
    \includegraphics[width=.7\textwidth]{img/usefulness_fm_abil.png}
    \caption{A histogram depicting the scores on the Perceived Usefulness items per flashmap user}
    \label{fig:usefulness_fm_abil}
\end{figure}

\begin{longtable}[c]{@{}lrrrrrrrrrr@{}}
\caption{Combined conditions}
\endfirsthead
\toprule\addlinespace
& N & min & max & mean & variance & skew & kurt & norm-t &
norm-p & $\alpha$
\\\addlinespace
\midrule
\textbf{ctt} & 23 & -4 & 14 & 7.61 & 21.98 & -0.86 & -0.02 & 3.864 &
0.1448 & 0.6509
\\\addlinespace
\textbf{irt} & 23 & -3 & 2 & 0.49 & 1.82 & -0.73 & 0.55 & 4.058 & 0.1315
& 0.4619
\\\addlinespace
\bottomrule
    \label{tab:usefulness_gen}
\end{longtable}

\begin{figure}
    \centering
    \includegraphics[width=.7\textwidth]{img/usefulness_gen_diff.png}
    \caption{A histogram depicting the scores per Perceived Usefulness item}
    \label{fig:usefulness_gen_diff}
\end{figure}
\begin{figure}
    \centering
    \includegraphics[width=.7\textwidth]{img/usefulness_gen_abil.png}
    \caption{A histogram depicting the scores on the Perceived Usefulness items per user}
    \label{fig:usefulness_gen_abil}
\end{figure}

\FloatBarrier
\section{Descriptives of Perceived Ease of Use questions}\label{ease-of-use}

\begin{longtable}[c]{@{}lrrrrrrrrrr@{}}
\caption{Flashcard condition}
\endfirsthead
\toprule\addlinespace
& N & min & max & mean & variance & skew & kurt & norm-t &
norm-p & $\alpha$
\\\addlinespace
\midrule
\textbf{ctt} & 12 & -4 & 17 & 6.58 & 38.08 & -0.26 & -0.62 & 0.232 &
0.8904 & 0.8794
\\\addlinespace
\textbf{irt} & 12 & 0 & 4 & 0.91 & 1.87 & 1.08 & 0.52 & 5.358 & 0.0686 &
0.2295
\\\addlinespace
\bottomrule
    \label{tab:easeofuse_fc}
\end{longtable}

\begin{figure}
    \centering
    \includegraphics[width=.7\textwidth]{img/easeofuse_fc_diff.png}
    \caption{A histogram depicting the scores per Perceived Ease of use item by flashcard users}
    \label{fig:easeofuse_fc_diff}
\end{figure}
\begin{figure}
    \centering
    \includegraphics[width=.7\textwidth]{img/easeofuse_fc_abil.png}
    \caption{A histogram depicting the scores on the Perceived Ease of use items per flashcard user}
    \label{fig:easeofuse_fc_abil}
\end{figure}

\begin{longtable}[c]{@{}lrrrrrrrrrr@{}}
\caption{Flashmap condition}
\endfirsthead
\toprule\addlinespace
& N & min & max & mean & variance & skew & kurt & norm-t &
norm-p & $\alpha$
\\\addlinespace
\midrule
\textbf{ctt} & 11 & 0 & 19 & 8.27 & 26.22 & 0.50 & 0.12 & 1.725 & 0.4220
& 0.7689
\\\addlinespace
\textbf{irt} & 11 & -2 & 2 & 0.22 & 1.87 & -0.20 & 1.01 & 3.041 & 0.2186
& 0.2538
\\\addlinespace
\bottomrule
    \label{tab:easeofuse_fm}
\end{longtable}

\begin{figure}
    \centering
    \includegraphics[width=.7\textwidth]{img/easeofuse_fm_diff.png}
    \caption{A histogram depicting the scores per Perceived Ease of use item by flashmap users}
    \label{fig:easeofuse_fm_diff}
\end{figure}
\begin{figure}
    \centering
    \includegraphics[width=.7\textwidth]{img/easeofuse_fm_abil.png}
    \caption{A histogram depicting the scores on the Perceived Ease of use items per flashmap user}
    \label{fig:easeofuse_fm_abil}
\end{figure}

\begin{longtable}[c]{@{}lrrrrrrrrrr@{}}
\caption{Combined conditions}
\endfirsthead
\toprule\addlinespace
& N & min & max & mean & variance & skew & kurt & norm-t &
norm-p & $\alpha$
\\\addlinespace
\midrule
\textbf{ctt} & 23 & -4 & 19 & 7.39 & 31.70 & -0.08 & -0.10 & 0.239 &
0.8876 & 0.8285
\\\addlinespace
\bottomrule
    \label{tab:easeofuse_gen}
\end{longtable}

\begin{figure}
    \centering
    \includegraphics[width=.7\textwidth]{img/easeofuse_gen_diff.png}
    \caption{A histogram depicting the scores per Perceived Ease of use item}
    \label{fig:easeofuse_gen_diff}
\end{figure}
\begin{figure}
    \centering
    \includegraphics[width=.7\textwidth]{img/easeofuse_gen_abil.png}
    \caption{A histogram depicting the scores on the Perceived Ease of use items per user}
    \label{fig:easeofuse_gen_abil}
\end{figure}

\FloatBarrier
\section{Comparisons of the Perceived Usefulness questions}


\begin{longtable}[c]{@{}lrrrr@{}}
\toprule\addlinespace
& \textbf{MW k} & \textbf{MW p} &
\textbf{t-test k} & \textbf{t-test p}
\\\addlinespace
\midrule
\textbf{ctt} & -1.196 & 0.2449 & -1.212 & 0.2395
\\\addlinespace
\textbf{irt} & -0.014 & 0.9891 & -0.014 & 0.9889
\\\addlinespace
\bottomrule
    \label{tab:usefulness_comp}
\end{longtable}

\begin{figure}
    \centering
    \includegraphics[width=.7\textwidth]{img/usefulness_diff.png}
    \caption{A comparison of figure~\protect\ref{fig:usefulness_fc_diff} and~\protect\ref{fig:usefulness_fm_diff}}
    \label{fig:usefulness_diff}
\end{figure}
\begin{figure}
    \centering
    \includegraphics[width=.7\textwidth]{img/usefulness_abil.png}
    \caption{A comparison of figure~\protect\ref{fig:usefulness_fc_abil} and~\protect\ref{fig:usefulness_fm_abil}}
    \label{fig:usefulness_abil}
\end{figure}

\FloatBarrier
\section{Comparisons of the Perceived Usefulness questions}

\begin{longtable}[c]{@{}lrrrr@{}}
\toprule\addlinespace
& \textbf{MW k} & \textbf{MW p} &
\textbf{t-test k} & \textbf{t-test p}
\\\addlinespace
\midrule
\textbf{ctt} & -0.711 & 0.4851 & -0.717 & 0.4816
\\\addlinespace
\textbf{irt} & 1.206 & 0.2411 & 1.206 & 0.2412
\\\addlinespace
\bottomrule
    \label{tab:easeofuse_comp}
\end{longtable}

\begin{figure}
    \centering
    \includegraphics[width=.7\textwidth]{img/easeofuse_diff.png}
    \caption{A comparison of figure~\protect\ref{fig:easeofuse_fc_diff} and~\protect\ref{fig:easeofuse_fm_diff}}
    \label{fig:easeofuse_diff}
\end{figure}
\begin{figure}
    \centering
    \includegraphics[width=.7\textwidth]{img/easeofuse_abil.png}
    \caption{A comparison of figure~\protect\ref{fig:easeofuse_fc_abil} and~\protect\ref{fig:easeofuse_fm_abil}}
    \label{fig:easeofuse_abil}
\end{figure}

    \chapter{Instance statistics}
\label{app:instance_stats}

\FloatBarrier
\section{Descriptives on the number of reviewed instances}

\begin{longtable}[c]{@{}lrrrrrrrrrr@{}}
    \caption{Flashcard condition}
    \endfirsthead
    \endhead
\toprule\addlinespace
& sample & min & max & mean & variance & skew & kurtosis & normal-t &
normal-p & $\alpha$
\\\addlinespace
\midrule
\textbf{abs} & 12 & 46 & 93 & 72.83 & 344.33 & -0.34 & -1.43 & 3.332 &
0.1890 & 0.9790
\\\addlinespace
\textbf{rel} & 12 & 0.49 & 1 & 0.78 & 0.04 & -0.34 & -1.43 & 3.332 & 0.1890
& 0.9790
\\\addlinespace
\bottomrule
    \label{tab:instance_fc}
\end{longtable}

\begin{figure}
    \centering
    \includegraphics[width=.7\textwidth]{img/instance_fc_diff.png}
    \caption{A histogram depicting the number of reviewed instances per item given by flashcard users}
    \label{fig:instance_fc_diff}
\end{figure}
\begin{figure}
    \centering
    \includegraphics[width=.7\textwidth]{img/instance_fc_abil.png}
    \caption{A histogram depicting the number of reviewed instances per flashcard user}
    \label{fig:instance_fc_abil}
\end{figure}

\begin{longtable}[c]{@{}lrrrrrrrrrr@{}}
    \caption{Flashmap condition}
    \endfirsthead
    \endhead
\toprule\addlinespace
& sample & min & max & mean & variance & skew & kurtosis & normal-t &
normal-p & $\alpha$
\\\addlinespace
\midrule
\textbf{abs} & 11 & 54 & 199 & 131.45 & 2363.47 & 0.07 & -1.18 & 0.961 &
0.6186 & 0.9924
\\\addlinespace
\textbf{rel} & 11 & 0.27 & 1 & 0.66 & 0.06 & 0.07 & -1.18 & 0.961 & 0.6186
& 0.9924
\\\addlinespace
\bottomrule
    \label{tab:instance_fm}
\end{longtable}

\begin{figure}
    \centering
    \includegraphics[width=.7\textwidth]{img/instance_fm_diff.png}
    \caption{A histogram depicting the number of reviewed instances per item given by flashmap users}
    \label{fig:instance_fm_diff}
\end{figure}
\begin{figure}
    \centering
    \includegraphics[width=.7\textwidth]{img/instance_fm_abil.png}
    \caption{A histogram depicting the number of reviewed instances per flashmap user}
    \label{fig:instance_fm_abil}
\end{figure}

\begin{longtable}[c]{@{}lrrrrrrrrrr@{}}
    \caption{Combined conditions}
    \endfirsthead
    \endhead
\toprule\addlinespace
& sample & min & max & mean & variance & skew & kurtosis & normal-t &
normal-p & $\alpha$
\\\addlinespace
\midrule
\textbf{rel} & 23 & 0.27 & 1 & 0.70 & 0.04 & -0.00 & -0.86 & 0.817 & 0.6647
& 0.9896
\\\addlinespace
\bottomrule
    \label{tab:instance_gen}
\end{longtable}

\section{Comparisons of reviewed instances}

\begin{longtable}[c]{@{}lrrrr@{}}
\toprule\addlinespace
& \textbf{Mann-Whitney-U k} & \textbf{Mann-Whitney-U p} &
\textbf{Welch's t-test k} & \textbf{Welch's t-test p}
\\\addlinespace
\midrule\endhead
\textbf{abs} & -3.886 & 0.0009 & -3.756 & 0.0025
\\\addlinespace
\textbf{rel} & 1.362 & 0.1875 & 1.350 & 0.1924
\\\addlinespace
\bottomrule
\end{longtable}

\begin{figure}
    \centering
    \includegraphics[width=.7\textwidth]{img/instance_diff.png}
    \caption{A comparison of figure~\protect\ref{fig:instance_fc_diff} and~\protect\ref{fig:instance_fm_diff}}
    \label{fig:instance_diff}
\end{figure}
\begin{figure}
    \centering
    \includegraphics[width=.7\textwidth]{img/instance_abil.png}
    \caption{A comparison of figure~\protect\ref{fig:instance_fc_abil} and~\protect\ref{fig:instance_fm_abil}}
    \label{fig:instance_abil}
\end{figure}

\FloatBarrier
\section{Descriptives of the number of responses}

\begin{longtable}[c]{@{}lrrrrrrrrrr@{}}
\caption{Flashcard condition}
\endfirsthead
\endhead
\toprule\addlinespace
& N & min & max & mean & variance & skew & kurt & norm-t &
norm-p & $\alpha$
\\\addlinespace
\midrule
\textbf{abs} & 12 & 298 & 1268 & 561.33 & 70295.70 & 1.69 & 2.26 &
13.701 & 0.0011 & 0.9378
\\\addlinespace
\textbf{rel} & 12 & 3 & 13 & 6.04 & 8.13 & 1.69 & 2.26 & 13.701 & 0.0011
& 0.9378
\\\addlinespace
\textbf{mean} & 12 & 5 & 14 & 7.61 & 5.84 & 2.33 & 4.62 & 23.813 &
0.0000 & 0.9378
\\\addlinespace
\bottomrule
    \label{tab:responses_fc}
\end{longtable}

\begin{figure}
    \centering
    \includegraphics[width=.7\textwidth]{img/responses_fc_diff.png}
    \caption{A histogram depicting the number of responses per item given by flashcard users}
    \label{fig:responses_fc_diff}
\end{figure}
\begin{figure}
    \centering
    \includegraphics[width=.7\textwidth]{img/responses_fc_abil.png}
    \caption{A histogram depicting the number of responses per flashcard user}
    \label{fig:responses_fc_abil}
\end{figure}

\begin{longtable}[c]{@{}lrrrrrrrrrr@{}}
\caption{Flashmap condition}
\endfirsthead
\toprule\addlinespace
& N & min & max & mean & variance & skew & kurt & norm-t &
norm-p & $\alpha$
\\\addlinespace
\midrule
\textbf{abs} & 11 & 344 & 1555 & 729.73 & 126333.42 & 1.13 & 0.55 &
5.693 & 0.0581 & 0.9832
\\\addlinespace
\textbf{rel} & 11 & 1 & 7 & 3.67 & 3.19 & 1.13 & 0.55 & 5.693 & 0.0581 &
0.9832
\\\addlinespace
\textbf{mean} & 11 & 3 & 7 & 5.61 & 1.75 & 0.03 & -0.80 & 0.063 & 0.9690
& 0.9832
\\\addlinespace
\bottomrule
    \label{tab:responses_fm}
\end{longtable}

\begin{figure}
    \centering
\includegraphics[width=.7\textwidth]{img/responses_fm_diff.png}
    \caption{A histogram depicting the number of responses per item given by flashmap users}
    \label{fig:responses_fm_diff}
\end{figure}
\begin{figure}
    \centering
    \includegraphics[width=.7\textwidth]{img/responses_fm_abil.png}
    \caption{A histogram depicting the number of responses per flashmap user}
    \label{fig:responses_fm_abil}
\end{figure}

\begin{longtable}[c]{@{}lrrrrrrrrrr@{}}
\caption{Combined conditions}
\endfirsthead
\toprule\addlinespace
& N & min & max & mean & variance & skew & kurt & norm-t &
norm-p & $\alpha$
\\\addlinespace
\midrule
\textbf{abs} & 23 & 298 & 1555 & 641.87 & 99969.48 & 1.41 & 1.42 &
11.547 & 0.0031 & 0.9579
\\\addlinespace
\textbf{mean} & 23 & 3 & 14 & 6.65 & 4.72 & 2.17 & 6.57 & 28.261 &
0.0000 & 0.9572
\\\addlinespace
\bottomrule
    \label{tab:responses_gen}
\end{longtable}

\begin{figure}
    \centering
    \includegraphics[width=.7\textwidth]{img/responses_gen_diff.png}
    \caption{A histogram depicting the number of responses per item}
    \label{fig:responses_gen_diff}
\end{figure}
\begin{figure}
    \centering
    \includegraphics[width=.7\textwidth]{img/responses_gen_abil.png}
    \caption{A histogram depicting the number of responses per user}
    \label{fig:responses_gen_abil}
\end{figure}

\section{Comparisons of the number of responses}

\begin{longtable}[c]{@{}lrrrr@{}}
\toprule\addlinespace
& \textbf{MW k} & \textbf{MW p} &
\textbf{t-test k} & \textbf{t-test p}
\\\addlinespace
\midrule
\textbf{abs} & -1.295 & 0.2092 & -1.279 & 0.2169
\\\addlinespace
\textbf{rel} & 2.361 & 0.0280 & 2.409 & 0.0265
\\\addlinespace
\textbf{mean} & 2.429 & 0.0242 & 2.489 & 0.0233
\\\addlinespace
\bottomrule
    \label{tab:responses_comp}
\end{longtable}

\begin{figure}
    \centering
    \includegraphics[width=.7\textwidth]{img/responses_diff.png}
    \caption{A comparison of figure~\protect\ref{fig:responses_fc_diff} and~\protect\ref{fig:responses_fm_diff}}
    \label{fig:responses_diff}
\end{figure}
\begin{figure}
    \centering
    \includegraphics[width=.7\textwidth]{img/responses_abil.png}
    \caption{A comparison of figure~\protect\ref{fig:responses_fc_abil} and~\protect\ref{fig:responses_fm_abil}}
    \label{fig:responses_abil}
\end{figure}

\FloatBarrier
\section{Descriptives of the exponents of instances}

\begin{longtable}[c]{@{}lrrrrrrrrrr@{}}
    \caption{Flashcard condition}
    \endfirsthead
\toprule\addlinespace
& sample & min & max & mean & variance & skew & kurtosis & normal-t &
normal-p & $\alpha$
\\\addlinespace
\midrule
\textbf{abs} & 12 & 218 & 966 & 495.42 & 41792.81 & 0.86 & 0.42 & 3.894
& 0.1427 & 0.8933
\\\addlinespace
\textbf{rel} & 12 & 2 & 10 & 5.33 & 4.83 & 0.86 & 0.42 & 3.894 & 0.1427
& 0.8933
\\\addlinespace
\textbf{mean} & 12 & 4 & 11 & 6.67 & 2.66 & 1.83 & 3.42 & 17.448 &
0.0002 & 0.8933
\\\addlinespace
\bottomrule
    \label{tab:exponent_fc}
\end{longtable}

\begin{figure}
    \centering
    \includegraphics[width=.7\textwidth]{img/exponent_fc_diff.png}
    \caption{A histogram depicting the exponents per item of the flashcard users}
    \label{fig:exponent_fc_diff}
\end{figure}
\begin{figure}
    \centering
    \includegraphics[width=.7\textwidth]{img/exponent_fc_abil.png}
    \caption{A histogram depicting the exponents per flashcard user}
    \label{fig:exponent_fc_abil}
\end{figure}

\begin{longtable}[c]{@{}lrrrrrrrrrr@{}}
    \caption{Flashmap condition}
    \endfirsthead
\toprule\addlinespace
& sample & min & max & mean & variance & skew & kurtosis & normal-t &
normal-p & $\alpha$
\\\addlinespace
\midrule
\textbf{abs} & 11 & 330 & 1523 & 842.27 & 132229.22 & 0.67 & -0.65 &
1.467 & 0.4803 & 0.9800
\\\addlinespace
\textbf{rel} & 11 & 1 & 7 & 4.21 & 3.31 & 0.67 & -0.65 & 1.467 & 0.4803
& 0.9800
\\\addlinespace
\textbf{mean} & 11 & 5 & 8 & 6.38 & 1.01 & 1.14 & 0.11 & 4.835 & 0.0891
& 0.9800
\\\addlinespace
\bottomrule
    \label{tab:exponent_fm}
\end{longtable}

\begin{figure}
    \centering
    \includegraphics[width=.7\textwidth]{img/exponent_fm_diff.png}
    \caption{A histogram depicting the exponents per item of the flashmap users}
    \label{fig:exponent_fm_diff}
\end{figure}
\begin{figure}
    \centering
    \includegraphics[width=.7\textwidth]{img/exponent_fm_abil.png}
    \caption{A histogram depicting the exponents per flashmap user}
    \label{fig:exponent_fm_abil}
\end{figure}

\begin{longtable}[c]{@{}lrrrrrrrrrr@{}}
    \caption{Combined conditions}
    \endfirsthead
\toprule\addlinespace
& sample & min & max & mean & variance & skew & kurtosis & normal-t &
normal-p & $\alpha$
\\\addlinespace
\midrule
\textbf{abs} & 23 & 218 & 1523 & 661.30 & 112385.58 & 1.09 & 0.62 &
6.912 & 0.0316 & 0.9632
\\\addlinespace
\textbf{mean} & 23 & 4 & 11 & 6.53 & 1.81 & 1.95 & 4.68 & 23.191 &
0.0000 & 0.9632
\\\addlinespace
\bottomrule
    \label{tab:exponent_gen}
\end{longtable}

\begin{figure}
    \centering
    \includegraphics[width=.7\textwidth]{img/exponent_gen_diff.png}
    \caption{A histogram depicting the exponents per item}
    \label{fig:exponent_gen_diff}
\end{figure}
\begin{figure}
    \centering
    \includegraphics[width=.7\textwidth]{img/exponent_gen_abil.png}
    \caption{A histogram depicting the exponents per user}
    \label{fig:exponent_gen_abil}
\end{figure}

\section{Comparisons of the exponents}

\begin{longtable}[c]{@{}lrrrr@{}}
\toprule\addlinespace
& \textbf{Mann-Whitney-U k} & \textbf{Mann-Whitney-U p} &
\textbf{Welch's t-test k} & \textbf{Welch's t-test p}
\\\addlinespace
\midrule\endhead
\textbf{abs} & -2.853 & 0.0095 & -2.786 & 0.0136
\\\addlinespace
\textbf{rel} & 1.319 & 0.2013 & 1.330 & 0.1978
\\\addlinespace
\textbf{mean} & 0.497 & 0.6246 & 0.507 & 0.6182
\\\addlinespace
\bottomrule
    \label{tab:exponent_comp}
\end{longtable}

\begin{figure}
    \centering
    \includegraphics[width=.7\textwidth]{img/exponent_diff.png}
    \caption{A comparison of figure~\protect\ref{fig:exponent_fc_diff} and~\protect\ref{fig:exponent_fm_diff}}
    \label{fig:exponent_diff}
\end{figure}
\begin{figure}
    \centering
    \includegraphics[width=.7\textwidth]{img/exponent_abil.png}
    \caption{A comparison of figure~\protect\ref{fig:exponent_fc_abil} and~\protect\ref{fig:exponent_fm_abil}}
    \label{fig:exponent_abil}
\end{figure}

\FloatBarrier
\section{Descriptives of percentage of responses marked as correct}

\begin{longtable}[c]{@{}lrrrrrrrrrr@{}}
\caption{Flashcard condition}
\endfirsthead
\toprule\addlinespace
& N & min & max & mean & variance & skew & kurt & norm-t &
norm-p & $\alpha$
\\\addlinespace
\midrule
\textbf{abs} & 12 & 35 & 86 & 62.33 & 267.67 & -0.30 & -1.07 & 0.993 &
0.6086 & 0.9780
\\\addlinespace
\textbf{rel} & 12 & 0.38 & 0.93 & 0.67 & 0.03 & -0.30 & -1.07 & 0.993 & 0.6086
& 0.9780
\\\addlinespace
\textbf{mean} & 12 & 0 & 0 & 0.86 & 0.00 & -0.19 & -0.82 & 0.242 &
0.8859 & 0.9780
\\\addlinespace
\bottomrule
    \label{tab:score_fc}
\end{longtable}

\begin{figure}
    \centering
    \includegraphics[width=.7\textwidth]{img/score_fc_diff.png}
    \caption{A histogram depicting percentages of correct answers by flashcard users per item}
    \label{fig:score_fc_diff}
\end{figure}
\begin{figure}
    \centering
    \includegraphics[width=.7\textwidth]{img/score_fc_abil.png}
    \caption{A histogram depicting percentages of correct answers per flashcard user}
    \label{fig:score_fc_abil}
\end{figure}

\begin{longtable}[c]{@{}lrrrrrrrrrr@{}}
\caption{Flashmap condition}
\endfirsthead
\toprule\addlinespace
& N & min & max & mean & variance & skew & kurt & norm-t &
norm-p & $\alpha$
\\\addlinespace
\midrule
\textbf{abs} & 11 & 46 & 185 & 117.55 & 2006.59 & 0.05 & -1.06 & 0.548 &
0.7605 & 0.9928
\\\addlinespace
\textbf{rel} & 11 & 0.23 & 0.93 & 0.59 & 0.05 & 0.05 & -1.06 & 0.548 & 0.7605
& 0.9928
\\\addlinespace
\textbf{mean} & 11 & 0 & 1 & 0.89 & 0.00 & 0.22 & -1.19 & 1.178 & 0.5549
& 0.9928
\\\addlinespace
\bottomrule
    \label{tab:score_fm}
\end{longtable}

\begin{figure}
    \centering
    \includegraphics[width=.7\textwidth]{img/score_fm_diff.png}
    \caption{A histogram depicting percentages of correct answers by flashmap users per item}
    \label{fig:score_fm_diff}
\end{figure}
\begin{figure}
    \centering
    \includegraphics[width=.7\textwidth]{img/score_fm_abil.png}
    \caption{A histogram depicting percentages of correct answers per flashmap user}
    \label{fig:score_fm_abil}
\end{figure}

\begin{longtable}[c]{@{}lrrrrrrrrrr@{}}
\caption{Combined conditions}
\endfirsthead
\toprule\addlinespace
& N & min & max & mean & variance & skew & kurt & norm-t &
norm-p & $\alpha$
\\\addlinespace
\midrule
\textbf{abs} & 23 & 35 & 185 & 88.74 & 1841.51 & 0.93 & -0.13 & 4.256 & 0.1191 & 0.9910
\\\addlinespace
\textbf{mean} & 23 & 0 & 1 & 0.87 & 0.00 & 0.07 & -0.67 & 0.272 & 0.8728
& 0.9910
\\\addlinespace
\bottomrule
    \label{tab:score_gen}
\end{longtable}

\begin{figure}
    \centering
    \includegraphics[width=.7\textwidth]{img/score_gen_diff.png}
    \caption{A histogram depicting percentages of correct answers per item}
    \label{fig:score_gen_diff}
\end{figure}
\begin{figure}
    \centering
    \includegraphics[width=.7\textwidth]{img/score_gen_abil.png}
    \caption{A histogram depicting percentages of correct answers per user}
    \label{fig:score_gen_abil}
\end{figure}

\section{Comparisons of the percentage of responses marked as correct}

\begin{longtable}[c]{@{}lrrrr@{}}
\toprule\addlinespace
& \textbf{MW k} & \textbf{MW p} &
\textbf{t-test k} & \textbf{t-test p}
\\\addlinespace
\midrule
\textbf{abs} & -16.597 & 0.0000 & -15.857 & 0.0000
\\\addlinespace
\textbf{rel} & -16.421 & 0.0000 & -15.689 & 0.0000
\\\addlinespace
\textbf{mean} & -12.448 & 0.0000 & -11.895 & 0.0000
\\\addlinespace
\bottomrule
    \label{tab:score_comp}
\end{longtable}

\begin{figure}
    \centering
    \includegraphics[width=.7\textwidth]{img/score_diff.png}
    \caption{A comparison of figure~\protect\ref{fig:score_fc_diff} and~\protect\ref{fig:score_fm_diff}}
    \label{fig:score_diff}
\end{figure}
\begin{figure}
    \centering
    \includegraphics[width=.7\textwidth]{img/score_abil.png}
    \caption{A comparison of figure~\protect\ref{fig:score_fc_abil} and~\protect\ref{fig:score_fm_abil}}
    \label{fig:score_abil}
\end{figure}

\FloatBarrier
\section{Descriptives of the amount of time spent on the application}

\begin{longtable}[c]{@{}lrrrrrrrrrr@{}}
\caption{Flashcard condition}
\endfirsthead
\toprule\addlinespace
& N & min & max & mean & variance & skew & kurt & norm-t &
norm-p & $\alpha$
\\\addlinespace
\midrule
\textbf{abs} & 12 & 1697 & 19721 & 12374.41 & 26140529.78 & -0.61 &
-0.33 & 1.445 & 0.4855 & 0.9239
\\\addlinespace
\textbf{rel} & 12 & 18 & 212 & 133.06 & 3022.38 & -0.61 & -0.33 & 1.445
& 0.4855 & 0.9239
\\\addlinespace
\textbf{mean} & 12 & 35 & 249 & 169.77 & 3881.30 & -1.00 & 0.11 & 4.064
& 0.1311 & 0.9239
\\\addlinespace
\bottomrule
    \label{tab:time_fc}
\end{longtable}

\begin{figure}
    \centering
    \includegraphics[width=.7\textwidth]{img/time_fc_diff.png}
    \caption{A histogram depicting the time spent in seconds per item by flashcard users} 
    \label{fig:time_fc_diff}
\end{figure}
\begin{figure}
    \centering
    \includegraphics[width=.7\textwidth]{img/time_fc_abil.png}
    \caption{A histogram depicting the time spent in seconds per flashcard user}
    \label{fig:time_fc_abil}
\end{figure}

\begin{longtable}[c]{@{}lrrrrrrrrrr@{}}
\caption{Flashmap condition}
\endfirsthead
\toprule\addlinespace
& N & min & max & mean & variance & skew & kurt & norm-t &
norm-p & $\alpha$
\\\addlinespace
\midrule
\textbf{abs} & 11 & 2612 & 26869 & 14121.58 & 51451085.83 & 0.65 & -0.07
& 1.907 & 0.3854 & 0.9501
\\\addlinespace
\textbf{rel} & 11 & 13 & 135 & 70.96 & 1299.24 & 0.65 & -0.07 & 1.907 &
0.3854 & 0.9501
\\\addlinespace
\textbf{mean} & 11 & 19 & 224 & 117.30 & 3303.44 & 0.24 & -0.34 & 0.380
& 0.8271 & 0.9501
\\\addlinespace
\bottomrule
    \label{tab:time_fm}
\end{longtable}

\begin{figure}
    \centering
    \includegraphics[width=.7\textwidth]{img/time_fm_diff.png}
    \caption{A histogram depicting the time spent in seconds per item by flashmap users} 
    \label{fig:time_fm_diff}
\end{figure}
\begin{figure}
    \centering
    \includegraphics[width=.7\textwidth]{img/time_fm_abil.png}
    \caption{A histogram depicting the time spent in seconds per flashmap user}
    \label{fig:time_fm_abil}
\end{figure}

\begin{longtable}[c]{@{}lrrrrrrrrrr@{}}
\caption{Combined conditions}
\endfirsthead
\toprule\addlinespace
& N & min & max & mean & variance & skew & kurt & norm-t &
norm-p & $\alpha$
\\\addlinespace
\midrule
\textbf{abs} & 23 & 1697 & 26869 & 13210.01 & 37253452.58 & 0.43 & 0.55
& 2.350 & 0.3088 & 0.9281
\\\addlinespace
\textbf{mean} & 23 & 19 & 249 & 144.67 & 4160.26 & -0.29 & -0.93 & 1.630
& 0.4427 & 0.9281
\\\addlinespace
\bottomrule
    \label{tab:time_gen}
\end{longtable}

\begin{figure}
    \centering
    \includegraphics[width=.7\textwidth]{img/time_gen_diff.png}
    \caption{A histogram depicting the time spent in seconds per item} 
    \label{fig:time_gen_diff}
\end{figure}
\begin{figure}
    \centering
    \includegraphics[width=.7\textwidth]{img/time_gen_abil.png}
    \caption{A histogram depicting the time spent in seconds per user}
    \label{fig:time_gen_abil}
\end{figure}

\section{Comparisons of the amount of time spent on the application}

\begin{longtable}[c]{@{}lrrrr@{}}
\toprule\addlinespace
& \textbf{MW k} & \textbf{MW p} &
\textbf{t-test k} & \textbf{t-test p}
\\\addlinespace
\midrule
\textbf{abs} & 7.522 & 0.0000 & 7.869 & 0.0000
\\\addlinespace
\textbf{rel} & 7.787 & 0.0000 & 8.148 & 0.0000
\\\addlinespace
\textbf{mean} & 8.720 & 0.0000 & 9.126 & 0.0000
\\\addlinespace
\bottomrule
    \label{tab:time_comp}
\end{longtable}

\begin{figure}
    \centering
    \includegraphics[width=.7\textwidth]{img/time_diff.png}
    \caption{A comparison of figure~\protect\ref{fig:time_fc_diff} and~\protect\ref{fig:time_fm_diff}}
    \label{fig:time_diff}
\end{figure}
\begin{figure}
    \centering
    \includegraphics[width=.7\textwidth]{img/time_abil.png}
    \caption{A comparison of figure~\protect\ref{fig:time_fc_abil} and~\protect\ref{fig:time_fm_abil}}
    \label{fig:time_abil}
\end{figure}


\end{document}
