\chapter{Design frameworks}
\label{ch:frameworks}

This chapter describes the general design frameworks used for the implementation of the use cases defined in the previous chapter. These frameworks are Concept map construction design features \cite{constructcmaps}, flashcard learning design features \cite{nakata}, the cognitive load theory \cite{cognitiveload} for multimedia design, and the two factor theory for website design \cite{websitedesign}. The aim is to implement these features from the next sections within the flashmap software in order to enhance the quality of the tool. In the \nameref{sec:screening} section of the \nameref{ch:evaluation} chapter on page~\pageref{sec:screening}, it is evaluated how the sofware adheres to the design features.

    \section{Concept map construction design features}

Unfortunately, little is written about the design of clear concept maps for supplantive use. Only one technical report by \cite{constructcmaps} is written on how students should construct concept maps, however the main focus of this document is on how this process can be scaffolded rather than discussing good design features of concept maps. The only two features mentioned are:

\begin{itemize}
    \item Cross-links are important in order to show relationships between sub-domains in the map
    \item One should avoid ``String maps'', which are maps mainly consisting of large sentences within the nodes
\end{itemize}

The developed concept map therefore focuses on meaningful relations between concepts rather than hierarchical structure, and divides nodes into smaller nodes whenever one node becomes to large.


    \section{Flashcard learning design features}

\citeA{nakata} describes a framework for developing flashcard applications based on several design features generally present in all major flashcard systems, and findings of earlier studies. The framework is split up in features aimed at creation and editing of flashcards, and at learning of flashcards. However, the creation and editing features are not relevant to the design of the flashmap system, since the content is already developed for the students.

The flashcard learning features are expounded in the following subsections.

        \subsection{Presentation and retrieval modes}

It is recommended to use two different modes, namely the presentation mode --- where users can familiarise themselves with not seen before flashcards ---, and the retrieval mode --- where the user tries to actively retrieve a target when shown the associated cue. The presentation mode is introduced, because retrieval of unfamiliar targets would only result in unsuccessful performance and negative effects on the motivation of the user. Nonetheless, it was decided to only include the retrieval mode, since the students have already familiarised themselves by listening to the teacher explanation and reading the book, which can already be regarded as a presentation mode.

        \subsection{Retrieval practices}



        \subsection{Increasing retrieval effort}

        \subsection{Generative use}

        \subsection{Block size and Adaptive sequencing}

            \paragraph{Pimsleur system}

            \paragraph{Leitner system}

        \subsection{Expanded rehearsal}


    \section{Cognitive load theory}

        \subsection{Modality effect}

        \subsection{Segmentation effect}

        \subsection{Pretraining effect}

        \subsection{Coherence effect}

        \subsection{Signaling effect}

        \subsection{Spatial contiguity effect}

        \subsection{Redundancy effect}

        \subsection{Temporal contiguity effect}

        \subsection{Spatial ability effect}


    \section{Two factor theory for website design}

        \subsection{Authorised use of the user's data for unanticipated purposes}

        \subsection{Authorised collection of user data}

        \subsection{Sharp displays}

        \subsection{Support for different browsers}

        \subsection{Stability of the website availability}

        \subsection{Effective navigation aids}

        \subsection{Clear directions for navigating the website}

        \subsection{Structure of information presentation is logical}

        \subsection{Presence of improper materials}

        \subsection{Accurate information}

        \subsection{Up-to-date information}

        \subsection{Content that supports the website's intended purpose}

        \subsection{Importance of the surfing activity to the user}

        \subsection{High level of learned new knowledge and/or skills by doing the surfing activity on the website}

        \subsection{Presence of the use of humor}

        \subsection{Presence of multimedia}

        \subsection{Fun to explore}

        \subsection{Presence of assurance that user entered data is encrypted}

        \subsection{Users can control how fast to go through the website}

        \subsection{Users can control complexity of mechanisms for accessing information}

        \subsection{Visually attractive screen layout}

        \subsection{Attractive screen background and pattern}

        \subsection{High reputation of the website owner}

        \subsection{Presence of external recognition of the website}

        \subsection{Presence of controversial material}

        \subsection{Presence of novel information}
