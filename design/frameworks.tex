\chapter{Design frameworks}
\label{ch:frameworks}

This chapter describes the general design frameworks used for the implementation of the use cases defined in the previous chapter. These frameworks are Concept map construction design features \cite{constructcmaps}, flashcard learning design features \cite{nakata}, and the two factor theory for website design \cite{websitedesign}. The aim is to implement these features from the next sections within the flashmap software in order to enhance the quality of the tool. In the \nameref{sec:screening} section of the \nameref{ch:evaluation} chapter on page~\pageref{sec:screening}, it is evaluated how the sofware adheres to the design features.

    \section{Concept map construction design features}

Unfortunately, little is written about the design of clear concept maps for supplantive use. Only one technical report by \cite{constructcmaps} is written on how students should construct concept maps, however the main focus of this document is on how this process can be scaffolded rather than discussing good design features of concept maps. The only two features mentioned are:

\begin{itemize}
    \item Cross-links are important in order to show relationships between subdomains in the map (also mentioned by \citeA{eppler})
    \item One should avoid ``String maps'', which are maps mainly consisting of large sentences within the nodes
\end{itemize}

The developed concept map therefore focuses on meaningful relations between concepts rather than hierarchical structure, and divides nodes into smaller nodes whenever one node becomes to large.

Finally, \citeA{eppler} describes that concept maps are top-down diagrams rather than radial diagrams, therefore the graph layout will be hierarchical with top-down orientation.

    \section{Flashcard learning design features}

\citeA{nakata} describes a framework for developing flashcard applications based on several design features generally present in all major flashcard systems, and findings of earlier studies. The framework is split up in features aimed at creation and editing of flashcards, and at learning of flashcards. However, the creation and editing features are not relevant to the design of the flashmap system, since the content is already developed for the students. Furthermore, the Flashcard systems reviewed by \citeA{nakata} are aimed at vocabulary learning, so not all of the guidelines mentioned in the review can be fully generalised to the flashmap system. That being said, there is still quite some overlap between the functionality of these reviewed systems and the flashmap system, and even if a principle is not (completely) applicable it is still relevant to know why this is not the case and what should be the guideline instead.

The flashcard learning features can be split up in two subcategories, which are features related to how the cards are presented (Presentation and retrieval modes, Retrieval practices, Increasing retrieval effort, and Generative use), and features related to the rescheduling of cards for review (Block size, Adaptive sequencing, and Expanded rehearsal). All features are expounded in the following subsections.

        \subsection{Presentation and retrieval modes}

It is recommended to use two different modes, namely the presentation mode --- where users can familiarise themselves with not seen before flashcards ---, and the retrieval mode --- where the user tries to actively retrieve a target when shown the associated cue. The presentation mode is introduced, because retrieval of unfamiliar targets would only result in unsuccessful performance and negative effects on the motivation of the user. Nonetheless, it was decided to only include the retrieval mode, since the students have already familiarised themselves by listening to the teacher explanation and reading the book, which can already be regarded as a presentation mode.

        \subsection{Retrieval practices}

Retrieval practices relate to the ways the system prompts the user to recall a target from memory. Within vocabulary learning, there are four different categories of retrieval practices divided into 2 axes: reception versus production, and recall versus recognition. 

\paragraph{Reception and production} The first axis of reception and production relates to which part of the associated pair should be retrieved. Reception means that the meaning of the word should be retrieved, whereas production refers to retrieving the target word. For example, when having the pair "Goedenmiddag" and "Good afternoon" while learnig Dutch vocabulary, with reception "Good afternoon" would have to be retrieve when being shown "Goedenmiddag", whereas in productive recall "Goedenmiddag" would have to be retrieved when being shown "Good afternoon".

\paragraph{Recall and recognition} The second axis refers to how the retrieval takes place, namely whether the student tries to recall the cue from memory (recall), or whether he chooses the correct answer from a list of possibilities (recognition). For example, when presented with "Goedenmiddag", when using recall the student should think of the correct answer on his own, where when using recognition the student should choose the correct answer from a list, e.g. "Good morning", "Good afternoon", or "Good evening".

According to \citeA{nakata}, it is difficult for students to aquire both the word form-meaning connection and the word form of a word simultaneously, mainly because of limited cognitive resources. Therefore, good vocabulary learning software should split these tasks into separate exercises: one using reception or recognition (or both), and one using specifically productive recall.

In the case of using flashcards for learning texts, reception and production could be interpreted as prompting the question versus prompting the answer. Recall and recognition still work in the same way, namely as an open prompt or a multiple-choice prompt. In flashmaps however there are not 2 elements but 3 elements per prompt --- the parent node, the edge, and the child node --- and this makes way for 6 options rather than 2 options for the reception-production axis, which are displayed in table~\ref{tab:retrievalmaps}. Within the table they are ordered from leaning towards reception to learning towards production, with the underlying hypothesis that retrieving the child node is more in allignment with reception whereas retrieving the parent node is more in allignment with production. This of course also depends on the concept map. The recall and recognition axis still stays the same, where in the latter case the possible options could be displayed next to the flashmap.

\begin{table}
    \centering
    \begin{tabular}{lccc}
        \hline
         & Parent node & Edge & Child node \\
        \hline
        1 & X & X & \\
        2 & X & & X \\
        3 &  & X & X \\
        4 & X & & \\
        5 &  & X & \\
        6 &  & & X \\
        \hline
    \end{tabular}
    \caption{The different possibilies for the reception-production axis in the flashmap system, with an hypothesised ordering from more leaning towards reception to more towards production. The shown elements for each retrieval mode are indicated by an X, whereas the other elements would be the targets for retrieval.}
    \label{tab:retrievalmaps}
\end{table}

Only one retrieval practice was chosen for this experiment in order to keep the amount of variables to a minimum and thereby make a better comparison between the conditions. For the flashcard condition, the receptive recall was chosen, since this was the skill most relevant to the test itself. The flashmap equivalent was retrieval practice 1 from table~\ref{tab:retrievalmaps} combined with recall (in contrast to recognition). Other retrieval practices are however still worth investigating in further research.

Another consideration here not mentioned by \citeA{nakata} how the answer retrieved by the student should be compared with the actual answer. In the case of recognition, the user could simply click on the answer he thinks to be the correct answer, which can then be compared to the correct answer. Within the case of recall however, the student either has to merely indicate whether his response was in allignment with the correct response, or he has to type his answer which is then compared to the correct response by the software. The latter option was chosen, since this is easier for students using a mobile plattform to only press a button in comparison to typing on the screen keyboard. Furthermore, typing out an answer can lead to unintended rigidity of the system, such as an answer being marked as incorrect with spelling errors or omitted articles, or a highly increased complexity when trying to take into account these alternative correct responses. The only downside to this decision is that within the results there is an increased unreliability in how leniant the students were with their response being correct or incorrect.

        \subsection{Increasing retrieval effort}

The design feature of increasing retrieval effort is strongly related to the choices made with regards to the previous design features. It entails that over the course of multiple presentations of one associated pair, the challenge of retrieval should be increasing. \citeA{nakata} describes that this can be achieved by starting with the presentation mode before introducing the retrieval mode, and by gradually shifting from recognition modes to recall modes and from receptive modes to production modes. However, since the presentation mode is omitted and only one retrieval practice is chosen within this project, increasing retrieval effort is not feasible in this way. This again could be incorporated in later prototypes where other retrieval practices are incorporated.

        \subsection{Generative use}

Generative use of words refers to presenting words in novel contexts. In the case of vocabulary learning, one can use specific words in different sentences which underly the different meanings of the words. Generative use also enhances the elaborative processing of certain concepts. Within the flashmap system, this is incorporated by presenting the concepts together with different edges, sometimes because a concept switches from child node to parent node, but in other cases because multiple unique edges direct from or to a concept. Within the flashcard system, this is also the case but more implicit, because every edge (or group of similar edges) being translated to flaschards, and thereby incorporating the concepts in multiple questions and answers. The only concepts appearing in one instance would be the concepts with only one outgoing edge (root node, K=1), since they only appear in one question, or only one incoming edge (leaf node, K=1), since they only appear in one answer. Of these, the root node promblems can be eliminated by creating more direct subconcepts in the hierarchy or omitting the direct subconcept and linking its subconcept direct to the root node. The leaf node problems are more difficult to eliminate, but still can be linked to other concepts by creating more cross-links generating more incoming or outgoing edges on the lower levels. However, this is not always possible, since some leaf nodes cannot be meaningfully connected to other nodes. In this case it is also worth considering whether this node could be eliminated altogether.

        \subsection{Block size}

Within the \nameref{subsec:spacingeffect} section on page~\pageref{subsec:spacingeffect} it is described how repetition of an item is more effective when interleaved by other items (spaced items) than when repeated in immediate succession (massed items). The block size is therein defined as the length of items after which items are repeated again. When using massed items, one would have a block size of one, whereas when interleaving each repetition by 8 other items, the block size would be 9. The \nameref{subsec:implicationsflashcards} section however describes why it is better to use adaptive spaced-repetion learning instead of fixed block sizes, which is also recommended by \citeA{nakata}. Therefore, no specific fixed block size will be used within the scheduling algorithm.
        
        \subsection{Expanded rehearsal}

According to \citeA{nakata}, expanded rehearsal is widely believed to be the most effective. The main difficulty for choosing the right slope for the expanded rehearsal is to balance between overlearning --- repeating items too often, reducing the effectiveness of each repetition --- and the forgetting curve --- repeating items too little, leading to the students forgetting the card and thereby frustrating the user and even ineffective studying.

The first system which implemented this system is the Pimsleur system, where over the course of a 30-minute audio lessons words would be presented in a progressive series of exponentially expanding intervals with a base value of 5 seconds \cite{microlearning}. This means that the first repetition would take place after $5^1=5$ s, the second after $5^2=25$, the third after $5^3=125$ s, etc. This system thereby already took into account the decreasing slope of the forgetting curve because of the power law of learning (see figure~\ref{fig:spacedrepetition} on page~\ref{fig:spacedrepetition}), preserving a steady retrieval chance with a decreasing amount of repetitions. This curve is possibly too flat and leads to increased overlearning, however overlearning can also lead to a benefitial confidence within the user because of the higher percentage in correct retrievals. The Pimsleur system is also rather simple, reducing the amount of variables in the research. Because of these reasons, it is chosen as the basis for the flashmap scheduling algorithm.

        \subsection{Adaptive sequencing}
        \label{subsec:adaptivesequencing}

An adaptive sequencing algorithm takes into account the learners' previous performance on individual items when rescheduling an item for the next review. Within the original Pimsleur system, one would always increase the time interval for the next review independent of whether the student could correctly recall the item or not. However, this does not account for the flashcards which are more difficult or forgotten at the time of the new review, resulting in the user not being able to keep up with those cards.

The first system to implement an adaptive sequencing element is the Leitner system, which is also the most basic adaptive sequencing system. The user has a number of piles, each representative of a expanded time interval, and a stack of physical flashcards. Each time a flashcard is answered correctly, it moves to the next pile, resulting in a larger time interval before the next repetition, and when answered incorrectly it would be moved back to the first pile. The rationale behind this system is that when an item cannot be retrieved it is forgotten, and the expanded rehearsal should therefore be reset to the lowest value.

The main problem with the Leitner system is that when it was introduced, managing the flashcards and different piles was quite a hassle. This problem was resolved with the introduction of digital flashcards, since the computer could take care of the scheduling and bookkeeping of the flashcards and their reviews. The Leitner system is therefore still prevalent in most digital flashcard systems (e.g. superMemo, Anki, and FaCT).

When combining the Pimsleur and Leitner system as one system, one gets the formula $i = 5^c$ for scheduling the flashcards, where $i$ is the time interval in seconds, and $c$ the amount of times the flashcard was correctly retrieved for this item since the last incorrect retrieval (or the total amount of retrievals when there were no incorrect retrieval).

\citeA{microlearning} introduced an even more sophisticated system, namely the adaptive spaced repetition system. On top of the Pimsleur intervals and Leitner adaptive sequencing, it also adapts the time intervals based on the amount of answers correctly and incorrectly answered by the student for each time interval. This results in specific time intervals better catered towards the user's ability. This is not included within the scheduling algorithm used within this project, because it adds another variable to the experiment. It can still be included however when aiming to improve the prototype.

    \section{The ARCS model}

In the \nameref{sec:learneranalysis} section, it was already found that it is likely for the students to have a low intrinsic motivation for engaging into the subject matter. Therefore, it is also important to include a framework within the motivational domain next to the previous framework focusing mainly on the cognitive domain. A commonly used and well researched model for incorporating motivational features is the ARCS model \cite{arcs}, which is an acronym for Attention, Relevance, Confidence, and Satisfaction. The factors were applied within the application where possible, however many are also applied within the context outside of the application. They all can be expressed in three subcategories, described in the following sections.

        \subsection{Attention}

In this instance, the category of attention mainly refers to gaining attention at the beginning of an instruction, whereas keeping the attention is covered by the other categories. It can be gained by simple unexpected events, or mentally stimulating problems engaging the learner. Furthermore, variety is important for the effect not wearing down.

\paragraph{Perceptual arousal} This mainly refers to the simple events --- such as whistles or strange imagery --- in order to capture the interest from the learner. In order to achieve this for drawing the user towards using the system, the researcher went to the classroom in the last lesson where the subject material was taught, where the attention was mainly gained from the effect of having someone else than the teacher appearing in the classroom. Within the system itself it was harder to achieve, since the content of the instruction was dynamically generated and therefore it is more difficult to find fitting content for each study session. It would then also add extraneous cognitive load, which is undesirable both for the learning achievement as well as the experimental setting.

\paragraph{Inquiry arousal} This relates to the deeper, mentally stimulating problems which can be offered to the student in order to activate engagement with the topic. Related to the content itself, this is mainly done by the teacher within the lessons and by the instructional material. The researcher also tried to stimulate inquiry by stating that this was an early opportunity for the students to experience what research on a university looks like. Finally, by continuously asking questions users are also stimulated to actively participate in the instruction.

\paragraph{Variability} Finally, if the setup for every instruction is the same, learners will eventually get disengaged because of the predictability. Using this system instead of the usual chapter reading in preparation of an exam might be an example of breaking such a predictable pattern. On the other hand, one of the downsides of drill and practice is its repetitive nature, so this could demotivate or disengage the students.

        \subsection{Relevance}

Gaining arousal is not enought to keep the learner engaged over a long period of time. One method therefore to keep the attention is for the user to understand why it is relevant to engage in the learning activity. The subject matter however is not that interesting towards the average high school student. Therefore, the presentation of the system by the researcher will mainly focus on how it can help students to effectively and efficiently prepare them for the exam, since this is likely to be their direct goal.

\paragraph{Goal orientation} The first step in establishing relevance is to relate to the needs and goals of the learner. In order to achieve this, the introduction by the researcher mainly delves into how a general flashcard system works and how it benefits learning, making the process more effective and efficient than purely reading through the book. Added to this, it is mentioned that some of the flashcard and test questions will be repeated on the actual test used for grading the students, making it extra attractive because of the sneak preview.

\paragraph{Motive matching} This relates more to provide learners with appropriate choices, responsibilities, and influences. One way to do this is by modeling, derived from the theory of planned behaviour \cite{theoryplannedbehaviour}. In order to convince the students that the software is reliable, the researcher adds an anecdote from his own experience of using the system, and that by using it he had a guarantee of being well prepared for exams because of how the algorithm works.

\paragraph{Familiarity} This subcategory mainly refers to tying the content to the learners' experiences. Unfortunately, the content of the instruction does not lend itself well for this subcategory, hence it is also not used in the instructional material and thereby in the flashmap and flashcards. However, the teachers made great effort within the lessons itself to explain how the renaissance genres and techniques are still prevalent in todays writing, with examples known to the learner.

        \subsection{Confidence}

The learner must, next to feeling the subject matter is relevant, also be confident that he is able to learn the subject and to perform the learning activities successfully. This confindence can be boosted by creating the right expectations and providing positive feedback related to the learning activity.
        
\paragraph{Learning requirements} One way the confidence can be boosted on beforehand is by assisting in building a positive expectation for success within the learner. In this case this is done by the researcher first acknowledging that learning and comprehending the core message of a text can be difficult, but that the system can assist this process greatly and makes it easy to do. This is also done within the initial presentation of the software. Furthermore, within the presentation as well as within the software it is made clear that the researcher could support individual learners at any moment if they would get stuck using the application. Within the software this was achieved by including a separate help page with explanations on how to use the software, by including a small text above the main content explaining what the user should do within the current step, and an email address in the navigation menu for contacting the researcher himself.

\paragraph{Learning activities} Additionally, the learning activities themselves can also support or even enhance the students' beliefs in their competence. Within the flashcard system this is done by having the relatively flat expanded rehearsal slope from the Pimsleur system, which generates more overlearning and thereby increases the amount of correct retrievals. A more steep slope would result in more failed retrieval attempts, increasing the frustration within the user and thereby requiring a more stoic attitude.

\paragraph{Success attributions} Finally, the learner also has to attribute his success to the use of the system in order to be motivated to use it. However, according to \citeA{logan}, judgement of learning of students using spaced items does not appear to be higher than when using massed items. Therefore, a separate overview is included within the system displaying the progress the learner has made and how much items are still left.

        
        \subsection{Satisfaction}

Finally, in order to sustain motivation after the student is attentive towards, understands the relevance of, and is confident about performing the learning activities, satisfaction is required. This is mainly related to positive feelings stemming from the reward system, and are generally categorised in intrinsic and extrinsic motivation.

\paragraph{Self-reinforcement} Intrinsic satisfaction is probably the most powerfull satisfaction, since it works directly on the short-term. This feeling can be instigated for example when the learner is intrigued by the subject, but also when he experiences some form of achievement. It is already stated within the \nameref{sec:learneranalysis} that most students will most likely not be highly interested within the subject matter, and therefore the first category of intrinsic satisfaction will only be met in rare cases. However, successful retrieval does lead to a sense of achievement, next to it being a confidence boost and effective learning tool. Therefore, the flashcard system itself might already facilitate self-reinforcement.

\paragraph{Extrinsic rewards} An important extrinsic motivator for the student is to pass the test at the end of the course, for which this system is an effective aid. However, as the teacher already stated before the experiment, this motivator is probably not enough motivation for the students to spend extra free time in such a tool. This is mainly because students often overestimate their own abilities and thereby deeming the extra preparation as not necessary, and them being less able to oversee long-term consequences because of their prefrontal cortex still being in development. In order to create an extra reward, the students are rewarded with a coupon for icecream if they successfully committed to he whole experiment. This entailed filling in the pre- and posttest and questionnaire, and spending 15 minutes every day over the course of 6 days on the system. This reward of icecream seemed to be what the students were most excited about during the initial presentation by the researcher.

\paragraph{Equity} Finally, it is imporant that each learner has a feeling of fair treatment. This entails that there was an appropriate amount of work, that the work they did is related to the final test, and that there is no favouritism among students. Using the system might feel as extra work to the students in comparison to just reading the textbook, even if it attributes to a higher chance of passing the test. This is an extra reason to emphasise the added value of using the software during the presentation. Furthermore, the consistency of the learning content and objectives and the test is guaranteed by deriving the concept map directly from the instructional material, by including flashcards and pre- or posttest question on the school test, and by letting the teacher confirm that the flashcards cover what the students have to know for the test and that there are no extraneous flashcards. Finally, equity is guaranteed by the fact that all students have anonymous accounts and thereby exactly the same general treatment during the learning procedure. The only contingency is that some students are using the flashcard system while others using the flashmap system, which might provide one group with advantages over the other group.
