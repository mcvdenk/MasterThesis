\chapter{Defining the general use cases}

\section{Supplantive or generative}

As already described before, instructional content such as flashcards or a concept map can be either supplied to or generated by the students. This dichotomy is described in further detail by \citeA{instructionaldesign}, where the implications of both sides are enlisted for the learner, the task and the context.

One of the aspects of generative strategies is that the learner requires a higher amount of prior knowledge, a higher aptitude, and a wider and more flexible range of cognitive strategies, because the content still has to be (partly) researched and constructed. This can be a disadvantage, because the learner might not possess these skills and therefore the instruction may not be suitable or highly inefficient using generative strategies. On the other hand, greater mental effort generally leads to greater depth of processing and therefore better, more meaningful learning, which was also stressed by \citeA{canas} and \citeA{nesbit}. Furthermore, learners experience a higher motivation and a lower amount of anxiety when using generative strategies, and their attribution of success is internal rather than external. Therefore, from a learners perspective, the question whether to supply or to let the learner generate the content is a choice between being efficiency and engagement.

Although in this thesis, supplantive and generative are described as extremes on a scale with options in between, in the case of this project it is more of a binary choice, since the chit is between either the students generating the content or someone else.

\section{Choice of platform}

\section{Supported user actions and system reactions}
