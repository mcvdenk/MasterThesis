\chapter{Theoretical Framework}

%Introduction chapter

%Make clear that we are talking about Declarative Semantic knowledge

%General memory theory

\section{Modal model of memory}

\begin{figure}
    \centering
    \includegraphics[width=0.5\textwidth]{img/brainareas.png}
    \caption{The brain areas mainly involved in storing and retrieving declarative knowledge \protect\cite{amnesia}}
    \label{fig:brainareas}
\end{figure}

Although the whole brain is involved in storing memories, the frontal lobes, medial septum and the hippocampus are the most prominent areas facilitating the process of memorising \cite{cognitivepsychology} (see figure~\ref{fig:brainareas}. The prefrontal regions are responsible for the creation and retrieval of memories, whereas the hipocampal and surrounding areas are responsible for permanent storage of these memories. Because of this dynamic, \citeA{modalmemory} conceived a modal theory of memory, displayed in figure~\ref{fig:modalmemory}. In this model, information is perceived as sensory input, and is then shortly stored in the sensory memory. If the perceiver has paid enough attention to the input, it is then transfered (or encoded) into short-term memory. When the input is strong enough, that is, rehearsed often enough within short term memory, it can be more permanently stored in long-term memory. If not, the input fades away from memory and is forgotten. When a memory exists in long-term memory, it has to be retrieved into short-term memory in order to be remembered and used.

This model was heavily influenced by developments in electrical engineering and computer sciences. Rather than thinking about memories in the form of physical neurons in the brain, Cajal found in the late 19th century that memories were patterns of electricity through neurons by means of synapses, which were thought to function like electrical wires \cite{longtermpotentiation}. The later modal model can be thought of as functioning like a complex computer, where data is written on a hard drive (the long-term memory), and can be used by first retrieving it into working memory (or short-term memory) and later be transferred to the hard drive again. However, the way the brain works is different from a computer in the sense that a brain has to put effort into memorising data, and that a brain forgets data over time. Therefore, instead of merely inputting the data, learning requires a more rigid approach.

\begin{figure}
    \centering
    \includegraphics[width=0.5\textwidth]{img/modalmemory.png}
    \caption{The modal model of memory proposed by \protect\citeA{modalmemory}}
    \label{fig:modalmemory}
\end{figure}

\citeA{karpicke4} describes two seperate learning practices based on the modal model of memory, namely encoding and retrieval practices, where encoding practices are focused on meaningful encoding or construction of knowledge, and retrieval practiced are more focused on the reconstruction and rehearsal of knowledge. He states that both practices are essential to enhancing learning. Flashcards are a famous retrieval practice, emphasising drilling the same facts over and over again by means of pairs by association, whereas concept maps are known to be an encoding practice where the student has to connect diverse concepts within one topic by meaningful relations.

\section{Flashcard system}

In the context of language learning, \citeA{nakata} defines a flashcard system as ``Target items [...] presented outside meaning-focused tasks, [where] learners are asked to associate the L2 [foreign] word form with its meaning, usually in the form of a first language translation, L2 synonym, or L2 definition'' (p. 17). In a generalised form, a flashcard system contains pairs by association, where the student is presented by one member of the pair and has to recall the other. This pair can indeed consist out of words in different languages, but can also consist out of a picture and a word for learning spelling, or a question with an answer for learning history. Traditionally, these pairs were presented as small cards (hence the name), and students would spend each day going through the stack of cards, preparing themselves for their exams. There are two main reasons for learning pairs in this form. The first is that students can repeat more difficult cards until they have answered them correctly, by putting the falsely answered cards on the bottom of the stack again and the correctly answered cards away for the next day. The second reason is a cognitive effect called the \emph{spacing effect}.

\subsection{Spacing effect}

The spacing effect is a well known effect occuring within paired-associate learning, and demonstrates that repeated items are better remembered when both occurences are seperated by other events or items than when they are presented in immediate succession \cite{verkoeijen, logan, siegel, xue, karpicke2}, which is demonstrated with diverse populations \cite{verkoeijen, logan}, under various learning conditions \cite{verkoeijen, logan}, and in both explicit and implicit memory tasks \cite{verkoeijen}. Items in immediate succession are called massed items, and items in seperated succession are called spaced items. Within flashcards, this effect is achieved by the student going through the deck on a card by card basis, interleaving each card by all other cards in the deck. 

One can test the spacing effect either by using pure lists or mixed lists. When using pure lists, one compares the effect of learning a list containing only massed items with a list containing only spaced items, and using mixed lists one measures the effect of learning both massed items and spaced items in one list, comparing their individual retentions. \citeA{verkoeijen} states that the vast majority of studies are conducted using mixed lists and found that spaced items where consistenly better recalled than massed items, yet studies using pure lists are relatively rare and have produced contradictory outcomes. They conducted a study providing participants first with an all-massed list, then letting them write down as many words as they could remember, and repeat an identical procedure for an all-spaced list with a 2 minute break inbetween. They conducted this experiment with short-lagged spaced items (with 1-4 items in between) and long-lagged spaced items (with 4-13), and found only a spacing effect in the latter experiment. However, \citeA{wahlheim} adds to this that repetition is only increases when a student detects the repetition of an item, and therefore the lag should not be too long.

Two theories have been presented explaining this phenomenon, namely the contextual variability theory and the study-phase retrieval theory \cite{siegel}. The first theory entails that because context is not static but continuous, and that therefore spaced items are studied in a greater variety of contexts and therefore easier to recall in yet other contexts than massed items due to the so-called encoding-specificity principle \cite{cognitivepsychology}. This principle entails that the probability of recalling an item depends on the similarity of the context during the encoding. The study-phase retrieval theory entails that additional retrieval cues for the repetition of an item are generated by earlier occurences and their associated contexts being associated with the repeated item. These theories are not mutually exclusive \cite{siegel}.

\citeA{karpicke} conducted an experiment to test the effect of constant or varying lags between items have a significant effect on learning. They tested this by conducting a similar experiment to \citeA{Verkoeijen}, however in this experiment they only tested pure lists with three different lag intervals to test for an absolute spacing effect, and for each lag interval category they tested for an expanding lag condition (where the lag would increase for the repetition of each next item), an equal lag condition (where the lag would remain constant) and a contracting lag condition (where the lag would decrease for the repetition of each next item) in order to test for a relative spacing effect. From their findings they confirmed the effect of absolute spacing, namely that longer gaps between items do have an effect on long-term retention, however they did not find a relative spacing effect.

Although the findings of the spacing effect are consistent, students do not judge their learning to be improved by it, even when they demonstrated a significantly higher recall rate \cite{logan}.



\subsection{Long-Term Potentiation}

%Flashcard system
    %First form of flashcards
    %Cognitive theory
        %Spacing effect
        %Effect on flashcards
        %Long term potentiation
            %Power law of learning and forgetting
            %Decay theory vs Interference theory
        %Effect on flashcards
            %Pimsleur system
            %Leitner system
    %Practical studies
        %Attitude
        %Effectiveness
    %Criticism

%Concept mapping
    %Cognitive theory
        %Spreading activation
        %Elaborative processing
        %Schemata
        %Constructionism
    %Visual mapping techniques
        %Formal definition concept mapping
    %Effect on concept mapping
        %canas
    %Practical studies
        %Attitude
        %Effectiveness
    %Criticism

%Comparison study (karpicke2)
