\chapter{Project Description}

\label{ch:problem}

Over the centuries, knowledge has been fundamental to any learning process. Socrates already stated that knowledge is the only true virtue, and the tragedian Aeschylus regarded memory as the mother of all knowledge. Moreover, it was not only regarded as important by ancient thinkers, but is still regarded as such by modern scholars on education. Both the taxonomy of learning by \citeA{bloom} as a revision of this taxonomy by \citeA{krathwohl}, as well as the three stages of skill acquisition by \citeA{skillacquisition}, propose that all learning should start with memorising factual knowledge. Furthermore, \citeA{glaserfield}, one of the main founders for critical constructivism, expresses a need for training students so that they permanently possess facts and are able to repeat them flawlessly whenever they are needed, while also understanding what is placed into their memory. \citeA{ltwm} adds to this by stating that in order to perform complex tasks, people must maintain access to large amounts of information, and that solely encoding knowledge is not sufficient. Despite all of this, \citeA{karpicke4} argues that ``[r]etrieval processes, the processes involved in using available cues to actively reconstruct knowledge, have received less attention'' (p. 158), whereas basic research on learning and memory has emphasised that retrieval must be considered in any analysis of learning.

Traditionally, when students have to gain complex and meaningful knowledge -- for example knowledge about a historical event or a chapter in a psychology textbook, they are asked to read the relevant chapter from a provided textbook. However, \citeA{learninginstruction} states that many students have difficulty gaining knowledge in this manner. He breaks reading for comprehension down into four separate skills, which are integrating, organising, elaborating, and monitoring. Integrating refers to relating a text to one's prior knowledge, for which evidence exists that rich background knowledge leads to better inferences about the text, and thereby to better comprehension. This need also has been stressed by \citeA{ausubel}, and forms different problems between individual readers having access to different background knowledge. After integration, the reader has to organise the text, so that the important ideas and the relationships among them are identified. This is mainly a problem for less experienced readers, possessing fewer strategies to quickly identify important parts and thereby spending too much time on reading unimportant information. While organising a text, the student also has to make necessary inferences while reading, or has to elaborate, which is quite difficult for readers when not prompted to do so. Finally, students have to monitor their comprehension, which refers to evaluating their understanding of the text and if necessary adjusting the reading strategy. This is again quite difficult for the average reader, however this can be trained.

While intergrating is something more dependent on the curriculum design, organising and elaborating can be facilitated by a technique called concept mapping, and monitoring by so-called flashcard systems. Furthermore, the latter might be helpful for the integration of the next topic with the current. This research aims to develop a new tool combining these learning tools. In this chapter, concept mapping, flashcard systems, and the new learning tool called the flashmap will be explored on a practical level in order to establish their definitions together with a summary of arguments in favour or opposition of using them as tools for studying textual material, while also describing their current applications within education.

\section{Concept mapping}

A Concept map is a learning tool deviced by Joseph Novak in 1970's, based on constructivist theories of learning. It was originally intended for assessing the structure of student conceptions, before and after instruction, in order to map their prior knowledge and compare it to what they learned during the instruction. This expanded on the notions of \citeA{ausubel}, who stated that what the learner already knows is most important, and that this had to be ascertained before teaching. Although the use of concept maps as an assessment tool remains prevalent \cite{canas, chung, hwang2, ruiz1}, over time, students began to use it as a tool to comprehend textual material by organising and elaborating on the included concepts \cite{canas, eppler, hwang2, karpicke2, nesbit}.

\subsection{Definition}

One definition provided by \citeA{burdo} states that "concept maps are hierarchical representations of knowledge. Construction of them involves linking concepts [...] through the use of linking phrases into propositional statements" (p. 335). The concepts are typically nouns or verbs with or without modifying adjectives or adverbs, and linking phrases specify the relationship between two concepts. \citeA{ruiz1} also mention these elements in their own definition, yet \citeA{canas} and \citeA{eppler} include a few extra features, such as the concepts being ordered in hierarchical fashion. They describe two different kinds of links, which are hierarchical links to indicate ranking between the concepts, and crosslinks to indicate relationships between concepts in different segments or domains of the concpt map. The latter would help to see how a concept in one domain of knowledge represented on the map is related to a concept in another part of the knowledge producer, enabling better connections to prior knowledge of the user. According to \citeA{eppler}, concept maps are always top-down and show systematic relationships among sub-concepts relating to one main concept, however \citeA{canas} state that they can also be cyclical as long as the concepts still have a conceptual hierarchy. Finally, most of the above mentioned articles describe the links between concepts to be directed. In conclusion, the definition of concept maps used within this thesis will be:

\begin{definition}
    A concept map refers to a directed graph, in which the nodes consist of concepts, and the edges of -- either hierarchical or cross- -- links labeled with linking phrases, forming several propositional statements about a knowledge domain.
\end{definition}

\noindent An example of a concept map is displayed in figure~\ref{fig:examplemap}.

For this study, the more interesting aspects of concept maps are the use of concept mapping for elaborating, and of demonstrating meaningful relationships between concepts to learners. The first use of the concept map is known as generative use, and the second as supplantive \cite{instructionaldesign}.

\begin{figure}
    \centering
    \includegraphics[width=\textwidth]{img/conceptmap.png}
    \caption{A fraction of the concept map used in this study}
    \label{fig:examplemap}
\end{figure}

\subsection{Effectiveness}

Multiple studies, both qualitative and quantitative, have demonstrated that concept maps can promote meaningful learning \cite{canas, hwang2, nesbit, subramaniam}. When comparing the concept mapping strategy with traditional teaching strategies (in a study conducted within the context of tertiary chemistry), \citeA{singh} found that the concept map teaching strategy was more effective, however that it was most effective if both strategies were used in combination. One of the positives of the concept map is that it does not provide learning by means of disconnected facts, but rather as a cohesive narrative placing emphasis on the connections between the concepts. However, most studies state that merely studying a concept map (supplantive use) is not sufficient, and that the activity of constructing the concept map (generative use) is essential for using it as a learning tool. \citeA{canas} even state that meaningful learning does not work by memorising a concept map, because the information is not integrated with other relevant knowledge. Furthermore, \citeA{nesbit} state that much of the benefits may be due to greater learner engagement rather than the properties of the concept map as an information medium. However, no studies were found testing these hypotheses, and yet \citeA{blankenship} have found that expert generated concept maps are believed to help students form conceptual understanding. Still, this study did also indicate that greater maps (more than 20 nodes) used within textbooks lead to \emph{map-shock}: ``a type of cognitive overload that prevents students from effectively processing the concept map, thereby inhibiting their ability to learn from it'' \cite[p.~3]{moore}. Finally, \citeA{eppler} enlists some of the main advantages and disadvantages in comparison to other visualisation formats (mind maps, conceptual diagrams, and visual metaphors). A positive aspect is that students can gain information rapidly, because of the systematic, proven approach to provide an overview and the emphasis on relationships and connections among concepts. On the other hand, the technique of concept mapping is not easy to apply by novices and requires exstensive training, since otherwise the maps tend to turn out to be idiosyncratic. Furthermore, although better understandability is provided, the overall pattern does not necessarily assist memorability. Finally, the quality of concept maps can be assessed through evaluation rules, however this turns out to be quite a time consuming task for the tutors.


\subsection{Applications of concept mapping}

An article by \citeA{desimone} states that despite the effectiveness of concept mapping, its use is not that widespread because students find it cognitively difficult, time consuming, or nonessential vis-\`{a}-vis task demands. The article then provides an overview of how concept maps are generally used in the classroom: as an external scratch pad to represent major ideas and their organisation, as a time-efficient tool for mental construction, and as a tool for exchange of diversifying ideas and gaining new insights; and provides benefits and limitations for each of these uses. When used as an external scratch pad, students map their ideas on paper by writing a main idea and linking it with other related concepts through action words and arrows. Although most students find it helpful to offload information externally and detect and correct gaps and inconsistencies in their knowledge, they still find the process of mapping to be time consuming. This is because they often have to make major revisions, requiring them to redraw the concept map multiple times. Therefore, a more time-efficient approach might be mental concept mapping, where they had to represent answers within the map to questions such as ``what are the key ideas?'' and ``how are these ideas related?''. This provided to be more efficient due to better mastery of the mapping strategies, and thereby more comfortable for the students. Finally, concept mapping enables students to draw relationships more freely, due to its flexibilities regarding layout and adding or removing concepts or relations. It also stimulated collaborative learning by enabling easier sharing and even co-construction. Nonetheless, of these strategies, the traditional strategy remains the most prevalent, since it is the best known use of concept mapping. Finally, as already stated before, \citeA{moore} state that multiple textbook publishers started including concept maps within their textbooks in order to provide an overview of the content.

\section{Flashcard system}

In contrast to concept maps, a flashcard system is not intended for meaningful knowledge encoding, but rather for the rehearsal of knowledge so that it keeps active and as such is prevented from being forgotten.

\subsection{Definition}

In the context of language learning, \citeA{nakata} defines flashcard systems as learning tools in which ``target items are presented outside meaning-focused tasks, and learners are asked to associate the L2 [foreign language] word form with its meaning, usually in the form of a first language translation, L2 synonym, or L2 definition'' (p. 17). This form of learning is also referred to as a \emph{paired-associate format}, which refers to learning by being presented by cues and the learner having to recall an associated counterpart. Besides vocabulary learning, it can also be used to memorise word definitions or topographical information. In order to be more inclusive of other use cases, the following general definition is proposed:

\begin{definition}
    A flashcard system refers to any system in which a learner is presented with cues and has to recall their counterparts from a paired-associate format.
\end{definition}

The most simple form of a flashcard system is a system where the learner has a stack of cards, with each containing a retrieval cue on one side and the correct associated response on the other side. A learning session then consists of going through the whole stack each day and trying to come up with correct answers. Efficiency can then be increased by repeating difficult cards more often, or skipping reviewing certain easy cards for multiple days. This way only on the pairs which are more needy of retrieval are focused on. Finally, the size of the stack of cards can be increased over multiple days in order to improve the spreading of cognitive load. Next to these paper flashcards, there is also a multitude of digital flashcard systems available \cite{hwang2, nakata, microlearning}, which allows for automating the rescheduling of flashcards, providing better access to more advanced algorithms for the rescheduling of flashcards.

\subsection{Effectiveness}

Flashcard systems have not been completely free from criticism by other researchers. \citeA{hulstijn} for example describes flashcards as a relic of the old-fashioned behaviourist learning model, and \citeA{mccullough} states that the main emphasis of flashcards is memorisation, not comprehension. However, \citeA{zirkle} states that it is still important for teachers and students to understand and utilise memory in such a way that a store of knowledge is produced that remains flexibly retrievable in a variety of contexts over a period of time, even more so because even though it is deemed useless to learn without comprehension, students still should learn by heart many conventional facts \cite{glaserfield}. Flashcards have been found to be both a time efficient tool for learning large numbers of facts and an effective tool for these facts to be more resistant to decay in comparison to traditional teaching methods \cite{nakata}. Their effectiveness also has been demonstrated accross studies in different contexts, for example that of language learning \cite{chien, macquarrie, mccullough, nakata}, word recognition \cite{joseph}, psychology courses \cite{burgess, golding}, and geography \cite{zirkle}. Therefore, many authors support pursuing research into flashcards and its effective application into classrooms.

\subsection{Design features}

\citeA{nakata} also describes general design features of flashcard software, which are seperated in terms of creation and editing of flashcards, and learning of flashcards. Examples are whether learners are able to create their own flashcards or flashcard sets, whether learners merely have to recall an answer or have to produce an answer, how big a learning session is and how repetitions are scheduled. Partly, these features are also applicable on paper flashcards. The features will be further elaborated later on page INSERT REFERENCE TO DESIGN CHAPTER, but for now it is sufficient to state that at the time of writing there are no commonly accepted guidelines for how flaschard software should be designed. This mainly is due to the fact that not a lot of research is conducted on specific design-features, because of research reviewing mostly the same program, and there being discrepancies in the way they are designed. Therefore, further research is necessary in order to establish these guidelines.

%TODO: add reference to design chapter

\subsection{Application of flashcards}

\label{subsec:fcapplication}

Multiple sources describe an increase in the use of flashcards in education: \citeA{kornell} states that ``perhaps no memorisation technique is more widely used than flashcards'' (p. 125), and more recently textbooks have also started making them available \cite{burgess, golding}. Two reasons for the popularity of flashcards are provided by \citeA{golding}: students can generate flashcards for themselves, they feel that they are `doing' something when they study. Most of the studies found are based around flashcard usage in language courses \cite{nakata, joseph, chien}, but there also exists a study by \citeA{golding} describing that 70\% of general psychology students used flashcards for at least one exam.

\citeA{chien} and \citeA{nakata} describe that multimedia and digital flashcards are used widely within vocabulary learning, because they can be easily programmed to keep track of performance and better control the sequency, which is cumbersome if done manually. Furthermore, students might be more motivated using digital flashcards because of the enhanced presentation of materials due to their multimedia capabilities. However, \citeA{golding} still found the majority of students using written flashcards. These findings surprised \citeA{burgess}, since many students have their smart phones with them most of the time -- 75\% of students report using smartphones during breaks, meetings etc, 55\% while waiting, and 45\% for school related uses -- and phones are more portable than large stacks of traditional flashcards. However, when he pursued the study by providing students with either written or digital flashcards, students used the digital flashcards less frequently than the traditional flashcards, even when the students had to make their own flashcards. Reasons students provided were technical issues such as battery consumption, simply forgetting about it, using entertainment apps instead of studying, and preference for traditional flashcards.

\section{Comparison of the two tools}

In summary, most studies describe concept mapping as a tool for meaningful encoding, whereas flashcards are described as a tool for rote memorisation, and therefore imply that the former approach leads to more comprehension than the latter. A recent study by \citeA{karpicke2} researched this hypothesis by having participants study a science text with four different learning conditions and prompting them afterwards with verbatim and inference questions and metacognitive predictions. Within the first condition, students only had to read the text and then answer the quesions. The second group studied the text in four consecutive study periods. Students within the third group studied the text in one initial study period and then created a concept map after being instructed in concept mapping. The final group studied the text in an initial study period and then had to recall as much as they could on a free recall test, and repeated this strategy. The time spent on concept mapping and recalling was equal. When analysing the results, it was found that the retrieval practice group performed highest on both the verbatim and the inference questions, whereas the repeated study and concept mapping groups performed about equally well and the study once group performed the worst. Interestingly enough, the retrieval practice group judged their own learning the lowest, and the repeated study group the highest. The same effect of concept mapping and retrieval practice was found again in a second reproduction study, and also in another study by \citeA{burdo}. It is theorised that during elaboration, subjects attain detailed representations of encoded knowledge by linking concepts together in meaningful ways, but that during retrieval, subjects use retrieval cues to reconstruct meaning en thereby already organise the content in a meaningful way. \citeA{karpicke2} conclude that these insights could pave the way for the design of new educational activities with retrieval practices in mind.

\section{Flashmap system}

\label{sec:intro_flashmap}

It can be concluded that both of these tools are helpful for studying, since concept maps help students organise by drawing hierarchical links and elaborate on the content by drawing cross-links, and flashcards help students monitor their understanding of the text and retain the knowledge in order to facilitate integration with a following topic where the knowledge may prove relevant. The object of this study is therefore to create a new learning tool, and intends to combine both the visual overview of concept maps with the retrieval mechanism of flashcard systems by means of a new digital tool, which from this point onwards will be referred to as the Flashmap system. It will present incomplete parts of a concept map, in which the student has to fill in the missing parts of propositions represented by that map (see figure~\ref{fig:flashmap}). These parts will consecutively be repeated according to algorithms already used by digital flashcard systems. The flashmap system might have the potential to bridge the gap between the two systems, and therefore make meaningful and effective rote memorisation possible, for it should make the relations between the concepts explicit to the student, thereby increasing the organisation of the knowledge and reducing the segregation of facts. Hereby, this tool might facilitate the needs stressed by both \citeA{karpicke4} and \citeA{zirkle} of more meaningful retrieval. Furthermore, by having the students memorise the concept map and graduately expanding on it, the generally experienced map shock occuring with expert-generated concept maps might also be mitigated (see also \citeNP{tzeng}).

\begin{figure}
    \centering
    \includegraphics[width=\textwidth]{img/flashmap}
    \caption{A display of the flashmap system, where the user has to think of the concepts fitting in the orange nodes on the left, and has to indicate which nodes were correct on the right}
    \label{fig:flashmap}
\end{figure}

\section{Evaluation}

\label{sec:intro_evaluation}

This project does not only aim to develop a flashmap system, but also to evaluate it by comparing it to a similarly functioning flashcard system. For evaluating this flashmap system, a group of Dutch high school teachers of the Stedelijk Lyceum has been found willing to participate, with their students using either the flashmap or the flashcard system for self study parallel with classroom instruction. The content of the instruction will be the history of Dutch literature during the sixteenth and seventeenth century. For example, the students have to learn what the influence is of the Dutch War of Independence on the \emph{Spaanschen Brabander} by Bredero. Because of the content existing mainly of concepts with meaningful relations it fits to the concept map technique and thereby the flashmap system could be significantly beneficial over the flashcard system.

\newcounter{researchquestion}
\renewcommand{\theresearchquestion}{\Roman{researchquestion}}
\newcounter{subquestion}[researchquestion]
\renewcommand{\thesubquestion}{\alph{subquestion}}

The research aims to investigate the following questions: Regarding high school students learning for Dutch literature using the flashmap system in comparison to them using the flashcard system...

\refstepcounter{researchquestion}\label{benefit}
\refstepcounter{subquestion}\label{effectiveness}
\Roman{researchquestion}\alph{subquestion}. ...is the learning gain larger?

\refstepcounter{subquestion}\label{efficiency}
\Roman{researchquestion}\alph{subquestion}. ...is the learning gain larger controlled for the time spend with the system?

\refstepcounter{researchquestion}\label{perception}
\refstepcounter{subquestion}\label{usefulness}
\Roman{researchquestion}\alph{subquestion}. ...do they perceive the system to be more useful?

\refstepcounter{subquestion}\label{ease}
\Roman{researchquestion}\alph{subquestion}. ...do they perceive the system to be easier to use?

\refstepcounter{researchquestion}\label{howused}
\Roman{researchquestion} How did the students use the flashmap or flashcard system?

For researching the effects of the flashmap system relative to the effects of the flashcard system, it is important to consider two main factors: its actual benefits (research question~\ref{benefit}\ref{effectiveness} and~\ref{efficiency}), and its perceived benefits (research question~\ref{perception}\ref{usefulness} and~\ref{ease}). Furthermore, for the validity of the system and of the experiment it is important to investigate how the system was used by the students (research question~\ref{howused}).

To research whether the flashmap system is more effective or efficient than the flashcard system, the learning gain of high school the students will be measured, referring to the knowledge obtained by a student over the course of an instruction. Sequentially, the efficiency of the system is determined by the learning gain controlled for time spend on the system.

For measuring the affectiveness of the systems, the Technology Acceptance Model by \citeA{tam} will be used (see figure~\ref{fig:tam}). This model predicts the use of an information system by measuring the Perceived Usefulness and the Perceived Ease of Use of the user. These variables are mediators between External Variables and Attitude toward using, leading to Behavioural intention to use, which in turn leads to the Actual system use.

Finally, for the answering the final question an interview will be conducted with a sample of the participants, and by the server logging usage information about the user.

\begin{figure}
    \centering
    \includegraphics[width=\textwidth]{img/tam}
    \caption{The Technology Acceptance Model by \protect\citeA{tam}}
    \label{fig:tam}
\end{figure}

Answering the research questions has both practical and scientific relevance. From a practical perspective, it has potential to overcome the criticism from various authors about flashcard systems and answer the need for meaningful rote memorisation. From a scientific perspective, it could confirm the hypothesis by \citeA{tzeng} that an expanding concept map might mitigate map shock. It also makes way for new research opportunities, for example what the effect is of integrating the flashmap with the games condition formulated by \citeA{canas}. 

The following two chapters within the introduction will elaborate further on the needs for memorisation on page~\pageref{ch:context} and the cognitive theories underlying concept mapping and flaschard systems on page~\pageref{ch:theory}, after which the design and development of the flashmap will be described in part II. After that, the research conducted within this project and its results will be described in part III, and finally part IV will be elaborating on additional features of the flashmap system and how these could be evaluated by further research.
