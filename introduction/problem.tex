\chapter{Project Description}

\label{ch:problem}

Over the centuries, knowledge has been fundamental to any learning process. Socrates already stated that knowledge is the only true virtue, and the tragedian Aeschylus regarded memory as the mother of all knowledge. Moreover, it was not only regarded as important by ancient thinkers, but is still regarded as such by modern scholars on education. Both the taxonomy of learning by \citeA{bloom} as a revision of this taxonomy by \citeA{krathwohl}, as well as the three stages of skill acquisition by \citeA{skillacquisition}, propose that all learning should start with memorising factual knowledge. Furthermore, \citeA{glaserfield}, one of the main founders for critical constructivism, expresses a need for training students so that they permanently possess facts and are able to repeat them flawlessly whenever they are needed, while also understanding what is placed into their memory. \citeA{ltwm} adds to this by stating that in order to perform complex tasks, people must maintain access to large amounts of information, and that solely encoding knowledge is not sufficient. Despite all of this, \citeA{karpicke4} argues that "[r]etrieval processes, the processes involved in using available cues to actively reconstruct knowledge, have received less attention" (p. 158), whereas basic research on learning and memory has emphasised that retrieval must be considered in any analysis of learning.

Traditionally, when students have to gain complex and meaningful knowledge -- for example knowledge about a historical event or a chapter in a psychology textbook, they are asked to read the relevant chapter from a provided textbook. However, \citeA{learninginstruction} states that many students have difficulty gaining knowledge in this manner. He breaks reading for comprehension down into four separate skills, which are integrating, organising, elaborating, and monitoring. Integrating refers to relating a text to one's prior knowledge, for which exists evidence that rich background knowledge leads to better inferences about the text, and thereby better comprehension. After integration, the reader has to organise the text, so that the important ideas and the relatinoships among them are identified. This is mainly a problem for less experienced readers, which will spend too much time on reading the unimportant information. At the same time, the student has to make necessary inferences while reading, or has to elaborate, which is quite difficult for readers when not prompted. Finally, students have to monitor their comprehension, which refers to evaluating comprehension of the text and if necessary adjusting the reading strategy. This is again quite difficult for the average reader, however can be trained.

While intergrating is something more dependent on the curriculum design, organising and elaborating can be facilitated by concept mapping, and monitoring by flashcard systems. Furthermore, this research aims to develop a tool which combines these tools, called the flashmap. In this chapter, concept mapping, flashcard systems, and the flashmap will be explored on a practical level in order to establish their definitions together with a summary of arguments in favour or opposition of using them as tools for studying textual material.

\section{Concept mapping}

Concept maps are a learning tool deviced by Joseph Novak in 1970's, based on the notions of the constructivist theories. It was originally intended for assessing the structure of student conceptions, before and after instruction, in order to map their prior knowledge and compare it to what they learned during the instruction. This expanded on the notions of \citeA{ausubel}, who stated that what the learner already knows is most important, and that this had to be ascertained before teaching. Although the use of concept maps as an assessment tool remains prevalent \cite{canas, chung, hwang2, ruiz1}, over time, students begun to use it as a tool to comprehend textual material by organising and elaborating on the included concepts \cite{canas, eppler, hwang2, karpicke2, nesbit2}.

\subsection{Definition}

One definition provided by \citeA{burdo} states that "concept maps are hierarchical representations of knowledge. Construction of them involves linking concepts [...] through the use of linking phrases into propositional statements" (p. 335). The concepts are typically nouns or verbs with or without modifying adjectives or adverbs, and linking phrases specify the relationship between two concepts. \citeA{ruiz1} also mention elements of this definition in their own definition, however \citeA{canas} and \citeA{eppler} include a few extra features, such as that the concepts are ordered in hierarchical fashion. They describe two different kinds of links, hierarchical links to indicate ranking between the concepts, and crosslinks to indicate relationships between concepts in different segments or domains of the concpt map. The latter would help to see how a concept in one domain of knowledge represented on the map is related to a concept in another part of the knowledge producer, enabling better connections to prior knowledge of the user. According to \citeA{eppler}, concept maps are always top-down and shows systematic relationships among sub-concepts relating to one main concept, however \citeA{canas} state that they can also be cyclical as long as the concepts still have a conceptual hierarchy. Finally, most researchers agree that the links between concepts are typically directed. In conclusion, the definition within this thesis will be:

\begin{definition}
    A concept map refers to a directed graph, in which the nodes consist of concepts and the edges of either hierarchical or cross-links, labeled with linking phrases, forming several propositional statements in a knowledge domain.
\end{definition}

\noindent An example of a concept map is displayed in figure~\ref{fig:examplemap}.

In this study, the more interesting aspects of concept maps are the use of concept mapping for elaborating and of demonstrating meaningful relationships between concepts to learners. The first use of the concept map is known as generative use, and the second as supplantive \cite{instructionaldesign}.

\begin{figure}
    \centering
    \includegraphics[width=\textwidth]{img/conceptmap.png}
    \caption{A fraction of the concept map used in this study}
    \label{fig:examplemap}
\end{figure}

\subsection{Effectiveness}

Multiple studies, both qualitative and quantitative, have demonstrated that concept maps can promote meaningful learning \cite{canas, hwang2, nesbit2, subramaniam}. When comparing the concept mapping strategy with traditional teaching strategies (in a study conducted within the context of tertiary chemistry), \citeA{singh} found that the concept map teaching strategy was more effective, however that it was most effective if both strategies were used in combination. One of the positives of the concept map is that it does not provide learning by means of disconnected facts, but rather as a cohesive narrative placing emphasis on the connections between the concepts. However, most studies state that seeing a concept map is not sufficient, and that the activity of mapping is essential for using it as a learning tool. \citeA{canas} even states that meaningful learning does not work by memorising a concept map, because the information is not integrated with other relevant knowledge. Furthermore, \citeA{nesbit2} states that much of the benefits may be due to greater learner engagement rather than the properties of the concept map as an information medium. However, no studies were found testing these hypotheses, and yet \citeA{blankenship} have found that expert generated concept maps are believed to help students form conceptual understanding. This study did indicate that greater maps (more than 20 nodes) used within textbooks lead to \emph{map-shock}: ``a type of cognitive overlaod that prevents students from effectively processing the concept map, thereby inhibiting their ability to learn from it'' \cite[p.~3]{moore}. Finally, \citeA{eppler} enlists some of the main advantages and disadvantages in comparison to other visualisation formats (mind maps, conceptual diagrams, and visual metaphors):

\begin{itemize}
    \item Advantages:
        \begin{itemize}
            \item Rapid information provision
            \item Systematic, proven approach to provide overview
            \item Emphasizes relationships and connections among concepts
            \item Ability to assess quality of concept map through evaluation rules
        \end{itemize}
    \item Disadvantages:
        \begin{itemize}
            \item Not easy to apply by novices; requires exstensive training
            \item Concept maps tend to be idiosyncratic
            \item Time consuming evaluation through tutors
            \item The overall pattern does not necessarily assist memorability
        \end{itemize}
\end{itemize}

\subsection{Applications of concept mapping}

\citeA{desimone} provides an overview of how concept maps are generally used in the classroom: as an external scratch pad to represent major ideas and their organisation, as a time-efficient tool for mental construction, and as a tool for exchange of diversifying ideas and gaining new insights, and provides benefits and limitations for each of these uses. In general, she states that despite the effectiveness of concept mapping, its use is not that widespread because students find it cognitively difficult, time consuming, or nonessential vis-\`{a}-vis task demands. When used as an external scratch pad, students map their ideas on paper by writing a main idea and linking it with other related concept through action words and arrows. Although most students find the aspect of offloading information externally helpful, plus the benefit of detecting and correcting gaps and inconsistencies in their knowledge, they still find it to be time consuming. This is because they have to make major revisions, requiring them to redraw the concept map multiple times. Therefore, a more time-efficient approach might be mental concept mapping, where they had to represent answers within the map to questions such as ``what are the key ideas?'' and ``how are these ideas related?''. This provided to be more efficient due to better mastery of the mapping strategies, and thereby more comfortable for the students. Finally, concept mapping enables students to draw relationships more freely, due to its flexibilities regarding layout and adding or removing concepts or relations. It also stimulated collaborative learning by enabling easier sharing and even co-construction. Nonetheless, of these strategies, the traditional strategy remains the most prevalent, since it is the best known use of concept mapping. Finally, as already stated before, \citeA{moore} state that multiple textbook publishers started including concept maps within their textbooks in order to provide an overview of the content.

\section{Flashcard system}

A flashcard system serves not for meaningful knowledge encoding, but rather for the rehearsal of knowledge so that it keeps active and as such is prevented from being forgotten.

\subsection{Definition}

In the context of language learning, \citeA{nakata} define flashcard systems where ``target items are presented outside meaning-focused tasks, and learners are asked to associate the L2 [foreign language] word form with its meaning, usually in the form of a first language translation, L2 synonym, or L2 definition'' (p. 17). This form of learning is also refered to as a \emph{paired-associate format}, which refers to learning by being presented by cues and the learner having to recall an associated counterpart. Therefore, the following general definition is proposed:

\begin{definition}
    A flashcard system refers to any system in which a learner is presented by cues and has to recall answers from an paired-associate format.
\end{definition}

The most simple form of a flashcard system is a system where the learner has a stack of cards, with each containing a retrieval cue on one side and the correct associated response on the other side. He can then study by going through the whole stack each day, trying to come up with the correct answer. He can then increase its efficiency by repeating difficult cards more often, or skipping reviewing certain easy cards for multiple days. This way he focuses only on the pairs which are more needy of retrieval. Finally, he can choose to not start with the whole stack on the first day, but increase the size of the stack over multiple days in order to improve the spreading of cognitive load. Next to these paper flashcards there is also a multitude of digital flashcard systems available \cite{hwang2, nakata, microlearning}, which allows for automating the rescheduling of flashcards, making it possible to create more advanced mechanisms.

\subsection{Effectiveness}

Flashcard systems have not been completely free from criticism by other researchers. \citeA{hulstijn} for example describes flashcards as a relic of the old-fashioned behaviourist learning model, and \citeA{mccullough} states that the main emphasis of flashcards is memorisation, not comprehension. However, \citeA{zirkle} states that it is still important for teachers and students to understand and utilise memory in such a way that a store of knowledge is produced that remains flexibly retrievable in a variety of contexts over a period of time, even more so because even though it is deemed useless to learn without comprehension, students still should learn by heart many conventional facts \cite{glaserfield}. Flashcards have found to be both a time efficient tool for learning large numbers of facts and an effective tool for these facts to be more resistant to decay in comparison to traditional teaching methods \cite{nakata}. They have also been found to be effective accross studies in different contexts, for example that of language learning \cite{chien, macquarrie, mccullough, nakata}, word recognition \cite{joseph}, psychology courses \cite{burgess, golding}, and geography \cite{zirkle}. Therefore, many authors support pursuing research into flashcards and its effective application into classrooms.

\subsection{Design features}

\citeA{nakata} also describe general design features of flashcard software, which are seperated in terms of creation and editing of flashcards, and learning of flashcards. These entail features whether learners are able to create their own flashcards or flashcard sets, whether learners merely have to recall an answer or have to produce an answer, how big a learning session is and how repetitions are scheduled. Partly, these features are also applicable on paper flashcards. The features will be further elaborated later on page INSERT REFERENCE TO DESIGN CHAPTER, but for now it is sufficient to state that there do not exist commonly accepted guidelines for how flaschard software should be designed. This mainly is due to the fact that not a lot of research is conducted on specific design-features, research reviewing mostly the same program, and there being discrepancies in the way they are designed, and further research is needed in order to establish these guidelines.

%TODO: add reference to design chapter

\subsection{Application of flashcards}

Multiple sources describe an increase in the use of flashcards in education: \citeA{kornell} states that ``perhaps no memorisation technique is more widely used than flashcards'' (p. 125), and more recently textbooks have also started making them available \cite{burgess, golding}. \citeA{golding} provides two reasons for the popularity of flashcards: students can generate flashcards for themselves, they feel taht they are `doing' something when they study. Most of the studies found is based around flashcard usage in language courses \cite{nakata, joseph, chien}, but there also exists a study by \citeA{golding} describing that 70\% of general psychology students used flashcards for at least one exam.

\citeA{chien} and \citeA{nakata} describe that multimedia and digital flashcards are used very widely, because they can be easily programmed to keep track of performance and better control the sequency, which is cumbersome if done manually, and that students might be more motivated using digital flashcards because of the enhanced presentation of materials due to their multimedia capabilities. However, \citeA{golding} still found the majority of students using written flashcards. These findings surprised \citeA{burgess}, since many students have their smart phones with them most of the time -- 75\% of students report using smartphones during breaks, meetings etc, 55\% while waiting, and 45\% for school related uses -- and phones are more portable than large stacks of traditional flashcards. However, when he pursued the study by providing students with either written or digital flashcards, students used the digital flashcards less frequently than the traditional flashcards, even when the students had to make their own flashcards. Reasons students provided were technical issues such as battery consumption, simply forgetting about it, using entertainment apps instead of studying, and preference for traditional flashcards.


\section{Comparison of the two tools}

In summary, most studies describe concept mapping as a tool for meaningful encoding, whereas flashcards are described as a tool for rote memorisation, and therefore imply that the former approach leads to more comprehension than the latter. A recent study by \citeA{karpicke2} researched this hypothesis by having participants study a science text with four different learning conditions and prompting them afterwards with verbatim and inference questions and metacognitive predictions. Within the first condition, students only had to read the text and then answer the quesions. The second group studied the text in four consecutive study periods. Students within the third group studied the text in one initial study period and then created a concept map after being instructed in concept mapping. The final group studied the text in an initial study period and then had to recall as much as they could on a free recall test, and repeated this strategy. The time spent on concept mapping and recalling was equal. When analysing the results, it was found that the retrieval practice group performed highest on both the verbatim and the inference questions, whereas the repeated study and concept mapping groups performed about equally well and the study once group performed the worst. Interestingly enough, the retrieval practice group judged their own learning the lowest, and the repeated study group the highest. The same effect of concept mapping and retrieval practice was found again in a second reproduction study, and also in another study by \citeA{burdo}. It is theorised that during elaboration, subjects attain detailed representations of encoded knowledge by linking concepts together in meaningful ways, but that during retrieval, subjects use retrieval cues to reconstruct meaning en thereby already organise the content in a meaningful way. \citeA{karpicke2} concludes that this could pave the way for the design of new educational activities with retrieval practices in mind.

\section{Flashmap}

%%%%%%%%%%%%%%%%%%%%%%%%%%%% FROM THE RESEARCH PROPOSAL (TODO: DELETE LATER) %%%%%%%%%%%%%%%%%%%%%%%%%%%%%%%%%%%%%%%%%%%%%%%%%%%%%

%In the context of language learning, \citeA{nakata} defines a flashcard system as ``Target items [...] presented outside meaning-focused tasks, [where] learners are asked to associate the L2 [foreign] word form with its meaning, usually in the form of a first language translation, L2 synonym, or L2 definition'' (p. 17). In a generalised form, a flashcard system contains pairs by association, where the student is presented by one member of the pair and has to recall the other. This pair can indeed consist out of words in different languages, but can also consist out of a picture and a word for learning spelling, or a question with an answer for learning history. Traditionally, these pairs were presented as small cards (hence the name), and students would spend each day going through the stack of cards, preparing themselves for their exams.

%\section{Problem statement}

%n1.1.7, n1.1.3

%Both the taxonomy of learning by \citeA{bloom} as a revision of this taxonomy by \citeA{krathwohl}, as well as the three stages of skill acquisition by \citeA{skillacquisition}, propose that all learning should start with memorising factual knowledge. Furthermore, \citeA{glaserfield}, one of the main founders for critical constructivism, expresses a need for training students so that they permanently possess facts and are able to repeat them flawlessly whenever they are needed, while also understanding what is placed into their memory. \citeA{ltwm} adds to this by stating that in order to perform complex tasks, people must maintain access to large amounts of information, and that solely encoding knowledge is not sufficient. Despite this, \citeA{karpicke4} argues that "[r]etrieval processes, the processes involved in using available cues to actively reconstruct knowledge, have received less attention" (p. 158), whereas basic research on learning and memory has emphasised that retrieval must be considered in any analysis of learning. Therefore, this project aims to research a tool for meaningfully enhancing the retrieval process. 

%\citeA{karpicke4} also states that meaningful learning often is defined in contrast to rote learning, and that active retrieval is thought of as an example of the latter leading to poorly organised knowledge that lacks coherence and integration. However, in another study they found active retrieval to enhance learning of meaningful educational materials and that these effects are long-lasting, not short-lived \cite{karpicke2}. In this study, he compared the effects of active retrieval using measures of meaningful learning contrasting to a popular learning strategy known as concept mapping. The latter involves a graph consisting of nodes representing concepts and labeled lines denoting the relation between a pair of nodes \cite{ruiz1} (see figure~\ref{fig:conceptmap}). Multiple researchers have found by means of both qualitative and quantitative studies that concept maps can promote meaningful learning leading to positive effects on students \cite{hwang2, subramaniam, canas}. This has been demonstrated in comparison to activities such as reading text passages, attending lectures, and participating in class discussions \cite{singh, nesbit2}. \citeA{canas} describes the process of concept mapping as the only effective way of using the concept map, which refers to students constructing their own concept maps. This is why the concept map is generally viewed as a tool in alignment with the constructivist perspective. Because of this, the concept map might seem as a solution to the need asked by \citeA{glaserfield} and his peers. However, the aforementioned article by \citeA{karpicke2} reveals that retrieval practices produced better performance than elaborative concept mapping for meaningful learning.

%\begin{figure}
%    \centering
%    \includegraphics[width=\textwidth]{img/conceptmap}
%    \caption{An example of a concept map}
%    \label{fig:conceptmap}
%\end{figure}

%One of the currently existing methods for efficiently rote memorising information is the flashcard system, which entails studying declarative knowledge using active retrieval in a so-called paired-associate format. Within this format, learners are asked to associate terms with other terms outside meaning-focused tasks \cite{nakata}, for example by associating a definition with a presented concept. With flashcards, large numbers of words can be memorised in a very short time, and are more resistant to decay \cite{nakata, joseph}. Furthermore, when evaluating flashcards in a psychology setting, it was found that students who use flashcards have a significantly higher final average than those who do not \cite{burgess, golding}.

%EXPLAIN THE PROBLEM

%n1.1.2, n1.1.3.2, n1.1.3.8, n1.5.9, n1.2.1.2
%Per contra, not all research favours using flashcards for textual comprehension. \citeA{zirkle} and \citeA{mccullough} state that flashcards are especially useful for learning declarative knowledge but not for textual comprehension. \citeA{zirkle} points out the overemphasis placed upon the rote memorisation of disconnected facts, whereas whatever it is that students are to place into memory they should, more importantly, understand. Furthermore, \citeA{hulstijn} describes flashcards as a relic of the old-fashioned behaviourist learning model, and states that we have to look for more modern constructivist models.

%EXPLAIN WHY THE PROBLEM IS IMPORTANT

%n1.1.1.7
%Solving these problems could lead to better utilisation by teachers and students of producing a store of knowledge that remains flexibly retrievable, in contrast to only segregated paired associations which depend on specific cues in order to be retrieved. Furthermore, using computer-based flashcards have been used very widely \cite{nakata,burgess, golding,kornell}, and improving currently existing flashcards could reach a wide audience of future users of flashcard systems.

%PROPOSE A SOLUTION/IDEA AND ITS BENEFITS

%introduction flashmaps

%n1.2.6.8 and n1.2.6.9 (Counterarguments), n1.2.5, n1.2.10
%Therefore, another solution might be the development of a new tool, which will from henceforth be referred to as the flashmap system. The intention behind the flashmap system is to combine the paired associate mechanism of the flashcard system with the visual representation of the concept map, and is a new tool designed and developed for this research project. This tool might have the potential to bridge the gap between the two systems and therefore make meaningful and effective rote memorisation possible, for it makes the relations between the concepts explicit to the student and thereby increasing the organisation of the knowledge and reducing the segregation of facts. Thereby, it might provide a solution for the problems by \citeA{zirkle} described before.
