\chapter{Project Description}

In this chapter, flashcard systems, concept mapping, and the flashmap will be explored on a practical level in order to establish their definitions together with a summary of arguments in favour or opposition of using them as tools for studying.

\section{Flashcard system}

%Short historical introduction

%Definition

%Summary of arguments in favour
    %Reference to context chapter
    %Reference to theoretical framework
    %Practical studies

\section{Concept mapping}

%Short historical introduction

%Definition

%Summary of arguments in favour
    %Reference to context chapter
    %Reference to theoretical framework
    %Practical studies

\section{Comparison of the two tools}

%Criticism flashcard systems
%Criticism concept mapping

%Study karpicke4

%Critical constructivism

\section{Flashmap}

%%%%%%%%%%%%%%%%%%%%%%%%%%%% FROM THE RESEARCH PROPOSAL (TODO: DELETE LATER) %%%%%%%%%%%%%%%%%%%%%%%%%%%%%%%%%%%%%%%%%%%%%%%%%%%%%

%In the context of language learning, \citeA{nakata} defines a flashcard system as ``Target items [...] presented outside meaning-focused tasks, [where] learners are asked to associate the L2 [foreign] word form with its meaning, usually in the form of a first language translation, L2 synonym, or L2 definition'' (p. 17). In a generalised form, a flashcard system contains pairs by association, where the student is presented by one member of the pair and has to recall the other. This pair can indeed consist out of words in different languages, but can also consist out of a picture and a word for learning spelling, or a question with an answer for learning history. Traditionally, these pairs were presented as small cards (hence the name), and students would spend each day going through the stack of cards, preparing themselves for their exams.

%\section{Problem statement}

%n1.1.7, n1.1.3

%Both the taxonomy of learning by \citeA{bloom} as a revision of this taxonomy by \citeA{krathwohl}, as well as the three stages of skill acquisition by \citeA{skillacquisition}, propose that all learning should start with memorising factual knowledge. Furthermore, \citeA{glaserfield}, one of the main founders for critical constructivism, expresses a need for training students so that they permanently possess facts and are able to repeat them flawlessly whenever they are needed, while also understanding what is placed into their memory. \citeA{ltwm} adds to this by stating that in order to perform complex tasks, people must maintain access to large amounts of information, and that solely encoding knowledge is not sufficient. Despite this, \citeA{karpicke4} argues that "[r]etrieval processes, the processes involved in using available cues to actively reconstruct knowledge, have received less attention" (p. 158), whereas basic research on learning and memory has emphasised that retrieval must be considered in any analysis of learning. Therefore, this project aims to research a tool for meaningfully enhancing the retrieval process. 

%\citeA{karpicke4} also states that meaningful learning often is defined in contrast to rote learning, and that active retrieval is thought of as an example of the latter leading to poorly organised knowledge that lacks coherence and integration. However, in another study they found active retrieval to enhance learning of meaningful educational materials and that these effects are long-lasting, not short-lived \cite{karpicke2}. In this study, he compared the effects of active retrieval using measures of meaningful learning contrasting to a popular learning strategy known as concept mapping. The latter involves a graph consisting of nodes representing concepts and labeled lines denoting the relation between a pair of nodes \cite{ruiz1} (see figure~\ref{fig:conceptmap}). Multiple researchers have found by means of both qualitative and quantitative studies that concept maps can promote meaningful learning leading to positive effects on students \cite{hwang2, subramaniam, canas}. This has been demonstrated in comparison to activities such as reading text passages, attending lectures, and participating in class discussions \cite{singh, nesbit2}. \citeA{canas} describes the process of concept mapping as the only effective way of using the concept map, which refers to students constructing their own concept maps. This is why the concept map is generally viewed as a tool in alignment with the constructivist perspective. Because of this, the concept map might seem as a solution to the need asked by \citeA{glaserfield} and his peers. However, the aforementioned article by \citeA{karpicke2} reveals that retrieval practices produced better performance than elaborative concept mapping for meaningful learning.

%\begin{figure}
%    \centering
%    \includegraphics[width=\textwidth]{img/conceptmap}
%    \caption{An example of a concept map}
%    \label{fig:conceptmap}
%\end{figure}

%One of the currently existing methods for efficiently rote memorising information is the flashcard system, which entails studying declarative knowledge using active retrieval in a so-called paired-associate format. Within this format, learners are asked to associate terms with other terms outside meaning-focused tasks \cite{nakata}, for example by associating a definition with a presented concept. With flashcards, large numbers of words can be memorised in a very short time, and are more resistant to decay \cite{nakata, joseph}. Furthermore, when evaluating flashcards in a psychology setting, it was found that students who use flashcards have a significantly higher final average than those who do not \cite{burgess, golding}.

%EXPLAIN THE PROBLEM

%n1.1.2, n1.1.3.2, n1.1.3.8, n1.5.9, n1.2.1.2
%Per contra, not all research favours using flashcards for textual comprehension. \citeA{zirkle} and \citeA{mccullough} state that flashcards are especially useful for learning declarative knowledge but not for textual comprehension. \citeA{zirkle} points out the overemphasis placed upon the rote memorisation of disconnected facts, whereas whatever it is that students are to place into memory they should, more importantly, understand. Furthermore, \citeA{hulstijn} describes flashcards as a relic of the old-fashioned behaviourist learning model, and states that we have to look for more modern constructivist models.

%EXPLAIN WHY THE PROBLEM IS IMPORTANT

%n1.1.1.7
%Solving these problems could lead to better utilisation by teachers and students of producing a store of knowledge that remains flexibly retrievable, in contrast to only segregated paired associations which depend on specific cues in order to be retrieved. Furthermore, using computer-based flashcards have been used very widely \cite{nakata,burgess, golding,kornell}, and improving currently existing flashcards could reach a wide audience of future users of flashcard systems.

%PROPOSE A SOLUTION/IDEA AND ITS BENEFITS

%introduction flashmaps

%n1.2.6.8 and n1.2.6.9 (Counterarguments), n1.2.5, n1.2.10
%Therefore, another solution might be the development of a new tool, which will from henceforth be referred to as the flashmap system. The intention behind the flashmap system is to combine the paired associate mechanism of the flashcard system with the visual representation of the concept map, and is a new tool designed and developed for this research project. This tool might have the potential to bridge the gap between the two systems and therefore make meaningful and effective rote memorisation possible, for it makes the relations between the concepts explicit to the student and thereby increasing the organisation of the knowledge and reducing the segregation of facts. Thereby, it might provide a solution for the problems by \citeA{zirkle} described before.
