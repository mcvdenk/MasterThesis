\chapter{Context}

As can be read in the previous chapter, the aim of this study is to develop and evaluate a tool designed for the purpose of meaningful memorisation. However, why is it actually important to memorise? This question has historically been debated since the days of the early Greek philosophers, and still remains relevant today. Therefore, before delving into the effectivity and specifications of the tool itself, it seems important to briefly reflect on this question first. This chapter does not aim to answer this age-old question, but rather tries to provide both some philosophical and historical context, for better understanding of the relevance of a better memorisation tool, and what `better' generally entails. Furthermore, it more specifically will relate these questions specifically to the tools investigated within this study.

\section{Five educational philosophies}

Curriculum theorisers have proposed many different systems of categories \cite{curriculumtheory}, of which the aim is to investigate which goals people involved with education have, and which aspects they therefore regard as being important. \citeA{educationalphilosophy5} differentiates between the five philosophies of education \emph{Perennialism}, \emph{Essentialism}, \emph{Progressivism}, \emph{Reconstructionalism}, and \emph{Existentionalist education}, which have also (at least partly) been acknowledged by other authors \cite{educationalphilosophy, educationalphilosophy2, educationalphilosophy3, educationalphilosophy4}. Furthermore, \citeA{educationalphilosophy3} have found these categories to be sufficiently valid and reliable upon measuring their prevalence among teachers. Therefore, these categories will be discussed further individually in order to provide philosophical context towards the function of knowledge.

\section{Perennialism}

According to perennialism, there is no alteriar motive for attaining knowledge, but rather that attaining knowledge is a purpose on itself. This is along the words of Socrates, who concluded that knowledge is the only virtue. This he concluded based on that wisdom is the same as knowledge \cite{wisdomknowledge}, that wisdom is one of the five cardinal virtues, and that all other virtues (e.g. justice) are merely derived from the virtue of wisdom.

The perennialists are mainly based on either the general philosophy of idealism or of realism. The most notable idealist perennialist are the scholastics, who focused on teaching the great classical and religious works in order to better understand their supreme being. Realist perennialists believe the classic works still have much implications today, and therefore should be taught to the next generation.

Methods generally practiced by are considered to be rather traditional, example of these are memorisation, reading, writing, drill, and recitation. It is also the only philosophy which has many of its followers believing that education should be directed towards the intellectually gifted, and that other students should only receive vocational education.

Perennialism has been the leading philosophy in academics before the enlightenment. In the classical era, Greek students had to memorise and recite famous poetry, such as the Iliad and Odyssey by Homer, because these were believed to ``provide great moral lessons and taught them what it meant to be a Greek'' \cite[p.139]{searchgreeks}. This academic tradition was then perpetuated throughout the middle ages by the scholastics, who used the rationalism of the Greek philosophers to defend christian doctrine -- most notably in the \emph{Summa Theologica} by Thomas Aquinas. Scholastic instruction consisted of four elements: \emph{lectio}, the reading of an authoritative text; \emph{mediatio}, a reflection on the text; \emph{quaestio}, questions from students about the text; and \emph{disputationes}, a discussion about controversial \emph{quaestiones}. With the coming of the enlightenment, academics transisted from using classical idealism as a source of truth and instead used experimentalism as a source of learning about the material world and verifying truth claims, and humanism as a means to a better understanding of the human endevour. Nonetheless, perennialism remained a prominent philosophy in education until the industrial revolution in the 19th century, and still has a place in modern society in the form of for example the Great Book program proposed by Hutchins, albeit in a far lesser degree than before the enlightenment.

\section{Essentialism}

Essentialism is generally seen as a child philosophy of perennialism, and is more goal oriented than its parent. Its purpose is to pass on knowledge to new generations in order for them to be able to function in society, and focuses on subject matter. It is also a very teacher oriented approach to education.

This philosophy also is based on both idealism and realism, whereas the idealists think the content comes from history, language and the classics, and the realists think it comes from the physical world, including mathematics and the natural sciences.

Just like perennialism, essentialist teaching methods are rather traditional, and include returning to the three R's, reading, lectures, memorisation, repetition, audio-visual materials, and examinations.

The earliest form recognisable as essentialist is the factory model of education \cite{honours}, which was a means to deliver education to the general public for the benefit of the whole society. This model was improved upon by introducing aspects of behaviourism with the introduction of reinforcement and repetition in order to shape the behaviour the teacher wanted. Furthermore, it introduced the audio-lingual method, where the whole class as a group chanted correct answers or key phrases. Furthermore, because of the importance of high-quality instruction, cognitivism contributed towards a better understanding of how to present materials more effectively. Essentialism still remains a popular philosophy in the form of people wanting to go `back to basics' or wanting more order in the classroom.

\section{Progressivism}

Progressivism go one step further than essentialists in a sense that new students should not only be taught to function in society, but to go beyond and improve society. This might seem like a small step, but is rather involved for it has its base in opposing authorianism instead of conforming to it.

It also has its root philosophy in experimentalism, where truth is not constant such as in idealism or realism, but rather is constantly in transition to a better understanding. Therefore, a progressivist curriculum focuses itself not on teaching already existing knowledge, but rather on the methods existing to discover knowledge such as the scientific method. This does not mean however that knowledge has become irrelevant. Students still have to be brought up to date with the newest developments in their field of interest, and thereby there is still some knowledge transfer necessary. The only difference is that this knowledge is never taught to be final, and the focus still lies within the transition and the parts still unknown.

Progressivists generally use more generative methods for instruction, such as enquiry learning, the scientific method and problem solving skills.

Starting from the philosophy of pragmatism of Peirce and James, progressivism became a serious contender for perennialism and essentialism in the 1920's, opposing their extreme authoritarian positions. As an educational practice, they grew larger with cognitivism and constructionism, where enquiry learning developed further and proved to be a more meaningful way of education. Yet, this approach was also criticised by the traditionalists, because it lacked rote learning and therefore could not be controlled, and was deemed highly inefficient for the students had to find out the wheel over and over again. However, progressivists argued that discovering truth is a very important part of learning, for it makes it meaningful and independent of an authoritarian truth. This idea of knowledge transmission also sprouted the idea of constructivism, a movement very close to progressivism.

\section{Reconstructivism}

There are a lot of similarities between progressivism and reconstructivism, such as both subscribing to experimentalism, moral and epistemological relativism, and the goal of improving society instead of conforming to society. Yet, reconstructivists differ from progressivists in the sense that they are more concerned with the ends than the means. Their goal is not to teach problem solving, but rather problem solving itself, and that society should be repaired. This emphasises the idea that the current society is broken, and focuses on social problems such as inequalities.

One might conclude that reconstructivism is thereby not different from the traditional perennialism and essentialism, because these philosophies also focus on the ends rather than the means. However, these philosophies still assume that the truth is absolute, unchanging, and provided by previous generations, whereas reconstructivism is still rooted in experimentalism and thereby states that the truth has to be discovered using the scientific method.

Reconstructivism stems from critical pedagogy, which is again based on postmodernism, anti-racism, feminism, and queer theories. This was first described by Paulo Freire, who was an educator and philosopher fighting for the less fortunate against the Brazilian dictatorship. Critical pedagogy was also applied in other countries with problems of social injustice and poverty, such as the Philippines and South-Africa during the apartheid. Reconstructivism was then created by Theodore Brameld, who advocated for using it in the US for avoiding facism and fighting the still prevalent institutionalised racism.

\section{Existentionalism}

Out of all described educational philosophies, existentionalism differentiates itself the most. Its core direction is towards individual self-fulfillment, and views education as an instrument for encouraging individual choice and autonomy. Not only does it oppose current authority, but it even goes far enough to state that there should be no authority, and that nobody should decide for students what to learn. It also states that what a person is capable of knowing and experiencing is more important than what he knows.

The main method of existentionalism is to put students into situations where they have to make meaningful choices, and to let them confront them alone in order to overcome personal crises so he develops selfreliance and overcomes despair. These are completely different from the methods used by other philosophies, since they do not rely on values preexistent to actions and thereby merely waiting to be discovered.

Existentionalism has seen the least progress in comparison to the aforementioned philosophies, both because of its relative novelty and its radical difference in methodology. It is also the philosophy which is most difficult to implement in current schools. One could even argue that existentialists are opposed to institutionalised education, since it revolves around self discovery and has a very anti-authoritarian viewpoint in the sense that no one should have the authority on deciding what students have to learn. One might argue that democratic schools are a form of an existentionalist curriculum, since here the students get to vote on the content they get to learn, and this school teaches democracy not from theory, but by experience. However, it is not a full realisation, for students do not learn by overcoming personal crises. Another form could be the Dutch \emph{Iederwijs}, a school where students are placed together in a learn-friendly environment and are allowed to do whatever they please. However, this \emph{laissez-faire} method of education still does not challenge the students in any way, which still would be part of existentionalism.

\section{Discussion}

\begin{table}[]
    \centering
    \resizebox{\textwidth}{!}{%
        \begin{tabular}{|p{2.5cm}|p{2.5cm}|p{2.5cm}|p{2.5cm}|p{3cm}|p{2.5cm}|}
            \hline
            \textbf{Educational Philosophy} & \textbf{Perennialism}                             & \textbf{Essentialism}                                                             & \textbf{Progressivism}                    & \textbf{Reconstructivism}                    & \textbf{Existentialism}                            \\ \hline
            \textbf{Function of knowledge}  & As a purpose on itself                            & In order to function in society                                                   & In order to improve society               & In order to change society                   & In order to discover oneself                       \\ \hline
            \textbf{Purpose of education}   & Preserving knowledge                              & Supplying knowledge                                                               & Supplying tools for discovering knowledge & Supplying tools for discovering inequalities & Encouraging maximum individual choice and autonomy \\ \hline
            \textbf{Philosophies}           & Classical idealism, realism                       & Idealism, realism                                                                 & Experimentalism                           & Experimentalism                              & Existentialism                                     \\ \hline
            \textbf{Subject matter}         & Classical literature                              & Three R's                                                                         & Scientific method                         & Social problems                              & Personal reflection                                \\ \hline
            \textbf{Methodology}            & Memorisation, reading, writing, drill, recitation & Reading, lectures, memorisation, repetition, audio-visual materials, examinations & Problem solving                           & Problem solving                              & Subjecting students to crises                      \\ \hline
            \textbf{Authority}              & Ancient works                                     & Teacher                                                                           & Science                                   & Socialists                                   & Student                                            \\ \hline
        \end{tabular}%
    }
    \caption{A comparative summary on the five educational philosophies \protect\cite{educationalphilosophy5}}
    \label{philosophies}
\end{table}

Table~\ref{philosophies} shows a comparitive summary on all above mentioned philosophies, giving an indication on the growing perspective on knowledge and learning methodology throughout history. In general the older philosophies, perennialism and essentialism, are labeled as the traditional philosophies, whereas the other three, progressivism, reconstructivism, and existentionalism, are often labeled as the modern philosophies. These two groups have the most apparent clashes: traditionalists place most trust in the current authorities where the modernists oppose them; traditionalists emphasise rote memorisation where modernists emphasise enquiry; and traditionalists want students to conform to society where modernists want students to change it.

Comparing these two general paradigms with the tools investigated within this thesis, the drill and practice used by the flashcards is most advocated for by the traditionalists, whereas the constructionist concept mapping technique fits mostly to the enquiry practice of the modernists. Flashcards are used by perennialists to memorise data such as dates and reproduction questions, and even more so by essentialists for drilling facts such as multiplication tables and spelling. Concept maps however would be used to shift the attention towards the meaning behind the surface concepts: progressivists use them to discover the ever expanding scientific body of knowledge, reconstructivists for demonstrating historical causality behind social inequalities and how these could be countered, and existentialists to let students map out their own experience and knowledge. However, this preference is not absolute, perennialists could for example also use concept mapping in order to let students figure out the arguments of Socrates in a philosophy assignment (an argument map), and a modernist could still use flashcards for drilling vocabulary.

It is important to consider the five educational philosophies when attempting to succesfully develop the new learning tool flashmaps which combines the flashcards and concept maps. For example, one might ask themselves the questions `what are the benefits of concept map visualisation of flashcards for essentialists' or `why would an existentialist want to memorise the concept map', but also more practical questions such as `should the concept map be provided to or constructed by the students' or `in which order shoud the student traverse through the map'. These are questions which have to be addressed during the design and development of the new tool.
