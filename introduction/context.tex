\chapter{Context}

\section{Importance of knowledge}

As can be read in the previous chapter, the aim of this study is to develop and evaluate a tool designed for the purpose of meaningful memorisation. However, why is it actually important to memorise? This question has historically been debated since the days of the early Greek philosophers, and still remains relevant today. Therefore, before delving into the effectivity and specifications of the tool itself, it seems important to briefly reflect on this question first. This chapter does not aim to answer this age-old question, but rather tries to provide both some philosophical and historical context, for better understanding of the relevance of a better memorisation tool, and what `better' generally entails.

Curriculum theorisers have proposed many different systems of categories \cite{curriculumtheory}, of which the aim is to investigate which goals people involved with education have, and which aspects they therefore regard as being important. \citeA{educationalphilosophy5} differentiates between the five philosophies of education \emph{Perennialism}, \emph{Essentialism}, \emph{Progressivism}, \emph{Reconstructionalism}, and \emph{Existentionalist education}, which have also (at least partly) been acknowledged by other authors \cite{educationalphilosophy, educationalphilosophy2, educationalphilosophy3, educationalphilosophy4}. Furthermore, \citeA{educationalphilosophy3} have found these categories to be sufficiently valid and reliable upon measuring their prevalence among teachers. Therefore, these categories will be discussed further individually in order to provide philosophical context towards the function of knowledge.

\subsection{Perennialism}

Perennialists base themselves mainly on classical idealism and realism, which entails that there is an consistent and unchanging set of principles and traditions. Perennialism is thereby the most conservative of the five philosophies, and emphasises the importance of memorising all these principles and traditions. The subject matter is mainly derived from those works that have survived the centuries, and focus on reading the classical work and disciplining the mind. Many perennialists also believe however that this education should be directed towards the intellectually gifted, and that others would be benefitted more by being provided with vocational education \cite{educationalphilosophy5}. Perennialists are also proponents of studying the greater works, such as the Great Books program by Robert M. Hutchins and Mortimer Adler.

Knowledge according to perennialists is thereby said to be importand \emph{an sich}, without there having to be a reason to learn it.

\subsection{Essentialism}

The essentialists are similar to the perennialists in the sence that their philosophy is mainly based on idealism and realism, however the focus is different. The role of education is not to teach for the sake of teaching, but it is directed towards real-world application instead. 

\subsection{Progressivism}

\subsection{Reconstructionalism}

\subsection{Existentionalist education}

\subsection{Discussion}

\section{Approaches to learning}

%Something about the historical debate about knowledge (scholastics, empericists, humanists)

\subsection{Behaviourism}

\subsection{Cognitivism}

\subsection{Constructivism}
