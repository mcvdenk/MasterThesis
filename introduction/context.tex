\chapter{Context}

\section{Importance of knowledge}

As can be read in the previous chapter, the aim of this study is to develop and evaluate a tool designed for the purpose of meaningful memorisation. However, why is it actually important to memorise? This question has historically been debated since the days of the early Greek philosophers, and still remains relevant today. Therefore, before delving into the effectivity and specifications of the tool itself, it seems important to briefly reflect on this question first. This chapter does not aim to answer this age-old question, but rather tries to provide both some philosophical and historical context, for better understanding of the relevance of a better memorisation tool, and what `better' generally entails.

Curriculum theorisers have proposed many different systems of categories \cite{curriculumtheory}, of which the aim is to investigate which goals people involved with education have, and which aspects they therefore regard as being important. \citeA{educationalphilosophy5} differentiates between the five philosophies of education \emph{Perennialism}, \emph{Essentialism}, \emph{Progressivism}, \emph{Reconstructionalism}, and \emph{Existentionalist education}, which have also (at least partly) been acknowledged by other authors \cite{educationalphilosophy, educationalphilosophy2, educationalphilosophy3, educationalphilosophy4}. Furthermore, \citeA{educationalphilosophy3} have found these categories to be sufficiently valid and reliable upon measuring their prevalence among teachers. Therefore, these categories will be discussed further individually in order to provide philosophical context towards the function of knowledge.

\subsection{Perennialism}

According to perennialism, there is no alteriar motive for attaining knowledge, but rather that attaining knowledge is a purpose on itself. This is along the words of Socrates, who concluded that knowledge is the only virtue. This he concluded based on that wisdom is the same as knowledge \cite{wisdomknowledge}, that wisdom is one of the five cardinal virtues, and that all other virtues (e.g. justice) are merely derived from the virtue of wisdom.

The perennialists are mainly based on either the general philosophy of idealism or of realism. The most notable idealist perennialist are the scholastics, who controlled the academic world from 1100 until the enlightenment of 1700, and focused on teaching the great classical and religious works in order to better understand the Supreme Being. Realist perennialists believe the classic works still have much implications today, and therefore should be thaught to the next generation.

Methods generally practiced by are considered to be rather traditional, example of these are memorisation, reading, writing, drill, and recitation. It is also the only philosophy which has many of its followers believing that education should be directed towards the intellectually gifted, and that other students should only receive vocational education.

\subsection{Essentialism}

Essentialism is generally seen as a child philosophy of perennialism, and is more goal oriented than its parent. Its purpose is to pass on knowledge to new generations in order for them to be able to function in society, and focuses on subject matter.

This philosophy also is based on both idealism and realism, whereas the idealists think the content comes from history, language and the classics, and the realists think it comes from the physical world, including mathematics and the natural sciences.

Just like perennialism, essentialist teaching methods are rather traditional, and include returning to the three R's, reading, lectures, memorisation, repetition, audio-visual materials, and examinations.

\subsection{Progressivism}

Progressivism go one step further than essentialists in a sense that new students should not only be taught to function in society, but to go beyond and improve society. This might seem like a small step, but is rather involved for it has its base in opposing authorianism instead of conforming to it.

It also has its root philosophy in experimentalism, where truth is not constant such as in idealism or realism, but rather is constantly in transition to a better understanding. Therefore, a progressivist curriculum focuses itself not on teaching already existing knowledge, but rather on the methods existing to discover knowledge such as the scientific method. This does not mean however that knowledge has become irrelevant. Students still have to be brought up to date with the newest developments in their field of interest, and thereby there is still some knowledge transfer necessary. The only difference is that this knowledge is never taught to be final, and the focus still lies within the transition and the parts still unknown.

Progressivists generally use more generative methods for instruction, such as enquiry learning, the scientific method and problem solving skills.

\subsection{Reconstructivism}

There are a lot of similarities between progressivism and reconstructivism, such as both subscribing to experimentalism, moral and epistemological relativism, and the goal of improving society instead of conforming to society. Yet, reconstructivists also differ from progressivists in the sense that they are more concerned with the ends than the means. Their goal is not to teach problem solving, but rather problem solving itself, and that society should be repaired. This emphasises the idea that the current society is broken, and focuses on social problems such as inequalities.

One might conclude that reconstructivism is thereby not different from the traditional perennialism and essentialism, because these philosophies also focus on the ends rather than the means. However, these philosophies still assume that the truth is absolute, unchanging, and provided by previous generations, whereas reconstructivism is still rooted in experimentalism and thereby states that the truth has to be discovered using the scientific method.

\subsection{Existentionalism}

Out of all described educational philosophies, existentionalism differentiates itself the most. Its core direction is towards individual self-fulfillment, and views education as an instrument for encouraging individual choice and autonomy. Not only does it oppose current authority, but it even goes far enough to state that there should be no authority, and that nobody should decide for students what to learn. It also states that what a person is capable of knowing and experiencing is more important than what he knows.

The main method of existentionalism is to put students into situations where they have to make meaningful choices, and to let them confront them alone in order to overcome personal crises so he develops selfreliance and overcomes despair. These are completely different from the methods used by other philosophies, since they do not rely on values preexistent to actions and thereby merely waiting to be discovered.

\subsection{Discussion}

\section{Approaches to learning}

%Something about the historical debate about knowledge (scholastics, empericists, humanists)

\subsection{Behaviourism}

\subsection{Cognitivism}

\subsection{Constructivism}
